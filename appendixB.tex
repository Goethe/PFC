\chapter{Pruebas y contenidos adicionales}

Este ap�ndice no debe adjuntarse en el documento final, en el se anotar�n ideas para modificar el contenido del grueso del documento y bases para generar los contenidos adicionales.


\section{Configuraci�n de p�gina}

La configuraci�n actual se ha hecho utilizando el paquete \textsf{typearea} que forma parte del conjunto del \textsc{koma}-\textsc{s}cript. Las opciones m�s importantes que deben pasarse a este paquete son el tipo de papel (A4, A3, A5, letter,\dots) y el espacio de correcci�n necesario para evitar fallos en la encuadernaci�n (\textsc{bcor} = \texttt{magnitud} o directamente \textsc{bcor}\texttt{magnitud}). Utilizando el paquete \textsf{layouts} pueden crearse figuras que representen la configuraci�n de p�gina actual.

\newlength{\auxmm}
\newlength{\auxin}
\newlength{\auxpt}
\setlength{\auxmm}{1mm}
\setlength{\auxin}{1in}
\setlength{\auxpt}{1pt}

\begin{table}[htbp]
	\centering
	\printinunitsof{mm}\pagevalues
	\hspace*{-22pt}\begin{tabular}{>{\raggedright}p{175pt} >{\raggedright\arraybackslash}p{140pt}}
		1in = \printinunitsof{mm}\prntlen{\auxin} & 1pt = \printinunitsof{mm}\prntlen{\auxpt} \\
		1mm = \printinunitsof{in}\prntlen{\auxmm} & 1mm = \printinunitsof{pt}\prntlen{\auxmm}
	\end{tabular}
	\caption[Valores actuales de la distribuci�n de p�gina]{Valores que completan el diagrama representado en la \vref{fig:layouts} sustituir la nomenclatura de referencia por los valores correspondientes}
\end{table}

El lector puede fijarse que se cumple la regla de construcci�n que aplica el paquete \textsf{typearea} en la que el margen inferior es dos veces el margen inferior, y que el margen interior de p�gina ---una vez eliminado el cent�metro (\textsc{bcor}\texttt{1cm}) que se deja para compensar el encuadernado--- es la mitad del margen exterior de p�gina. De ese modo se crea una distribuci�n de p�gina en la que el margen interior de p�gina de las p�ginas par e impar juntas es igual a cada uno de los m�rgenes exteriores de ambas p�ginas.

\begin{figure}[htbp]
	\pagediagram
	\caption{Distribuci�n del texto en las p�ginas de este documento}
	\label{fig:layouts}
\end{figure}

\begin{figure}[htbp]\ContinuedFloat
	\currentpage
	\pagedesign
	\caption[]{Continuaci�n del \vref{fig:layouts}}
\end{figure}


\section{Gr�ficos con MetaPost}

Anotaciones destinadas a obtener gr�ficos escalables de calidad empleando el paquete MetaPost para la creaci�n de gr�ficos en PostScript.


\subsection{Perspectiva isom�trica}

La perspectiva isom�trica o caballera pueden, en teor�a, conseguirse aplicando una transformaci�n a los objetos protagonistas del gr�fico, como circunferencias y cuadrados. Esta transformaci�n puede almacenarse en una variable que se crear� a tal efecto en el fichero \textsf{*.mp}. A diferencia de estas dos perspectivas, la perspectiva c�nica no puede obtenerse de este modo ya que la proyecci�n en el plano depende de la posici�n del espectador y de la posici�n del objeto proyectado.\par
Finalmente se demuestra que a partir de las transformaciones b�sicas de que dispone MetaPost tampoco es posible obtener la perspectiva caballera o isom�trica de un trazo, s�lo de puntos aislados. Para obtener la perspectiva caballera de una circunferencia puede aplicarse el m�todo tradicional, sin embargo el resultado obtenido con curvas Beizer dista de ser aceptable. Parece conveniente hallar la expresi�n correspondiente a la elipse que es la forma que adquiere la circunferencia en dicha perspectiva.
