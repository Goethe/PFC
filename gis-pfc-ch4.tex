\chapter{Resultados y conclusiones}\label{chap:part1conclusions}

\section{Prueba de funcionamiento}\label{sec:working-test}

En el apartado anterior se han dado las claves necesarias para desarrollar
un software como el empleado en este proyecto, no obstante, en este
documento no se contempla entrar con mayor detalle en el código
desarrollado más allá de lo que hasta ahora se ha hecho. Para obtener mayor
información sobre las características y funcionalidades de la aplicación
desarrollada el lector puede remitirse al \vref{chap:appendixA}, en el cual
se ha desarrollado lo que podría considerarse un manual de usuario para
dicha aplicación.

Para comprobar el buen funcionamiento del dispositivo de adquisición
analógica y, en general, del sistema de representación de señales, se ha
diseñado un experimento que pretende evaluar el comportamiento del conjunto
por medio de la comparación, utilizando como referencia un osciloscopio de
laboratorio.

%	El párrafo de abajo es complicado de leer y no da una idea clara de
%	lo que quiero expresar. Más o menos lo que hago es aprovechar el
%	montaje que hago para la prueba y matar dos pájaros de un tiro,
%	certificar que el aparato funciona y dejar por escrito los pasos
%	que deben seguirse al realizar cualquier experimento con el mismo.
%	No creo que sea la mejor idea. Puedo utilizar algún artículo de la
%	web de Tom's Hardware como muestra de como tengo que exponer el
%	experimento: una introducción en la que digo lo que estoy evaluando
%	del aparato (si pueden manejarse de tanta frecuencia, que formas
%	reconoce, como funciona el trigger en distintas situaciones de
%	offset y desfase.. .); después debo dar una tabla con la
%	configuración básica de los experimentos (tipos de señales
%	empleadas, frecuencias, referencias a las fuentes, quizá sondas..
%	.); características vistas en los experimentos del dispositivo y de
%	la referencia; conclusiones que se extraen.
%
%	Tom da también, entre la intro y la configuración de la prueba, una
%	breve descripción de las placas y una comparación de las mismas
%	(puertos disponibles de tal tipo, controladores de lo otro.. .)
%	quizá pueda hacer algo parecido, decir cuantas sondas pueden
%	conectarse al osciloscopio, velocidades de muestreo, y ese tipo de
%	cosas.

Para completar esta parte del documento se ha considerado conveniente la
inclusión de éste apartado cuyo objetivo es cubrir desde un punto de vista
práctico cual es el procedimiento a seguir durante una prueba que se
realice mediante el sistema propuesto en este \sig{pfc} y cuales son los
resultados de dicha prueba en comparación con los de un osciloscopio de
laboratorio. La opción elegida para dar fin a tal propósito es la de
documentar un ejercicio estándar a modo de ejemplo práctico.


\subsection{Montaje del dispositivo de medida}

Puede completarse el montaje del sistema siguiendo un sencillo
procedimiento que contempla una serie de pasos expuestos a continuación.

\begin{enumerate}
	\item En primer lugar debe conectarse la interfaz ampliada
		desarrollada para el proyecto (véase el
		\vref{subsec:conbox}) a los dos conectores mini-\sig{d}
		provistos por la \kpci{}. Los dos conectores deben
		encontrarse en la parte posterior de la carcasa del
		ordenador al que se haya conectado la tarjeta. Los
		conectores hembra de la tarjeta y los conectores macho de
		los cables de la caja de conexiones están etiquetados con
		las palabras \emph{analog} y \emph{digital}, deben
		conectarse los cables de forma que las etiquetas en los
		conectores coincidan.
	\item Después es necesario establecer conexiones entre las fuentes
		de las señales que se quiera intervengan en el proceso de
		adquisición analógica y la caja de conexiones. Para ello se
		recomienda emplear sondas terminadas por un lado en
		conector macho banana o coaxial y por otro en conector de
		tipo cocodrilo. La terminación en conector de tipo banana o
		coaxial debe insertarse en uno de los conectores tipo
		hembra que la caja de conexiones dispone a tal efecto, por
		ejemplo, en el conector que comunica con el puerto
		analógico \can{ch00} de la tarjeta\footnote{Se recomienda
		en cualquier caso emplear los puertos analógicos \can{ch00}
		y \can{ch08} de la \kpci{} ya sea desde la interfaz banana
		o desde la interfaz coaxial.}. Y la terminación en
		cocodrilo a la fuente de la señal analógica de interés.
	\item Tras el paso anterior debe encenderse el ordenador y, una vez
		se ha accedido al sistema, lanzar \matlab{}.
	\item Por último, ya en la consola de \matlab{}, acceder al
		directorio en el que se encuentren los archivos
		\func{single-channel.m} y \func{single-channel.fig} y
		lanzar la aplicación de control.
\end{enumerate}
