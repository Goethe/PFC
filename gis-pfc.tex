\documentclass[a4paper, 12pt]{book}
\includeonly{gis-pfc-pro, gis-pfc-ch1, gis-pfc-ch2, gis-pfc-ch3-1, gis-pfc-appa, gis-pfc-appb}
\usepackage[paper=a4, BCOR1cm]{typearea}
\usepackage[english, spanish, es-ucroman]{babel}
\usepackage[format=hang, font=small, labelfont=bf, labelsep=endash, width=.9\textwidth]{caption}
\usepackage[listofformat=subsimple, format=hang, labelfont=footnotesize, %
	textfont={it, footnotesize}]{subfig}
\usepackage[color]{changebar}
\usepackage[latin1]{inputenc}
\usepackage[OT1]{fontenc}
\usepackage[bf, sf]{titlesec}
\usepackage{titlesec}
\usepackage[nottoc]{tocbibind}
\usepackage[spanish]{varioref}
\usepackage{graphicx}
\usepackage{amsfonts, amsmath, amssymb, calc, color, fancyhdr, hyperref, ifthen, layouts, %
	listings, nextpage, textcase}
\usepackage{array, booktabs, tabularx, tabulary, dcolumn, multirow, threeparttable}
\usepackage[fixlanguage, noisbn]{babelbib}
\titleformat{\part}[display]{\centering\huge\sffamily\bfseries}{\partname\ \thepart}%
	{.5\baselineskip}{}
\hypersetup{%
	pdftitle=Proceso de adquisici�n y tratamiento de se�ales de %
	ultrasonidos, pdfauthor=Jos� Ram�n Gisbert Valls, pdfcreator=Vim 7.2,%
	pdfstartview=FitH, colorlinks=true, pdfdisplaydoctitle=true,%
	naturalnames=true, breaklinks=true, linkcolor=black,%
	citecolor=black, bookmarksopen=true, bookmarksopenlevel=1
}
\fancypagestyle{plain}{%
	\fancyhf{}
	\renewcommand\headrulewidth{0pt}
}
\lstloadlanguages{Matlab}
\lstset{language=Matlab, extendedchars=true, breaklines, captionpos=b, %
	morekeywords={addchannel, analoginput, catch, datadqcallback, %
	daqfind, daqhwinfo, errordlg, gcbo, getdata, guidata, localDaqCallback, %
	peekdata, single_channel, start, stop, strcmpi, trigger, try, warning} %
	}
\lstdefinestyle{displayed}{float=htbp, frame=l, tabsize=5, framerule=1.2pt, %
	abovecaptionskip=\medskipamount, rulecolor={\color[gray]{0.8}}, gobble=4, %
	linewidth=.975\textwidth, xleftmargin=.05\textwidth}
\cbcolor{red}
\labelformat{equation}{ecuaci�n~#1}
\labelformat{figure}{figura~#1}
\labelformat{section}{secci�n~#1}
\labelformat{subsection}{secci�n~#1}
\labelformat{subsubsection}{secci�n~#1}
\labelformat{table}{cuadro~#1}
\labelformat{footnote}{#1\protect\iscurrentchapter{\thechapter}}
\setkeys{Gin}{width=.8\textwidth}
% \setkeys{Gin}{scale=1, keepaspectratio=true}
\graphicspath{{./pictures/}{../pictures/}}
\DeclareCaptionLabelFormat{cont}{#1~#2\alph{ContinuedFloat}}
\captionsetup[ContinuedFloat]{labelformat=cont}
\selectbiblanguage{spanish}
\setbtxfallbacklanguage{english}
% Se aplica el siguiente convenio tipogr�fico a la primera aparici�n de una sigla.
\newcommand\psig[1]{\emph{\MakeTextUppercase{#1}}}
% Se aplica el siguiente convenio tipogr�fico a siglas.
\newcommand\sig[1]{\textsc{\MakeTextLowercase{#1}}}
% Se aplica el siguiente convenio tipogr�fico a funciones.
\newcommand\func[1]{\texttt{#1}}
% Se aplica el siguiente convenio a argumentos de funci�n distintos de propiedades.
\newcommand\argu[1]{\texttt{#1}}
% Se aplica el siguiente convenio tipogr�fico a propiedades.
\newcommand\prop[1]{\textsf{#1}}
% Se aplica el siguiente convenio a atributos de propiedades.
\newcommand\atr[1]{\textsf{#1}}
% Se aplica el siguiente convenio tipogr�fico a nombres de canal o puerto.
\newcommand\can[1]{\sig{#1}}
\newcommand\matlab{\sig{MATLAB}}
\newcommand\kpci{\sig{KPCI}-3108}
\newcommand\datx{\sig{DAT}}
\newcommand\gui{\sig{GUI}}
\newcommand\guide{\sig{GUIDE}}
\newcommand\pc{\sig{pc}}
\newcommand\ram{\sig{ram}}
\newcommand\iscurrentchapter[1]{\ifthenelse{\equal{#1}{\thechapter}}{  en el Cap�tulo~#1}}
\newcommand\miniit[1]{\begin{itemize} #1 \end{itemize}}
\newcommand\tnotetext[2]{\item\hspace*{-5pt}\raisebox{.7ex}%
	{\fontsize{7pt}{8.4pt}\selectfont\normalfont #1}{\footnotesize #2}}
\newenvironment{TableNotes}{\begin{tablenotes}\vspace{\footnotesep}%
	\footnoterule\vspace{.2ex}}{\end{tablenotes}}
\renewcommand\cleardoublepage{\cleartooddpage[\thispagestyle{empty}]}
\newcolumntype{d}[1]{D{,}{,}{#1}}
\newcolumntype{+}{D{/}{\mbox{\ --\ }}{5}}
\begin{document}

\fancyhf{}
\cfoot{\sffamily\bfseries\thepage}
\renewcommand\headrulewidth{0pt}
\pagestyle{fancy}

\frontmatter

\tableofcontents

\listoftables

\listoffigures

\chapter{Pr�logo}

\section*{Justificaci�n}

\subsubsection{Ensayos no destructivos con ultrasonidos}

Es en la pr�ctica habitual en la actualidad emplear los \emph{ensayos no destructivos} (\psig{end}) en controles de calidad efectuados en la industria de manufactura de materiales, principalmente de metales y de compuestos para la construcci�n. Este tipo de ensayos garantizan ---demostrada su efectividad en este tipo de aplicaciones--- la ausencia de defectos internos como fisuras y de otras imperfecciones como alteraciones en la composici�n del producto, que de no superar controles semejantes pueden pasar inadvertidos. Realizando estos controles se impide la salida al mercado de partidas de producto defectuosas que pudiesen comprometer la calidad de productos finales. En este sentido comprobar el estado de un material resulta en un ejercicio de responsabilidad puesto que muchos de los materiales que proceden de esta industria se destinan a la construcci�n de edificios o de medios de transporte como barcos y aviones. De ah� la necesidad de aplicar en los controles de calidad m�todos que proporcionen resultados efectivos y confiables como lo son los obtenidos con los \sig{end}.\par
La ventajas que presentan los ensayos no destructivos frente a otro tipo de ensayos, no s�lo se limitan al hecho de que no es necesario sacrificar parte de la producci�n para evaluar el producto, si no que tambi�n incluyen una evaluaci�n continua y uniforme del material. Esto redunda en materiales de gran calidad cuyas propiedades se mantiene invariables uniformemente a lo largo del producto. Adem�s, otra ventaja de los \sig{end} es que es posible realizar de nuevo el ensayo una vez el producto final se ha terminado o en posteriores revisiones ya que no causa ning�n da�o en el material. Por �ltimo, cabe destacar que los controles de calidad mediante \sig{end} pueden efectuarse durante el mismo proceso de fabricaci�n de forma automatizada lo que supone una reducci�n en los costes.\par
Por su parte, el uso de ultrasonidos en \sig{end} est� muy extendido, esto es en parte debido a los precisos resultados que proporcionan los \emph{ensayos no destructivos mediante ultrasonidos} (\psig{endus}). Empleando transductores de alta frecuencia y gran ancho de banda es posible distinguir defectos en el material de tama�o muy peque�o de forma inequ�voca. A esto debe sum�rsele la sencillez con la que se aplican los \sig{endus} y las ventajas de emplear transductores de peque�o volumen que permiten trabajar con materiales de formas intrincadas e irregulares.\par
Recientemente se est�n empleando los \sig{endus} en campos experimentales con el prop�sito de detectar la presencia de anomal�as en materiales procedentes de la naturaleza. En estos experimentos el procedimiento a seguir es conforme al uso de los controles de calidad que se practican en el �mbito industrial. Conviene, sin embargo, diferenciar el tipo de medio en el que los ensayos son aplicados. Los ensayos realizados habitualmente a nivel industrial eval�an materiales sint�ticos, bien conocidos, mientras que en su lugar, los ensayos experimentales contemplan materiales org�nicos heterog�neos como es en este caso la madera. Es por ello que la eficacia de los \sig{endus} aplicados sobre materiales org�nicos no est� todav�a demostrada.\par
De la singularidad del material el inter�s de las pruebas, inter�s que se ve reforzado por el hecho de que la madera es un medio <<vivo>>, que cambia de muestra a muestra, las condiciones del ensayo var�an incluso para una misma muestra. Es, por tanto, el uso \sig{endus} para la detecci�n de defectos en madera de palmera un tema de inter�s, de actualidad y muy atractivo, hacia el cual orientar el desarrollo de un proyecto fin de carrera.


\subsubsection{Tratamiento digital de se�ales}

El tratamiento digital de se�ales es una de las disciplinas fundamentales que abarca la ingenier�a de telecomunicaciones. Pese a los inconvenientes inherentes al uso de circuiter�a digital (muestreo y errores de cuantificaci�n principalmente) el procesado digital de se�ales ha demostrado ser una herramienta �til y vers�til.\par
Esa versatilidad se sigue de la capacidad del hardware digital de ser programado, o lo que es lo mismo la capacidad de un circuito de alterar su funcionamiento y proporcionar varias funciones de acuerdo con una configuraci�n por software. Virtud que sumada al reciente desarrollo de las tecnolog�as digitales ha propiciado su explotaci�n y el uso extensivo de aplicaciones basadas en circuiter�a digital.\par
La configuraci�n de un sistema digital a partir de una tarjeta de adquisici�n digital, un ordenador y una suite de software matem�tico supone una oportunidad para comprender aspectos del funcionamiento de los sistemas digitales dif�ciles de observar en la teor�a.


\section*{Objetivos del proyecto fin de carrera}\label{sec:goals}

Inicialmente este proyecto persigue completar dos objetivos distintos: la implementaci�n de un sistema de medida a partir de una tarjeta de adquisici�n con interfaz \sig{pci}; y la evaluaci�n de los \sig{endus} como m�todo para la detecci�n de defectos en madera de palmera.\par
Por un lado, se ha llevado a cabo la puesta en funcionamiento de un sistema de medida con el que es posible obtener un conjunto de par�metros de una determinada se�al. Permite observar la forma que presenta la se�al en cada instante, la forma del espectro de la se�al cuando se enmarca en una ventana de una determinada duraci�n temporal, valores instant�neos y valores medios en un determinado intervalo.\par
Por otro lado, se han llevado a cabo una serie de pruebas experimentales en madera de palmera, empleando para ello se�ales ultras�nicas generadas por un transductor, de forma similar a como se hace en los \sig{endus} de car�cter industrial. Posteriormente las muestras se han procesado para averiguar a partir de los datos que informaci�n sobre el medio ---presencia de fisuras internas, presencia de agua en distintos niveles dependiendo de la altura en la que se realiza la prueba o del estado de salud del esp�cimen, o en general cualquier otra informaci�n de utilidad--- puede extraerse en este tipo de ensayos.\par
Existe una relaci�n visible entre ambos objetivos, la pretensi�n inicial consiste en emplear el sistema de medida digital en la realizaci�n de las pruebas. No obstante, pese a todo, durante el transcurso del proyecto ambos objetivos se tratan por separado. Finalmente, la imposibilidad de realizar las pruebas experimentales con el sistema confeccionado (por motivos que se ver�n m�s adelante) ha conducido al uso de un sistema de medida alternativo. Es por este motivo por el cual resulta comprensible la divisi�n de este documento en dos partes, una por cada objetivo diferenciado.


\section*{Estructura del documento}

Como se ha mencionado en el apartado anterior el documento consta de dos partes. Se ha decidido ordenar las partes en orden cronol�gico. Puesto que ha resultado necesario confeccionar el sistema de medida con anterioridad a la realizaci�n de las primeras pruebas ---esto resulta obvio, pues sin este sistema no es posible realizar las mediciones necesarias---, la primera de las partes trata sobre la implementaci�n del sistema digital de medida. La divisi�n de las partes no es, sin embargo, natural por completo, puesto que para dise�ar el instrumento de medida es necesario haber reunido previamente un determinado conocimiento en materia de \sig{endus}. Todo lo relativo a la teor�a de los \sig{endus} y a los resultados de las pruebas se ha reservado para la segunda parte de la memoria.

\begin{itemize}
	\item La primera parte se divide a su vez en cuatro cap�tulos: subsistema de interacci�n con el medio f�sico; subsistema de adquisici�n; subsistema de control y presentaci�n; y resultados, conclusiones y l�neas futuras de trabajo. Los tres primeros cap�tulos responden a una divisi�n funcional del sistema de medida, observando el sistema de medida como una pila de capas en el que cada capa comprende un subsistema que proporciona servicio a los subsistemas por encima de �l y abstrae las capas inferiores, es posible tratar cada subsistema por separado.
		\begin{itemize}
			\item El primer subsistema interact�a directamente con el medio f�sico, comprende el transmisor de ultrasonidos, el receptor y las etapas de acondicionamiento que los preceden. El primer cap�tulo resume las principales caracter�sticas de estos tres elementos y realiza un recorrido por el proceso de dise�o de este subsistema.
			\item El subsistema intermedio transforma la se�al anal�gica que le es entregada por la secci�n de recepci�n en una se�al digital. El n�cleo y el todo de este subsistema es la tarjeta de adquisici�n. El segundo cap�tulo repasa las caracter�sticas t�cnicas clave de este dispositivo, realiza una descripci�n funcional del mismo y reproduce los consejos que el fabricante proporciona en el manual de usuario para su uso correcto.
			\item El �ltimo de los subsistemas, el subsistema de m�s alto nivel, es el subsistema de control y presentaci�n. Este subsistema interviene como interfaz entre el usuario administrador y el resto de capas (el resto del sistema de medida). Como tal es su funci�n traducir la se�al digital que recibe de capas inferiores en informaci�n �til para el usuario, recibir e interpretar los comandos administrados por el supervisor y gestionar el funcionamiento del resto de subsistemas. El cap�tulo tercero de la primera parte hace hincapi� en el dise�o conceptual de este subsistema y proporciona una serie de detalles t�cnicos en relaci�n con el entorno de programaci�n en el que se ha desarrollado.
			\item En el �ltimo de los cap�tulos de la primera parte se exponen las conclusiones extra�das tras poner a prueba el sistema de medida ya terminado. Se comentan posteriormente l�neas futuras de trabajo que persiguen mejorar la funcionalidad del sistema de medida.
		\end{itemize}
	\item La segunda parte est� dividida en tres cap�tulos: fundamentos te�ricos de los \sig{endus}, descripci�n del medio, resultados y conclusiones.
		\begin{itemize}
			\item En el primer cap�tulo se describen de forma superficial los distintos elementos que intervienen en un \sig{endus} desde un punto de vista te�rico. De las distintas t�cnicas existentes para combatir el ruido estructural se dan detalles sobre las t�cnicas de procesado por partici�n del espectro que son las utilizadas en este proyecto.
			\item El segundo cap�tulo trata sobre el medio en el que se realizan las pruebas experimentales, la madera de palmera. Se proporciona una descripci�n te�rica del material de acuerdo con la documentaci�n consultada. Se caracteriza el material para un posterior an�lisis de los resultados encontrados en las pruebas experimentales.
			\item El �ltimo cap�tulo recoge los resultados extra�dos de las pruebas en forma de grafos y tablas comentados. Despu�s se dan las conclusiones a las que se ha llegado a partir de estos resultados y finalmente se realizan una serie de comentarios sobre nuevas l�neas de investigaci�n que pueden seguirse de este proyecto
		\end{itemize}
	\item Dos ap�ndices cierran este documento: el primero es un manual de instrucciones para el uso del sistema de medida, y el segundo es un ap�ndice que muestra detalles sobre este documento.
		\begin{itemize}
			\item El primer ap�ndice proporciona un manual de usuario para el uso de la aplicaci�n de software inform�tico que funciona como subsistema de control del sistema digital de medida implementado para este proyecto. Este contenido se ha extra�do del grueso del documento, pues no concierne al desarrollo del proyecto.
			\item El segundo ap�ndice muestra detalles sobre la memoria como por ejemplo la estructura de sus p�ginas, glosarios, �ndices de contenidos y referencias.
		\end{itemize}
\end{itemize}
% Existe una relaci�n visible entre ambos objetivos, la pretensi�n inicial consiste en emplear el sistema de medida en la realizaci�n de las pruebas necesarias en el estudio que persigue el segundo objetivo del proyecto. A pesar de lo cual, el tratamiento que se da a cada objetivo es independiente con respecto del otro/se da un tratamiento por separado a cada objetivo.
% Aunque existe una visible relaci�n entre los objetivos, �stos se plantean por separado y se desarrollan, en la pr�ctica, con independencia uno del otro.
% Aprender a programar dos cosas en matlab: primero una interfaz gr�fica; y, por otro lado, una interfaz con el dispositivo de adquisici�n.
% (puesta a punto, puesta en funcionamiento) de todos los elementos que conforman un sistema de medida convencional (que se ajustan al esquema tradicional/extendido de un sistema de medida)
% Tengo que hacer una prueba de ultrasonidos y tengo un sistema de medida que puede adaptarse a las pruebas que quiero hacer, pero en principio no pienso en utilizarlo para ese prop�sito


\fancyhf{}
\fancyhead[RO, LE]{\sffamily\bfseries\thepage}
\fancyhead[LO]{\sffamily\nouppercase{\rightmark}}
\fancyhead[RE]{\sffamily\nouppercase{\leftmark}}
\renewcommand\headrulewidth{0.4pt}
\pagestyle{fancy}
\renewcommand\chaptermark[1]{\markboth{#1}{}}

\mainmatter

\part{Sistema de adquisici�n y procesado de se�ales}

\chapter{Subsistema de interacción con el medio físico}

\section{Introducción al sistema digital de medida}

Un sistema de medida es cualquier instrumento formado por más de un
elemento que permite a quien lo usa evaluar una determinada propiedad de un
objeto, medio o evento. Para una definición más rigurosa de un sistema
electrónico de medida puede optarse por la encontrada en
\cite{pallas2004sas}, y dice lo siguiente:

{\small\begin{quotation}
	Se denomina sistema a la combinación de dos o más elementos,
	subconjuntos y partes necesarias para realizar una o varias
	funciones. En los sistemas de medida, esta función es la asignación
	objetiva y empírica de un número a una propiedad o cualidad de un
	objeto o evento, de tal forma que la describa.
\end{quotation}}

Un sistema de medida digital está constituido habitualmente por los
elementos mostrados en el esquema de la \cref{fig:digmeasstm}. Pueden
agruparse estos elementos según la función que desempeñan en el sistema, si
se hace de ese modo se encuentran tres subsistemas: un subsistema para la
interacción con el medio físico, un subsistema de adquisición, y un
subsistema de control y presentación. Puede observarse entonces el sistema
como una pila de capas superpuestas, en el que cada capa provee de servicio
a la capa inmediatamente superior y abstrae las capas inferiores, en cada
una de las capas se situaría cada uno de los subsistemas propuestos. De ese
modo cada subsistema puede estudiarse por separado con independencia de los
demás, así se ha hecho en esta memoria. Consecuentemente cada subsistema da
lugar a cada uno de los tres capítulos que empezando por este describen el
sistema digital de medida implementado durante el curso del desarrollo de
este proyecto.

\begin{figure}
	\begin{center}
		\includegraphics{gis-pfc-ch1-01.mps}
	\end{center}
	\caption[Sistema digital de medida] {Distintos elementos
	funcionales que conforman el sistema digital de medida.}
	\label{fig:digmeasstm}
\end{figure}

Si se define una jerarquía en la que el nivel más bajo es aquel ocupado por
los elementos en contacto directo con el medio, y se tiene como nivel más
alto aquel en el que se encuentran los elementos que interactúan con el
supervisor, entonces los niveles ordenados de inferior a superior en la
jerarquía establecida se encuentran ocupados por los subsistemas precisados
anteriormente del siguiente modo: subsistema para la interacción con el
medio, subsistema de adquisición, y subsistema de control y presentación.
El orden en el que se han dispuesto los capítulos que tratan acerca de la
configuración del sistema digital de medida, los tres primeros capítulos,
coincide con el orden que guardan los distintos subsistemas en esta
jerarquía.



Para que un sistema pueda considerarse un sistema de medida debe haber una
interacción con el objeto o evento de estudio. El propósito de este sistema
es medir las propiedades de un determinado medio físico, en este caso
madera de palmera. La interacción con el medio ocurre en el primer
subsistema que se encuentra dentro de la jerarquía de capas que define el
sistema de medida. Atendiendo a su función dentro del sistema, este
subsistema recibe en esta memoria el nombre de subsistema para la
interacción con el medio físico. Es el único de los tres subsistemas que
conforman el sistema de medida que está directamente relacionado con los
ensayos no destructivos, mientras que el resto de subsistemas, el
subsistema de adquisición y el subsistema de control y presentación se
encuentran más ligados al tratamiento digital de señales. En una inspección
ultrasónica, el subsistema para la interacción con el medio físico, este
subsistema, es imprescindible y sin él no puede realizarse el ensayo; sin
embargo, los otros dos subsistemas pueden sustituirse por otros elementos
que realicen la misma función, por un osciloscopio digital, por ejemplo. El
subsistema para la interacción con el medio físico está compuesto por los
elementos presentes en la figura \cref{fig:submedium}. Los transductores
son los dispositivos que realmente interactúan con el medio, a través de la
emisión de una señal acústica que se recibe en el otro extremo. Por su
parte los acondicionadores adecuan las señales de tal modo que, pueda haber
una comunicación entre los elementos presentes en este subsistema y los
pertenecientes a subsistemas por encima de éste.

\begin{figure}
	\begin{center}
		\includegraphics{gis-pfc-ch1-01.mps}
	\end{center}
	\caption[Subsistema para la interacción del medio físico]{Esquema
	que representa los distintos elementos o bloques presentes en el
	subsistema para la interacción con el medio físico.}
	\label{fig:submedium}
\end{figure}

Este capítulo se divide pues en tres apartados, en el primero se hace una
descripción técnica de los transductores utilizados en el proyecto, y se
hace una pequeña observación de su utilidad en relación con el tipo de
ensayo programado para el proyecto. En los dos apartados posteriores se
explica el proceso de diseño de los circuitos acondicionadores de la
sección de transmisión y de la sección de recepción respectivamente, se
exponen los requisitos que deben satisfacer y se explica brevemente el
funcionamiento de los circuitos propuestos para que el lector pueda
comprobar que satisfacen dichos requisitos. En el \cref{chap:endus} se hace
un desarrollo de la teoría de los ensayos no destructivos que puede ayudar
a comprender mejor algunos de los conceptos propuestos a lo largo de este
capítulo.


\section{Transductores de ultrasonidos}\label{sec:transducers}

Los transductores de ultrasonidos son transductores piezoeléctricos y, por
tanto, entran en la categoría de transductores generadores. La principal
característica de un sensor generador es que al ser expuesto a la magnitud
no eléctrica de interés genera una diferencia de potencial eléctrico
proporcional sin necesidad de una alimentación eléctrica. Este efecto es
reversible y puede utilizarse para provocar una respuesta no eléctrica a
una variación en una corriente eléctrica por medio de un actuador. Los
actuadores y los sensores generadores suelen presentar características
similares, sobretodo debido a este motivo.  En determinados materiales
aparece una polarización eléctrica si se somete el material a un esfuerzo
que provoca deformación, este fenómeno se conoce como efecto piezoeléctrico
y es un efecto reversible. Si se aplica una diferencia de potencial
eléctrico a dos de las caras de un material piezoeléctrico, en éste aparece
una deformación.  Un actuador de ultrasonidos vibra cuando se excita con
una onda eléctrica y esta vibración se transmite al medio en forma de onda
de presión. La señal ultrasónica se propaga por el medio y finalmente
incide en el sensor, ejerciendo una presión sobre su superficie que el
transductor traduce en una señal eléctrica.

La principal diferencia entre los transductores de ultrasonidos y los
transductores acústicos convencionales es que los transductores de
ultrasonidos trabajan a frecuencias superiores a los 20 kHz donde la onda
acústica se hace inaudible para el oído humano, de esta manera es posible
transmitir la energía acústica en pulsos estrechos. Los transductores de
ultrasonidos pueden ser de dos tipos dependiendo de como se construyan,
abiertos o cerrados. Los transductores de tipo abierto están formados por
una pantalla protectora y la pieza de material piezoeléctrico montada sobre
un cono metálico, suelen ser más eficientes y presentan un mayor ancho de
banda, no obstante, están menos preparados para trabajar en condiciones
duras y su frecuencia de resonancia se determina con menor precisión. En
transductores de diseño cerrado la pieza resonante está en contacto directo
con la carcasa protectora, y ésta se ajusta durante el proceso de
fabricación para que el conjunto vibre a la frecuencia de resonancia
elegida, estos transductores son menos eficientes pero más resistentes y
más precisos. Las principales características de los transductores
utilizados en el proyecto vienen recogidas en el \cref{tab:transducers}.
% Y su aspecto viene reflejado en la representación que realiza la figura
% 1.3

\begin{sidewaystable}
	\centering
	\begin{threeparttable}
	\begin{tabular}{l c c}
		\toprule
		& \multicolumn{2}{c}{Transductor} \\
		\cmidrule(l){2-3}
		Propiedad & Transmisor & Receptor \\
		\midrule
		Nivel de presión acústica transmitida & 110 dB & --- \\
		Sensibilidad del receptor & --- & -70 dB \\
		Frecuencia fundamental de resonancia
		& $40 \text{kHz} \pm 1 \text{kHz}$
		& $40 \text{kHz} \pm 1 \text{kHz}$ \\
		Ancho de banda a -6 dB & 2.5 kHz & 3.0 kHz \\
		Ancho del lóbulo principal & 60º & \\
		Máxima potencia disipable & 200 mWef & \\
		Impedancia & $700\ \Omega$ & $30\ \text{k}\Omega$ \\
		Capacidad & $2 \text{nF} \pm 20\%$
		& $2 \text{nF} \pm 20\%$ \\
		Tiempo de subida & $700\ \mu\text{s}$ & \\
		Rango de temperatura de funcionamiento %
		& (-20, 60) [ºC] & (-20, 60) [ºC] \\
		\bottomrule
	\end{tabular}
	\begin{TableNotes}
		\tnotetext{*}{Todas las magnitudes de la tabla se dan con
		respecto a la frecuencia natural de resonancia de los
		transductores.}
	\end{TableNotes}
	\end{threeparttable}
	\caption[Características de los transductores empleados en el
	sistema de medida]{Características de los transductores empleados
	en el sistema de medida.}
	\label{tab:transducers}
\end{sidewaystable}

El actuador de ultrasonidos puede funcionar de dos formas diferentes: si es
alimentado con una sinusoide emite una onda acústica continua al medio; por
el contrario, si se alimenta con un pulso rectangular (o de continua),
emite un tono modulado por un pulso gaussiano de la misma duración. En
cualquier caso la potencia disipada por el actuador no puede exceder el
límite de 20 mW eficaces especificado por el fabricante. Dependiendo del
tipo de alimentación es posible adoptar una u otra estrategia para acotar
la potencia suministrada al sensor. Si se alimenta el sensor con una onda
sinusoidal puede, o bien limitarse la tensión de alimentación, o bien
utilizarse una sinusoide pulsada, empleando una configuración de ciclo de
trabajo que garantice que no se supera el límite de potencia soportado. En
caso de utilizar un pulso rectangular, o un tren de pulsos rectangulares,
las opciones son similares, reducir el nivel de tensión o configurar el
ciclo de trabajo de la señal. El actuador de ultrasonidos carga el circuito
de alimentación con una impedancia que presenta un carácter variable con
respecto a la frecuencia de la onda eléctrica y que debe tenerse en cuenta
para calcular la transferencia de potencia. La impedancia equivalente
presenta un comportamiento capacitivo a frecuencias por debajo de la
frecuencia fundamental de oscilación del actuador, carácter resistivo a
frecuencias en torno a la frecuencia de resonancia y carácter inductivo a
frecuencias por encima de esta frecuencia. Por tanto, la máxima
transferencia de potencia ocurre en frecuencias cercanas a la frecuencia
natural de oscilación del actuador, donde éste se comporta como una
impedancia resistiva de un valor típico que ronda los 700 $\Omega$. La
potencia suministrada al actuador se reparte entre la frecuencia
fundamental de resonancia y los distintos armónicos a frecuencias
superiores. El actuador siempre vibra a su frecuencia fundamental de
resonancia o a frecuencias múltiplo de ésta, independientemente de la
frecuencia de la onda eléctrica utilizada para el suministro de potencia.
Si bien esto es cierto, para evitar excitar armónicos secundarios distintos
de la frecuencia fundamental de resonancia es aconsejable que la frecuencia
de la onda eléctrica sea lo más próxima a la frecuencia de resonancia del
actuador, además esto garantiza la máxima transferencia de potencia.

En ensayos no destructivos mediante ultrasonidos es imprescindible utilizar
pulsos acústicos como el representado en la \cref{fig:pulse}, para poder
evaluar las propiedades del medio de propagación empleando una de las dos
técnicas expuestas en el \cref{chap:endus} en la \cref{sec:technics}. Como
se explica en dicho capítulo, el actuador continua vibrando a su frecuencia
fundamental de resonancia incluso después de que la onda de alimentación
pase a un estado bajo, durante un tiempo determinado durante el proceso de
fabricación del dispositivo ---esto es algo que no se aprecia en la
figura---, dando lugar a lo que en teoría de \sig{endus} se conoce como
<<zona ciega>> o <<zona muerta>>, este fenómeno se explica con mayor
detalle en la \cref{sec:field}.

\begin{figure}
	\begin{center}
		\includegraphics{gis-pfc-ch1-04.mps}
	\end{center}
	\caption[Pulso acústico generado por el actuador de
	ultrasonidos]{Pulso acústico que se obtiene del actuador de
	ultrasonidos cuando se alimenta con un pulso rectangular.}
	\label{fig:pulse}
\end{figure}

El parámetro más importante asociado al actuador de ultrasonidos es el
nivel de presión acústica transmitida (\emph{Transmitting Sound Pressure
Level}, o \psig{spl}), representa la presión que la onda acústica ejerce
sobre la presión estática del aire a una distancia determinada del actuador
cuando éste se alimenta con una determinada tensión eficaz. Proporciona una
medida de la eficiencia del sensor y puede utilizarse conjuntamente con la
sensibilidad del sensor para calcular de forma teórica cual es el alcance
máximo de un experimento de ultrasonidos que emplea estos transductores. El
\sig{spl} es una magnitud que varía con la frecuencia de la onda eléctrica,
y alcanza su máximo en la frecuencia natural de resonancia del actuador. El
fabricante suele proporcionar, o bien el valor típico del \sig{spl} a la
frecuencia de resonancia del actuador, o bien un perfil de la magnitud
frente a la frecuencia de la onda eléctrica. Con independencia del tipo de
valor proporcionado, éste suele darse en escala logarítmica utilizando una
para ello una referencia de presión que también se especifica. Para
calcular el \sig{spl} que se da en condiciones distintas de aquellas en que
se miden las especificaciones, debe calcularse la variación en escala
lineal y después trasladar este resultado a escala logarítmica, antes de
poder operar con el \sig{spl}. El sensor de ultrasonidos presenta unas
propiedades muy similares a las propiedades del actuador, si bien se
caracteriza por presentar una impedancia característica bastante mayor. En
el sensor la conversión se da en sentido contrario, las ondas acústicas de
presión que inciden en la cara externa del sensor se traducen en
variaciones de la tensión eléctrica que existe en bornes del dispositivo.
Si la propiedad más característica del actuador de ultrasonidos es el
\sig{spl}, la propiedad más característica del sensor es la sensibilidad.
La sensibilidad suele proporcionarse también en escala logarítmica, igual
que ocurre con el \sig{spl}, suele darse un valor típico o una
característica de la propiedad frente a la frecuencia que, en ocasiones se
superpone con la gráfica del \sig{spl} si los transductores pertenecen a
una misma serie. La magnitud lineal es la relación entre la presión
detectada y la diferencia de potencial eléctrico que se crea en bornes del
sensor y, en este caso, representa una medida de la eficiencia del sensor.

La madera de palmera es un medio peculiar y la atenuación que introduce en
la señal acústica es mucho mayor que la que introduce un medio como el
aire. Trabajar con transductores cuya frecuencia de resonancia se encuentra
localizada en las frecuencias bajas de la región ultrasónica presenta una
serie de ventajas y desventajas. Desde el punto de vista del sistema de
medida digital, la frecuencia de la señal eléctrica que el sensor envía al
subsistema de adquisición no puede superar los 50 kHz, de lo contrario, con
una frecuencia de muestreo de 10 \kms{} la señal digitalizada se vería
afectada por el aliasing. Por otro lado, atendiendo a las propiedades de
propagación de las ondas acústicas, la dispersión afecta en menor medida a
las bajas frecuencias (vid. \cref{chap:endus}), con lo cual resulta
beneficioso trabajar a tan bajas frecuencias. No obstante, las ondas
acústicas de baja frecuencia tienen una tendencia mayor a propagarse por la
superficie de un medio, lo cual provocaría ecos indeseados en recepción que
dificultarían la realización de los ensayos. El verdadero problema de estos
transductores es que emiten una energía ultrasónica muy débil, que no
resulta suficiente para atravesar un tronco de palmera de 50 cm de
diámetro, y esto impide aplicar la técnica de transmisión. Utilizar la
técnica de transmisión con dos transductores resulta complicado y poco
preciso, menos preciso aún con transductores de gama baja, por lo que en
resumidas cuentas los transductores con los que se cuenta para realizar los
ensayos que requiere la segunda parte del proyecto no reúnen las
condiciones para poder realizar correctamente dichos ensayos. Al menos así
lo afirma la teoría, en la práctica (vid. \cref{chap:part1conclusions}) se
comprueba que realmente ocurre así.


\section{Circuito acondicionador de la sección de emisión}

El primer paso en el diseño de un circuito acondicionador es determinar los
requisitos que dicho circuito debe satisfacer. Como se veía en el apartado
anterior en la página \pageref{sec:transducers}, el actuador de
ultrasonidos puede emitir pulsos acústicos pero para ello requiere que la
alimentación sea pulsada, un pulso rectangular. Por tanto, el circuito
acondicionador debe generar un tren de pulsos rectangulares con el que
alimentar el actuador. La tensión eficaz de la alimentación debe ser lo más
alta posible, dadas las condiciones de trabajo ---se requiere una alta
potencia acústica para atravesar un tronco de palmera---, pero siempre sin
superar el límite de 10 $\text{V}_\text{eff}$. Para favorecer la máxima
transferencia de potencia, la impedancia de salida debe ser mínima, de modo
que toda la corriente circule por la carga. De entre todas las opciones
disponibles, se ha optado por un circuito que utiliza dos temporizadores
555 (el encapsulado se556 de Texas Instruments contiene dos de estos
temporizadores) para generar una señal cuadrada cuyo ciclo de trabajo puede
ajustarse utilizando un par de potenciómetros. La principal ventaja que
presenta un circuito de estas características es sólo requiere una fuente
de alimentación para funcionar, puede funcionar incluso con una pila, y no
requiere de un generador de señales. Por tanto, utilizando un diseño como
el propuesto es necesario menos equipo para realizar los ensayos y menos
energía con lo que es más sencillo hacer pruebas de campo. El 555 se
caracteriza por generar pulsos

% Habla de pendientes de subida de ms, de pequeña impedancia de salida, de
% ciclo de trabajo por encima del 50%, que con dos 555 puedo generar un
% tren de pulsos con un ciclo de trabajo por debajo del 50%, habla de la
% tensión en estado alto y de la máxima transferencia de potencia. Habla un
% poco del funcionamiento del 555, funcionamiento como astable y como
% monoestable, de que las resistencias y condensadores controlan el ancho
% del pulso y de que si utilizas un potenciómetro en R1 o R3 puedes hacer
% que cambie el ciclo de trabajo. Muestra el circuito y pídele ayuda a José
% Luis Alonso para mejorar el diseño actual. Mejor adaptación de
% impedancias, mayor transferencia de potencia.

% VAYA TROLACA!!! NO TIENES NI IDEA DE ADAPTACIÓN DE IMPEDANCIAS, la máxima
% transferencia de potencia se obtiene cuando la impedancia de salida del
% circuito acondicionador es la conjugada de la carga. Primero tendrías que
% ver que ocurre con el actuador cuando lo alimentas con una señal
% cuadrada, como se comporta, con que impedancia carga el circuito y
% después debes diseñar un circuito para poder acoplar el sensor al
% temporizador.


\chapter{Subsistema de adquisici�n}

Uno de los objetivos que persigue este proyecto desde el comienzo es constituir un sistema de adquisici�n y procesado de se�ales. La idea es utilizar el sistema a la hora de afrontar la segunda parte del proyecto, o parte principal podr�a decirse, la experimentaci�n con palmeras, con sus troncos y madera, para evaluar la efectividad de los ensayos no destructivos, de esos que emplean los ultrasonidos como herramienta de exploraci�n. Es esta secci�n del documento la que se ocupa en la materia de retratar la experiencia de componer un instrumento que muestrea una se�al anal�gica y proporciona una versi�n digital de la misma, forma que permite su posterior procesado, almacenamiento y representaci�n.\par
La tarjeta \kpci{} adquiere un gran protagonismo en el sistema de adquisici�n y procesado, constituye la base del mismo. Este dispositivo constituye el fundamento a partir del cual se construye el resto del sistema de medida, incluyendo no s�lo la etapa de adquisici�n si no tambi�n las de acondicionamiento y detecci�n, parte de sus capacidades las hereda el sistema, sus limitaciones deben tenerse en cuenta en la fase de dise�o, de ah� la importancia de estudiar a conciencia su comportamiento. Adem�s, la \kpci{}, disponer de ella, es una de las razones principales por las que se ha optado por desarrollar un sistema de adquisici�n propio, de prop�sito, sin tener en cuenta la principal raz�n, como es obvio, contar con un sistema dise�ado a medida.\par
Resulta primordial pues conocer las caracter�sticas de una pieza tan importante del conjunto, de hecho es el primer paso que se toma en la ejecuci�n del proyecto, estudiar el aparato y averiguar hasta d�nde alcanzan sus posibilidades. En la memoria del proyecto queda plasmado el inter�s que suscita la tarjeta desde un punto de vista pr�ctico en relaci�n con la s�ntesis de un sistema de adquisici�n y procesado puesto que se dedica un cap�tulo completo a tratar en clave t�cnica y de forma resumida c�mo funciona la tarjeta y cu�les son sus propiedades m�s destacadas.


\section{Caracter�sticas t�cnicas del hardware}\label{sec:technical}

La tarjeta \kpci{} puede emplearse para la adquisici�n y conversi�n de se�ales anal�gicas en se�ales digitales, para sintetizar se�ales anal�gicas a partir de se�ales digitales previamente generadas o almacenadas, o ---gracias a sus 32 puertos digitales de prop�sito general--- trabajar con se�ales digitales.\par
El primer bloque del sistema electr�nico de medida propuesto, es decir el sistema de adquisici�n y procesado de se�ales, tan s�lo requiere de la funci�n de adquisici�n anal�gica de la tarjeta. Por ello, de entre todas las caracter�sticas del dispositivo, se ha cre�do conveniente resumir a continuaci�n aquellas que tienen relaci�n directa con dicha funci�n. Para obtener informaci�n detallada sobre la relaci�n que estos atributos guardan con el proceso de adquisici�n de se�ales anal�gicas en la tarjeta \kpci{}, v�ase la \vref{sec:funcdesc}.

\begin{itemize}
	\item El m�dulo de adquisici�n anal�gica dispone de 16 puertos f�sicos.
	\item La impedancia de entrada equivalente de cada puerto es aproximadamente igual a una capacidad de 200 pF en serie con una resistencia de valor inferior, pero aproximadamente igual, a 1 k$\Omega$.
	\item En condiciones �ptimas, es posible conseguir un rendimiento m�ximo de 100 KS/s (cien mil operaciones de conversi�n por segundo). Este valor est� sujeto a un error relativo del $0.02\%$.
	\item La resoluci�n del conversor anal�gico digital es de 16 bits por muestra. El rango de amplitudes en el que opera depende de como est� configurado el modo de adquisici�n. Puede ir de 0 V a 10 V si el modo de adquisici�n es unipolar, o de -10 V a 10 V si es bipolar.
	\item La cola de muestreo tiene capacidad para hasta 256 canales distintos. Cada uno de los cuales puede configurarse independientemente en t�rminos de ganancia, frecuencia de muestreo, modo de adquisici�n o modo de terminaci�n.
	\item La ganancia, responsable en parte de la resoluci�n con la que se cuantifica las muestras, puede tomar cada ciclo de reloj uno de entre 16 valores posibles (v�ase el \vref{tab:acqmodes}).
\end{itemize}


\section{Descripci�n funcional}\label{sec:funcdesc}

Es necesario programar el comportamiento de la tarjeta de adquisici�n antes de ponerla en funcionamiento. Desde la cola de muestreo se controlan los principales aspectos del proceso de adquisici�n, como por ejemplo, en que instantes se encuentra activo.\par
La cola de muestreo, como su propio nombre indica, es una estructura de datos ordenada. Se encuentra almacenada en una memoria \sig{ram} de 256 entradas que forma parte del hardware de la tarjeta. En el \vref{tab:queue} puede verse una representaci�n de un ejemplo de la cola de muestreo.\par

\begin{table}
	\centering
	\begin{threeparttable}
	\begin{tabular}{lccccccccc}
		\toprule
		Posici�n en la cola & 1 & 2 & 3 & 4 %
		& \multicolumn{2}{c}{$\cdots$} & 254 & 255 & 256 \\
		\midrule
		N�mero de canal & 15 & 15 & 02 & 02 %
		& \multicolumn{2}{c}{$\cdots$} & 17 & 13 & 01 \\
		N�mero de puerto\tnote{a, b} & 07 & 07 & 11 & 11 %
		& \multicolumn{2}{c}{$\cdots$} & 07 & 09 & 01 \\
		Ganancia & 1 & 1 & 40 & 40 %
		& \multicolumn{2}{c}{$\cdots$} & 200 & 8 & 80 \\
		Modo de adquisici�n\tnote{c} & $\pm$ & $\pm$ & $+$ & $+$ %
		& \multicolumn{2}{c}{$\cdots$} & $\pm$ & $+$ & $\pm$ \\
		Modo de terminaci�n\tnote{d} & \sig{d} & \sig{d} & \sig{m} %
		& \sig{m} & \multicolumn{2}{c}{$\cdots$} & \sig{d} & \sig{m} %
		& \sig{m} \\
		\bottomrule
	\end{tabular}
	\begin{TableNotes}
		\tnotetext{a}{Si el canal es diferencial el n�mero de puerto identifica un par de puertos. Un canal diferencial no puede estar asociado a un n�mero de puerto superior a 07.}
		\tnotetext{b}{Dos canales pueden estar relacionados con los mismos puertos f�sicos.}
		\tnotetext{c}{Se define: configuraci�n bipolar ($\pm$); configuraci�n unipolar ($+$).}
		\tnotetext{d}{Se define: canal diferencial (\sig{d}); canal monoterminal (\sig{m}).}
	\end{TableNotes}
	\end{threeparttable}
	\caption[Ejemplo de cola de muestreo]{Ejemplo de cola de muestreo.}
	\label{tab:queue}
\end{table}

Cada entrada en la memoria \sig{ram} se identifica con una posici�n en la cola. Las posiciones en la cola pueden encontrarse vac�as o estar ocupadas por un canal. Varias posiciones en la cola, consecutivas o no, pueden estar ocupadas por un mismo canal. Por tanto, la cola puede estar ocupada, como m�ximo, por 256 canales independientes.\par
Las posiciones ocupadas contienen informaci�n correspondiente al canal y a los atributos asociados a este. Un canal es una entidad l�gica que relaciona un puerto f�sico con un b�ffer de informaci�n y una serie de atributos. Durante el proceso de adquisici�n un puntero recorre las distintas posiciones de la cola, una a una y en orden. El canal activo, aquel que ocupa la posici�n a la que apunta el puntero en cada ciclo de reloj, determina tres cosas:

\begin{itemize}
	\item De qu� puerto debe proceder la se�al anal�gica\footnote{Por convenio se ha elegido hablar de una sola se�al que entra al amplificador. Si se ha hecho esta elecci�n, es porque si bien al amplificador pueden entrar una o dos se�ales simult�neamente, esto no supone otra diferencia para el proceso que la expuesta en el \vref{subsubsec:termmodes}. Es por ello, y para mantener la claridad, que se ha omitido esta posibilidad.} que llega al amplificador de instrumentaci�n interno de la tarjeta.
	\item D�nde, en qu� b�ffer, debe almacenarse el valor resultante de muestrear y cuantificar esta se�al.
	\item Por �ltimo, los atributos asociados al canal: ganancia, modo de adquisici�n y modo de terminaci�n; indican, respectivamente: cual debe ser la ganancia del amplificador de instrumentaci�n, cual debe ser el rango de trabajo del conversor anal�gico digital, y que se debe conectar a los terminales de entrada del amplificador de instrumentaci�n. Se da m�s informaci�n al respecto en apartados subsiguientes.% la polaridad de las muestras y el n�mero de terminales de entrada, uno o dos dependiendo de si la adquisici�n es diferencial o no, durante el ciclo actual.
\end{itemize}

Para concluir el apartado cabe remarcar lo siguiente. Es posible inferir dos cosas de esta mec�nica de funcionamiento basada en la cola. Una de ellas es que el proceso de adquisici�n afecta a una sola se�al cada vez. Y la segunda, que la frecuencia de muestreo ligada a un canal depende de dos factores, de la velocidad de la se�al de reloj, y de la cantidad de veces que un canal aparece repetido en la cola.


\subsection{M�todos de entrada}

La \kpci{} permite dos modos de adquisici�n y dos modos de terminaci�n. Aprender a diferenciar cuando es oportuno seleccionar entre cada uno de ellos beneficiar� la calidad de la se�al digital resultante.


\subsubsection{Modos de adquisici�n}
Una se�al es bipolar cuando toma valores positivos y negativos. Por el contrario, se distingue a las se�ales unipolares porque todos sus valores mantienen la misma polaridad, ya sea �sta positiva o negativa. Para cada canal, debe configurarse el modo de adquisici�n como bipolar o unipolar\footnote{La tarjeta \kpci{} s�lo admite se�ales unipolares de polaridad positiva.} atendiendo a la se�al de inter�s.\par
Si se sabe a ciencia cierta que la se�al de entrada es unipolar debe emplearse el modo de adquisici�n unipolar. De ese modo, se duplica la resoluci�n del conversor anal�gico digital.

\begin{table}
	\centering
	\begin{tabular}{>{\raggedleft}p{1.2cm}d{5.2}d{3.1}+d{3.1}}
		\toprule
		& \multicolumn{2}{c}{Bipolar} & \multicolumn{2}{c}{Unipolar} \\
		\cmidrule(r){2-3}\cmidrule(l){4-5}
		\multicolumn{1}{c}{Ganancia}
		& \multicolumn{1}{c}{Rango ($\pm\text{V}$)} %
		& \multicolumn{1}{c}{Precisi�n ($\mu\text{V}$)} %
		& \multicolumn{1}{c}{Rango (V)} %
		& \multicolumn{1}{c}{Precisi�n ($\mu\text{V}$)} \\
		\midrule
		1 & 10,0 & 305 & 0/$10,0$ & 153 \\
		2 & 5,0 & 153 & 0/$5,0$ & 76 \\
		4 & 2,5 & 76 & 0/$2,5$ & 38 \\
		8 & 1,25 & 38 & 0/$1,25$ & 19 \\
		10 & 1,0 & 31 & 0/$1,0$ & 15 \\
		\\
		& \multicolumn{2}{c}{Bipolar} & \multicolumn{2}{c}{Unipolar} \\
		\cmidrule(r){2-3}\cmidrule(l){4-5}
		\multicolumn{1}{c}{Ganancia}
		& \multicolumn{1}{c}{Rango ($\pm\text{mV}$)} %
		& \multicolumn{1}{c}{Precisi�n ($\mu\text{V}$)} %
		& \multicolumn{1}{c}{Rango (mV)} %
		& \multicolumn{1}{c}{Precisi�n ($\mu\text{V}$)} \\
		\midrule
		20 & 500 & 15 & 0/500 & 7,6 \\
		40 & 250 & 7,6 & 0/250 & 3,8 \\
		80 & 125 & 3,8 & 0/125 & 1,9 \\
		100 & 100 & 3,1 & 0/100 & 1,5 \\
		200 & 50 & 1,5 & 0/50 & 0,8 \\
		400 & 25 & 0,8 & 0/25 & 0,4 \\
		800 & 12,5 & 0,4 & 0/$12,5$ & 0,2 \\
		\bottomrule
	\end{tabular}
	\caption[Relaci�n entre ganancia, rango de trabajo y resoluci�n seg�n el modo de adquisici�n]{Relaci�n entre ganancia, rango de trabajo y resoluci�n seg�n el modo de adquisici�n.}
	\label{tab:acqmodes}
\end{table}


\subsubsection{Modos de terminaci�n}\label{subsubsec:termmodes}

Internamente, la tarjeta \kpci{} emplea un amplificador de instrumentaci�n diferencial. En principio este hecho implicar�a que a cada canal se asociasen dos puertos f�sicos, uno por cada uno de los dos terminales de entrada del amplificador. No obstante, es posible configurar la tarjeta para que uno de los terminales del amplificador se conecte a masa. En ese caso, el terminal restante se conecta a un puerto f�sico. La \vref{fig:termmodes} muestra un ejemplo de esta configuraci�n.\par

\begin{figure}
	\begin{center}
		\includegraphics{gis-pfc-ch2-02.mps}
	\end{center}
	\caption[Ejemplo de configuraci�n de terminaci�n]{Figura que muestra el modo de terminaci�n sencillo. La entrada superior del amplificador de instrumentaci�n se conecta al puerto 9 y la entrada inferior se conecta a masa.}
	\label{fig:termmodes}
\end{figure}

Para habilitar semejante configuraci�n, cabr�a pensar que es suficiente con modificar el atributo que controla el modo de terminaci�n del canal correspondiente. Al contrario de lo que pudiera parecer, el modo de terminaci�n es una propiedad que no es atribuible al canal, si no que se atribuye a un par de puertos\footnote{Aunque seg�n las especificaciones del fabricante es posible configurar cada par de puertos para que opere seg�n un modo de terminaci�n independiente del resto, el entorno de desarrollo utilizado durante el proyecto s�lo admite un modo de terminaci�n para el conjunto total de puertos (v. el \vref{subsec:environment}).}. En concreto, el modo de terminaci�n afecta a los pares de puertos compuestos por un primero de entre los puertos 0 y 7, y un segundo cuyo n�mero de puerto es igual al del primero m�s 8. De ah� que todos los canales relacionados con el mismo par de puertos deban estar configurados con el mismo modo de terminaci�n, una configuraci�n distinta no est� permitida.\par
Los dos modos de terminaci�n posibles se conocen como: diferencial, si es que la se�al que entra en cada terminal del amplificador procede de cada uno de los puertos del par; sencillo, en caso de que uno de los terminales de entrada del amplificador se conecte a la referencia de tensi�n. En este documento se llama canal diferencial a los canales cuyo modo de terminaci�n sea diferencial, y canal con un s�lo terminal o monoterminal a aquellos para los que el modo de terminaci�n es sencillo. \par
Las ventajas que presenta el uso de uno u otro tipo de canales son comprensibles. Emplear canales con un s�lo terminal permite aplicar el proceso de adquisici�n sobre un mayor n�mero de se�ales simult�neamente. Por el contrario, utilizar canales diferenciales redunda en una mayor inmunidad frente al ruido. Adem�s, los amplificadores diferenciales eliminan de forma inherente la componente en continua.\par


\subsection{Rendimiento}\label{subsec:throughput}

Se entiende como rendimiento la cantidad m�xima de operaciones de conversi�n que el dispositivo puede realizar por unidad de tiempo. Para que una de estas operaciones contribuya a la medida de rendimiento debe superar un requisito de precisi�n.\par
El rendimiento �ptimo de la tarjeta \kpci{} especificado por el fabricante es de cien mil operaciones por segundo (100 KS/s). No obstante se advierte, para obtener este nivel de rendimiento es necesario alimentar la tarjeta con una fuente de tensi�n ideal. Adem�s, es importante que exista adaptaci�n de impedancias entre el circuito de alimentaci�n y el puerto por el que se quiere introducir la se�al. A�n en estas condiciones el valor proporcionado por la casa Keithley est� sujeto a un error relativo m�ximo del $0.02\%$, el cual supone un error absoluto m�ximo de dos mil operaciones por segundo (2 KS/s).\par
Es bien sabido, a trav�s del teorema de Nyquist, que la frecuencia de muestreo con la que debe muestrearse una se�al anal�gica para que esta pueda ser recuperada a partir de la versi�n digital obtenida en el proceso de muestreo debe ser mayor o igual al doble de la mayor de las frecuencias con las que oscila la se�al original, siempre que esta sea peri�dica y se encuentre limitada en banda. No obstante, en la pr�ctica, para poder elaborar una representaci�n visual aceptable de se�ales que no se encuentren sujetas a tales limitaciones ---ancho de banda finito y periodicidad de la se�al--- la frecuencia de muestreo debe estar en torno a unas cinco veces el l�mite superior de frecuencias correspondiente a la se�al que desea representarse. Por este motivo el rendimiento del dispositivo de adquisici�n y por ende del sistema de adquisici�n y procesado que en �ste se sustenta es un factor limitante que determina la frecuencia de oscilaci�n m�xima de la se�ales con las que dicho sistema trabaja. En este caso, considerando el montaje propuesto para el desarrollo de los experimentos que ata�en a este proyecto, el rendimiento de la \kpci{} limita la frecuencia de operaci�n de los transductores empleados y, en consecuencia, la profundidad de penetraci�n de los pulsos ultras�nicos y, por consiguiente, el espesor m�ximo de las muestras empleadas en los experimentos. Esta relaci�n surge inequ�vocamente de las conclusiones que se extraen de los desarrollos llevados a cabo en este mismo documento en el \cref{subsec:field}, expuestas en parte en otro \nameref{subsec:quality}, el \vref{subsec:quality}.


\subsubsection[Factores limitantes del rendimiento]{Amplificador de instrumentaci�n y p�rdida del rendimiento}

El amplificador de instrumentaci�n interno de la \kpci{} es de ganancia variable. Es posible configurar una ganancia distinta para cada canal. El prop�sito del amplificador es permitir al usuario modificar la amplitud de la se�al que entra al conversor. La intenci�n que se persigue es conseguir que la conversi�n se enfoque en los detalles de la se�al que sean de mayor inter�s y se pierda la m�nima informaci�n posible. Todo ello a�n trabajando simult�neamente con m�ltiples se�ales cuyo rango de amplitudes es con frecuencia muy diferente.\par
La desventaja que presenta esta configuraci�n ---multiplexor, amplificador de instrumentaci�n, conversor--- es una p�rdida de rendimiento que se produce en situaciones determinadas a causa de la intervenci�n del amplificador en la operaci�n de adquisici�n.\par
Cada ciclo de reloj cambia el canal activo y debe cambiar, si es oportuno, la se�al que accede al amplificador. Este proceso no es inmediato. Tras conmutar el multiplexor que precede al amplificador, se da paso al puerto conveniente. No obstante, la se�al que recibe el amplificador presenta, hasta transcurrido un determinado periodo de tiempo, una componente residual de la se�al que se amplific� en el anterior ciclo de reloj. Transcurrido dicho periodo de tiempo la se�al que entra al amplificador se ve libre de esa componente residual y se corresponde �nicamente con la se�al que entrega el multiplexor, se dice que se ha fijado la se�al.\par
Y es as� como el amplificador es causa de p�rdida de rendimiento, por medio de las componentes residuales. Si la conversi�n se realiza antes de fijar la se�al, el conversor toma un valor de la se�al corrompido por la componente residual de la se�al precedente. Por tanto, la muestra resultante queda igualmente corrompida incluso hasta el punto de perder su validez. Las operaciones de conversi�n que tengan como resultado muestras inv�lidas s�lo contribuyen a falsear la medida de rendimiento, haciendo que parezca mayor de lo que en realidad es.\par
El fabricante da a entender que existe una soluci�n de dise�o que resuelve en parte el problema planteado por las componentes residuales. Esta soluci�n consiste en alargar de forma deliberada la duraci�n del ciclo de reloj, de esa forma se proporciona tiempo suficiente para fijar la se�al. Sin embargo, esta soluci�n presenta dos inconvenientes: no solventa el problema en la totalidad de los casos y es, asimismo, una forma de perder rendimiento. Lo cual conduce inevitablemente a una soluci�n de compromiso, alargar el ciclo de reloj lo suficiente para que en la mayor�a de los casos el efecto de las componentes residuales sobre la precisi�n de la conversi�n no invalide las muestras resultantes y se produzca, por tanto, una ca�da del rendimiento, sin que la duraci�n del nuevo ciclo contribuya por s� misma a una p�rdida notable de �ste.\par


\subsubsection{Optimizaci�n del rendimiento}

Como se ha visto, la inclusi�n del amplificador en el dise�o de la tarjeta es causa directa o indirecta de una p�rdida de rendimiento. La magnitud de esa ca�da en el rendimiento depende de la configuraci�n de la cola de muestreo y de la amplitud de la se�al una vez llega �sta al dispositivo de adquisici�n.

\begin{itemize}
	\item Las se�ales cuya tensi�n absoluta se encuentra por debajo de los 100 mV al llegar a la \kpci{} sufren en mayor medida las consecuencias del empleo de un amplificador en la operaci�n de conversi�n. En primer lugar la se�al tarda m�s en fijarse de modo que el rendimiento se reduce a la mitad en las mejores condiciones, de 100 KS/s pasa a 50 KS/s. Esto es debido a que, al ser la amplitud de la se�al y la del ruido comparables, especialmente despu�s de que �ste se vea reforzado por el efecto de las componentes residuales, se genera una mayor incertidumbre.\par
	Por otro lado las se�ales que requieren que el amplificador opere con alta ganancia son las m�s perjudicadas por los problemas que causa el amplificador en configuraciones multiganancia, tal y como se explica a continuaci�n.
	\item Por lo general, el rendimiento se ve afectado de forma m�s pronunciada por el efecto de las componentes residuales en configuraciones multiganancia en las que se encadenan secuencias de canales con ganancia diferente. Una configuraci�n multiganancia de la cola de muestreo implica que en diferentes ciclos de reloj el amplificador act�a con ganancias distintas. Eso con frecuencia significa que el rango en el que se encuentra comprendida la amplitud de las se�ales que est�n entrando al dispositivo de adquisici�n es diferente de una se�al a otra. Cuando as� ocurre puede sucederse en ocasiones que en ciclos de reloj consecutivos entren al amplificador dos se�ales de amplitud diferente, siendo la amplitud de la se�al que ocupa el primer ciclo mucho mayor que la de la otra se�al en el tiempo en el que ambas permanecen a la entrada del dispositivo. Por otro lado, parece l�gico considerar que la componente residual asociada a una se�al cuya amplitud sea predominantemente mayor que la de otra se�al es de mayor amplitud inicial y mayor duraci�n temporal que la asociada a la segunda se�al. Por tanto si ocurre como se ha dicho y se tiene en cuenta la base probable que se ha propuesto, cuando la segunda de las se�ales se convierte en la se�al activa la amplitud de la componente residual asociada a la primera de ellas puede ser suficiente, incluso, para enmascararla.\par No s�lo eso, la amplitud de la se�al que llega m�s tarde al amplificador puede ser, en t�rminos absolutos, la mayor parte del tiempo, menor que la de la otra se�al, al ser as� lo m�s probable es que se amplifique empleando un mayor factor de ganancia. De ser as�, la amplitud de la componente residual a la que se enfrenta esta se�al puede provocar en el peor de los casos que el conversor sature y la p�rdida de precisi�n sea mucho mayor. Sea cual sea el caso, es posible observar entonces, que en configuraciones multiganancia las muestras resultantes se obtienen de una conversi�n menos precisa, en especial si se trabaja con se�ales de peque�a amplitud ---tal y como se especific� en el punto anterior--- o si las ganancias configuradas en la cola de muestreo difieren mucho unas de otras. Aplicando la relaci�n entre la validez de las muestras y el rendimiento de la que se habl� anteriormente, la consecuencia de emplear configuraciones multiganancia es una mayor p�rdida de rendimiento.
	\item En configuraciones monoganancia el uso del amplificador supone una causa indirecta de la ca�da de rendimiento. El dise�o del dispositivo de adquisici�n est� pensado primordialmente para su uso en configuraciones multiganancia, de lo contrario la inclusi�n de un amplificador de ganancia variable en el esquem�tico de la tarjeta ser�a incomprensible. Por la misma raz�n, Keithley adopta una soluci�n de dise�o como la expuesta en el anterior apartado, para tratar de obtener un rendimiento �ptimo en configuraciones multiganancia. Sin embargo, el efecto de las componentes residuales en configuraciones monoganancia es m�nimo y la consecuente p�rdida de rendimiento tambi�n lo es. Por tanto, una soluci�n que consiste en alargar el ciclo de reloj resulta, en configuraciones monoganancia, innecesaria y perjudicial para el rendimiento.
\end{itemize}

Las acciones que el fabricante adopta para tratar de que el usuario obtenga el mayor rendimiento posible del dispositivo no se limitan a aplicar una soluci�n de compromiso en el dise�o de la duraci�n del ciclo de reloj. En el manual de usuario se hacen una serie de recomendaciones de uso orientadas a conseguir este fin.\par
Se proponen varias soluciones, la m�s trivial de las cuales pasa por preamplificar todas las se�ales que vayan a ser objeto del proceso de adquisici�n efectuado por la tarjeta consiguiendo que su amplitud var�e en un mismo rango. Si se hace as�, es suficiente con emplear una configuraci�n monoganancia para minimizar los efectos de las componentes residuales en el rendimiento. Adem�s al preamplificar las se�ales, �stas presentan una mejor relaci�n se�al a ruido, es decir, son menos vulnerables al ruido. Aunque buena, esta soluci�n no deja de ser trivial puesto que el amplificador de instrumentaci�n de la tarjeta pierde toda funcionalidad y pasa a ser un estorbo en el proceso de adquisici�n.\par
La soluci�n de car�cter pr�ctico propuesta por Keithley radica configurar la cola de muestreo de forma minuciosa, persiguiendo optimizar el rendimiento. Como se ha visto, en determinadas ocasiones una configuraci�n inapropiada de la cola de muestreo puede inducir que la p�rdida de productividad que provoca la inclusi�n del amplificador en el circuito de adquisici�n sea todav�a mayor. Para evitar que esto ocurra y sacar el m�ximo partido del dispositivo se dan en el manual dos condiciones que de cumplirse garantizan que la cola se encuentre configurada de forma �ptima en t�rminos de rendimiento.

\begin{itemize}
	\item La primera consiste en agrupar canales con distinta ganancia en posiciones consecutivas de la cola, a�n si al hacerlo se pierde el orden de muestreo definido en una primera instancia por el usuario. Si como se presupuso en el apartado anterior, com�nmente dos se�ales que requieren ser amplificadas con el mismo factor de ganancia var�an en el mismo rango de amplitudes, en estas secuencias monoganancia las consecuencias de las componentes residuales en la precisi�n de la conversi�n son m�nimas.
	\item A pesar de emplear una configuraci�n como la anterior, la aparici�n de saltos de ganancia en la cola de muestreo es todav�a probable. Por ejemplo, en la transici�n entre dos secuencias monoganancia como las descritas arriba. El salto es a�n m�s problem�tico si la transici�n se realiza para dar paso a una secuencia de ganancia mayor. El primer canal de esta secuencia sufre en mayor proporci�n los efectos de las componentes residuales y el rendimiento asociado al canal se ve reducido dram�ticamente. Para minimizar el impacto que en determinados canales como �ste tienen los problemas causados por el amplificador, es posible modificar la configuraci�n de la cola para que dichos canales ocupen varias posiciones consecutivas. Esta segunda condici�n persigue dar m�s tiempo para que se fije la se�al cuando los mencionados canales est�n activos. Para ello se necesitan posiciones vac�as en la cola, posiciones que es posible obtener desalojando canales previamente configurados.
\end{itemize}


\section{Comunicaci�n con el perif�rico}

Para la interacci�n con dispositivos externos, la \kpci{} dispone de dos conectores formato mini-\sig{d} de 36 terminales que cumplen con el est�ndar \sig{ieee} 1284 de protocolos de comunicaci�n en paralelo. Si se observa la tarjeta como el rect�ngulo que forma desde un punto de vista geom�trico y observ�ndola de frente, los conectores quedan ubicados en un lado de la tarjeta contiguo a aquel en el que se sit�a la conexi�n \sig{pci}. Todo ello de forma que tras el montaje del perif�rico en la placa base los conectores quedan expuestos en la parte posterior de la carcasa en la que providencialmente debe encontrarse instalada dicha placa, tal y como es habitual en este tipo de dispositivos.\par
En las \cref{fig:portanalog,fig:portdigital} se etiqueta cada terminal seg�n su ubicaci�n relativa con respecto al conector y al resto de terminales en el mismo, para cada conector, mostrando la distribuci�n definida por el fabricante. Los \cref{tab:analog,tab:digital} describen el prop�sito de cada terminal y se especifica que tipo de se�al debe circular por los mismos.

\begin{figure}
	\begin{center}
		\subfloat[Conector etiquetado como \emph{analog}][Conector etiquetado como \emph{analog}.]{
			\label{fig:portanalog}
			\begin{minipage}{.9\textwidth}
				\centering\includegraphics{gis-pfc-ch2-03.mps}
			\end{minipage}}
			\vspace*{.1\textheight}
		\subfloat[Conector etiquetado como \emph{digital}][Conector etiquetado como \emph{digital}.]{
			\label{fig:portdigital}
			\begin{minipage}{.9\textwidth}
				\centering\includegraphics{gis-pfc-ch2-04.mps}
			\end{minipage}}
	\end{center}
	\caption[Conectores traseros de la tarjeta de adquisici�n]{Conectores traseros de la tarjeta de adquisici�n.}
	\label{fig:ports}
\end{figure}

\begin{table}
	\centering
	\begin{tabular}{>{\raggedleft}p{1cm} >{\scshape}c >{\arraybackslash}l}
		\toprule
		\multicolumn{1}{c}{Terminal} & {\upshape Puerto asignado} & \multicolumn{1}{c}{Descripci�n} \\
		\midrule
		1 & ip5 & \multirow{12}{0.6\textwidth}{Bits digitales de entrada multifunci�n. Pueden ser configurados por el usuario para que ejerzan la funci�n de:\miniit{\item Base temporal para el contador/temporizador y/o entrada a \emph{gate} \\\item Reloj externo para conversiones A/D o D/A \\\item Disparador digital externo \\\item Entrada digital en el modo \emph{target-mode}}} \\
		2 & ip3 & \\
		3 & ip1 & \\
		\\\\\\\\\\\\\\\\
		\midrule
		4 & op5 & \multirow{12}{0.6\textwidth}{Bits digitales de salida multifunci�n. Pueden ser configurados por el usuario para que ejerzan la funci�n de:\miniit{\item Salidas del contador/temporizador\\\item Salida del disparador\\\item Salida de control para accesorios\\\item Salida del reloj interno\\\item Salida digital en el modo \emph{target-mode}}} \\
		5 & op3 & \\
		6 & op1 & \\
		\\\\\\\\\\\\\\\\
		\midrule
		7 & dgnd & \multicolumn{1}{c}{Tierra digital} \\
		\midrule
		8 & ch07 lo/ch15 & \multirow{4}{0.6\textwidth}{Entradas anal�gicas, cuya funci�n depende del modo de terminaci�n configurado: puerto asociado a un canal monoterminal o puerto bajo de un canal diferencial} \\
		9 & ch06 lo/ch14 & \\
		10 & ch05 lo/ch13 & \\
		$\vdots$ & $\vdots$ & \\
		15 & ch00 lo/ch08 & \\
		\midrule
		16 & {\upshape Sin conexi�n} & \\
		\midrule
		\multicolumn{1}{l}{17, 18} & agnd & Tierra anal�gica \\
		\bottomrule
	\end{tabular}
	\caption[Relaci�n entre los puertos y terminales en el conector \emph{analog}]{Relaci�n entre los puertos y terminales que presenta el conector trasero de la \kpci{} etiquetado como \emph{analog}.}
	\label{tab:analog}
\end{table}

\begin{table}\ContinuedFloat
	\centering
	\begin{tabular}{>{\raggedleft}p{1cm} >{\scshape}c >{\arraybackslash}l}
		\toprule
		\multicolumn{1}{c}{Terminal} & {\upshape Puerto asignado} & \multicolumn{1}{c}{Descripci�n} \\
		\midrule
		19 & ip4 & \multirow{12}{0.6\textwidth}{Bits digitales de entrada multifunci�n. Pueden ser configurados por el usuario para que ejerzan la funci�n de:\miniit{\item Base temporal para el contador/temporizador y/o entrada a \emph{gate} \\\item Reloj externo para conversiones A/D o D/A \\\item Disparador digital externo \\\item Entrada digital en el modo \emph{target-mode}}} \\
		20 & ip2 & \\
		21 & ip0 & \\
		\\\\\\\\\\\\\\\\
		\midrule
		22 & op4 & \multirow{12}{0.6\textwidth}{Bits digitales de salida multifunci�n. Pueden ser configurados por el usuario para que ejerzan la funci�n de:\miniit{\item Salidas del contador/temporizador\\\item Salida del disparador\\\item Salida de control para accesorios\\\item Salida del reloj interno\\\item Salida digital en el modo \emph{target-mode}}} \\
		22 & op2 & \\
		21 & op0 & \\
		\\\\\\\\\\\\\\\\
		\midrule
		25 & {\upshape +5 V} & \multirow{2}{0.6\textwidth}{Referencia de tensi�n de 5 voltios de corriente continua extra�dos del bus \sig{pci} del ordenador} \\
		& & \\
		\midrule
		26 & ch07 hi & \multirow{3}{0.6\textwidth}{Entradas anal�gicas restantes, en el modo de terminaci�n diferencial representan el puerto alto de un canal diferencial} \\
		27 & ch06 hi & \\
		28 & ch05 hi & \\
		$\vdots$ & $\vdots$ & \\
		33 & ch00 hi & \\
		\midrule
		34 & {\upshape +10 V} & \multirow{6}{.6\textwidth}{Entrada dise�ada para proporcionar al dispositivo una referencia externa de precisi�n de 10 voltios mediante una fuente de alta impedancia de salida (La impedancia de entrada de este puerto es equivalente a una resistencia de 1 k$\Omega$ en serie con la impedancia de entrada de la fuente)} \\
		\\\\\\\\\\
		\bottomrule
	\end{tabular}
	\caption[]{Continuaci�n del \vref{tab:analog}.}
\end{table}

\begin{table}\ContinuedFloat
	\centering
	\begin{tabular}{>{\raggedleft}p{1cm} >{\scshape}c >{\arraybackslash}l}
		\toprule
		\multicolumn{1}{c}{Terminal} & {\upshape Puerto asignado} & \multicolumn{1}{c}{Descripci�n} \\
		\midrule
		35 & dac1 & \multirow{2}{.6\textwidth}{Salida n�mero 1 del conversor digital a anal�gico de la \kpci{}} \\
		\\
		\midrule
		36 & dac0 & \multirow{2}{.6\textwidth}{Salida n�mero 0 del conversor digital a anal�gico de la \kpci{}} \\
		\\
		\bottomrule
	\end{tabular}
	\caption[]{Continuaci�n del \vref{tab:analog}.}
\end{table}

\begin{table}
	\centering
	\begin{tabular}{>{\raggedleft}p{1cm} >{\scshape}c >{\arraybackslash}l}
		\toprule
		\multicolumn{1}{c}{Terminal} & {\upshape Puerto asignado} & \multicolumn{1}{c}{Descripci�n} \\
		\midrule
		1 & {\upshape Bit 0} & \multirow{6}{0.6\textwidth}{Canal 0 de bits de entrada/salida de prop�sito general (En la \kpci{} los bits digitales se agrupan de en ocho en ocho en canales. Los canales de este tipo puede configurarse para que los bits que lo integran se comporten como todo salidas o todo entradas)} \\
		2 & {\upshape Bit 1} & \\
		3 & {\upshape Bit 2} & \\
		$\vdots$ & $\vdots$ & \\
		8 & {\upshape Bit 7} & \\
		\\
		\midrule
		9 & {\upshape Bit 8} & \multirow{2}{0.6\textwidth}{Canal 1 de bits de entrada/salida de prop�sito general} \\
		10 & {\upshape Bit 9} & \\
		11 & {\upshape Bit 10} & \\
		$\vdots$ & $\vdots$ & \\
		16 & {\upshape Bit 15} & \\
		\midrule
		\multicolumn{1}{l}{17, 18} & dgnd & Tierras digitales \\
		\midrule
		19 & {\upshape Bit 16} & \multirow{2}{0.6\textwidth}{Canal 2 de bits de entrada/salida de prop�sito general} \\
		20 & {\upshape Bit 17} & \\
		21 & {\upshape Bit 18} & \\
		$\vdots$ & $\vdots$ & \\
		26 & {\upshape Bit 23} & \\
		\midrule
		27 & {\upshape Bit 24} & \multirow{2}{0.6\textwidth}{Canal 3 de bits de entrada/salida de prop�sito general} \\
		28 & {\upshape Bit 25} & \\
		29 & {\upshape Bit 26} & \\
		$\vdots$ & $\vdots$ & \\
		34 & {\upshape Bit 31} & \\
		\midrule
		\multicolumn{1}{l}{35, 36} & {\upshape +5 V} & +5 V\sig{dc} desde el bus del ordenador \\
		\bottomrule
	\end{tabular}
	\caption[Relaci�n entre los puertos y terminales en el conector \emph{digital}]{Relaci�n entre los puertos y terminales que presenta el conector trasero de la \kpci{} etiquetado como \emph{digital}.}
	\label{tab:digital}
\end{table}


\subsection{Desarrollo de una nueva interfaz}\label{subsec:conbox}

Por razones pr�cticas, la interfaz original de la \kpci{} basada en dos conectores mini-\sig{d} no se acomoda a las necesidades que se prev� surgir�n durante el desarrollo de las pruebas requeridas para la finalizaci�n de este proyecto de fin de carrera. En consecuencia, con el prop�sito de agilizar en la medida de lo posible la realizaci�n de estos ensayos, se decide ampliar la interfaz existente.\par
Para ello se parte de dos extremos de cable terminados en un conector tipo macho y desprovistos de cubierta, de cada uno de los cuales surgen 36 hilos de cobre recubiertos de pl�stico flexible coloreado. El patr�n de coloreado de las cubiertas individuales es �nico y no se repite en ninguno de los 36 hilos de que se compone cada extremo de cable. A partir de ah�, se identifica cada terminal de cada conector con el hilo de cobre correspondiente, relacionando las etiquetas que el fabricante pone a los terminales con el c�digo de colores de la cubiertas individuales.\par
Una vez recopilada esta informaci�n, se planea la construcci�n de una caja de conexiones. El dise�o de la caja contempla que se equipe �sta con 72 conectores banana hembra soldados a los 72 hilos de cobre y que comunican con los terminales de los conectores mini-\sig{d} en los que termina el otro extremo de los cables. Adem�s se planifica la inclusi�n de cuatro conectores coaxiales con rosca con conexiones redundantes a las que se presupone son las cuatro puertas cuyo uso ser� m�s frecuente, las correspondientes al primer canal anal�gico diferencial y las tierras anal�gicas; para as� facilitar el uso continuado de las mismas. Por �ltimo se decide etiquetar la caja de conexiones con marbetes que reproduzcan la nomenclatura visible en las \vref{fig:portanalog,fig:portdigital}.\par
Puede observarse una representaci�n de la caja de conexiones ya terminada en la \vref{fig:conbox}.

\begin{figure}
	\begin{center}
		\includegraphics[scale=1, keepaspectratio=true]{gis-pfc-ch2-05.jpg}
	\end{center}
	\caption[Plano de la caja de conexiones ya terminada]{Plano de la caja de conexiones ya terminada.}
	\label{fig:conbox}
\end{figure}


\part{Ensayos no destructivos con ultrasonidos}

\include{gis-pfc-ch3-1}

\part*{Ap�ndices}

\newcounter{totalpages}
\setcounter{totalpages}{\value{page}}
\setcounter{page}{1}
\renewcommand\thepage{\thechapter.\arabic{page}}
\appendix

\chapter{Guía de usuario de la aplicación de control}\label{chap:appendixA}

\section{Instalación de los drivers y la aplicación}

Antes de poder utilizar la aplicación es preciso instalar correctamente en
el equipo la tarjeta de adquisición y los drivers de la tarjeta. Los
drivers pueden descargarse desde el sitio web del fabricante
(\url{http://www.keithley.com}).

Para poder utilizar la aplicación es necesario haber instalado previamente
en el \pc{} anfitrión una distribución de \matlab{} (revisión posterior a
2006a).

Para instalar la aplicación copiar el archivo \func{single\_channel.m} y el
archivo con extensión \func{single\_channel.fig} en el directorio de
trabajo.


\section{Especificaciones}

Para obtener información acerca de las prestaciones y limitaciones de la
aplicación véase el \vref{subsec:transducerconclusions}.


\section{Procedimientos de llamada}

Desde la ventana principal de \matlab{} es posible lanzar la aplicación de
tres modos diferentes.

\begin{itemize}
    \item En el navegador del sistema de archivos click derecho en el
	fichero \func{single\_channel.m}. Después seleccionar la opción
	<<ejecutar archivo>>.
    \item Ejecutar \sig{guide} (editor de interfaces gráficas de usuario).
	Abrir el fichero \func{single\_channel.fig}. Después click en el
	botón ejecutar de \sig{guide}.
    \item En el prompt de la ventana de comandos introducir el siguiente
	comando (cerciorarse previamente de que \matlab{} se encuentra en
	el directorio de trabajo).

	\begin{center}
	    \begin{lstlisting}[gobble=12]
		[handles = ]single_channel[(opciones)]
	    \end{lstlisting}
	\end{center}
\end{itemize}

El tercer procedimiento de llamada da acceso a las variables almacenadas en
el dominio interno de la aplicación como, por ejemplo: el búffer en el que se
almacenan las muestras recogidas o el objeto dispositivo que controla el
comportamiento de la tarjeta de adquisición.

La aplicación reutiliza cualquier objeto dispositivo asociado a la tarjeta
de adquisición, si no lo hubiera creo uno propio.


\section{Interfaz de usuario}

La \vref{fig:interface} muestra el aspecto de la interfaz gráfica de
usuario de la aplicación. La interfaz gráfica permite la interacción con el
sistema de medida, pueden encontrarse en ella 6 paneles.

\begin{figure}
    \begin{center}
	\includegraphics{gis-pfc-appa-01.png}
    \end{center}
    \caption[Aspecto de la interfaz de usuario]{Disposición de los
	distintos paneles de control y visualizadores en la interfaz de
	usuario.}
    \label{fig:interface}
\end{figure}

\begin{enumerate}
    \itembf{Panel de control principal} Alberga los controles que
	permiten configurar las propiedades básicas de la sesión de
	adquisición.
    \itembf{Panel de control secundario} Los mandos contenidos en
	este panel permiten controlar aspectos secundarios de la sesión de
	adquisición.
    \itembf{Visor} En el visor se muestran los resultados
	numéricos.
    \itembf{Ventana de representación grande} En esta ventana la
	representación ocupa un tamaño mayor.
    \itembf{Ventana de representación pequeña} En esta ventana la
	representación se hace a escala reducida.
    \itembf{Controles gráficos} Permiten habilitar o inhabilitar una o
	ambas representaciones gráficas.
\end{enumerate}


\subsection{Panel de control principal}

A continuación se enumeran los distintos botones y visualizadores que
integran el panel de control, su disposición en el panel se ilustra en la
\vref{fig:firstcontrolpanel}.

\begin{enumerate}
    \itembf{Selector de canal} Permite seleccionar el canal del que se
	extrae información. Sólo se puede seleccionar un canal a la vez.
    \itembf{Selector de frecuencia} Permite ajustar la frecuencia de
	muestreo de la tarjeta de adquisición. El valor de la frecuencia de
	muestreo debe estar contenido entre 1 kHz y la máxima frecuencia de
	muestreo permitida.
    \itembf{Control de ganancia} Ajusta la ganancia de amplificación del
	amplificador de instrumentación integrado en la tarjeta. Determina
	el rango máximo de la señal de entrada.
    \itembf{Indicador de rango} Rango máximo en el que debe estar contenida
	la señal de entrada (si la señal excede este rango tras ser
	amplificada internamente satura el conversor \sig{a/d} de la
	tarjeta de adquisición).
    \itembf{Control \sig{u/b}} Controla el modo de adquisición
	(unipolar/bipolar).
    \itembf{Control d/s} Controla el modo de terminación
	(diferencial/sencillo). Altera el comportamiento del selector de
	canal (lista 8 canales en modo diferencial y 16 en modo sencillo).
    \itembf{Control de on/off} Inicia o detiene respectivamente la sesión
	de adquisición.
\end{enumerate}


\subsection{Panel de control secundario}

Los elementos contenidos en el panel de control secundario son los listados
a continuación. La \cref{fig:secondcontrolpanel} muestra su ubicación
relativa en el panel.

\begin{enumerate}
    \itembf{Selector de tipo de medida} Permite seleccionar el tipo de
	resultado devuelto: muestra individual, media aritmética o
	representación gráfica de la señal y su espectro en frecuencia.
    \itembf{Ventana \sig{fft}} Permite seleccionar el número de puntos con
	el que se realiza la \sig{fft} utilizada para representar el
	espectro de la señal.
    \itembf{División temporal} Controla la duración del fragmento de señal
	representado gráficamente. Activa o desactiva el modo de
	representación continuo.
    \itembf{Cambio de ventana} Intercambia la ventana (ventanas de
	representación grande y pequeña) en la que se representan
	gráficamente la señal y su espectro en frecuencia.
\end{enumerate}

\newlength{\biggestpanel}
\settoheight{\biggestpanel}{\includegraphics{gis-pfc-appa-02.png}}

\begin{figure}
    \begin{center}
	\subfloat[Elementos del panel de control principal][Panel de
	    control principal.]{
	    \label{fig:firstcontrolpanel}
	    \begin{minipage}[top][\biggestpanel][c]{.425\textwidth}
		\centering
		\includegraphics{gis-pfc-appa-02.png}
	    \end{minipage}}
	\subfloat[Elementos del panel de control secundario][Panel de
	    control secundario.]{ \label{fig:secondcontrolpanel}
	    \begin{minipage}[top][\biggestpanel][c]{.425\textwidth}
		\centering
		\includegraphics{gis-pfc-appa-03.png}
	    \end{minipage}}
    \end{center}
    \caption[Paneles de control de la interfaz de usuario]{Esta figura
    muestra los distintos elementos que integran los paneles de control
    principal y secundario de la interfaz de usuario.}
    \label{fig:controlpanels}
\end{figure}

\subsection{Visor}

En el visor se pueden observar los resultados que proporcionan los modos de
funcionamiento numéricos. Los resultados se dan con una precisión de
decenas de milivoltios.


\subsection{Ventanas de representación grande y
pequeña}\label{subsec:windows}

Las representaciones se realizan en estas dos ventanas, cada ventana
alberga simultáneamente una única representación. El control de cambio de
ventana permite intercambiar la ventana en la que se representan señal y
espectro en frecuencias, por defecto la señal se representa en la ventana
grande y su espectro frecuencial en la ventan pequeña. Al pulsar el botón
cambio de ventana se intercambian también las marcas de gráfico, el título
y la rejilla. Los controles situados en el margen de cada ventan duplican o
dividen por dos la escala de amplitud de la representación.

\subsection{Controles gráficos}\label{subsec:gfoptions}

Los controles de este panel gráfico habilitan o inhabilitan la
representación gráfica de la señal y de su espectro en frecuencia
respectivamente. No tienen ninguna función en los modos numéricos de
funcionamiento.


\section{Inicio y detención de la sesión de adquisición}

El control on del panel principal de la interfaz de usuario inicia la
sesión de adquisición. Por su parte el control off detiene el muestreo. Al
pulsar en el control on se inhabilitan todos los controles de la interfaz
de usuario exceptuando los siguientes: control off, ventana \sig{fft},
cambio de ventana, controles de las ventanas de representación y controles
del panel opciones gráficas. Al pulsar off todos los controles
inhabilitados se restablecen.


\section{Modos de funcionamiento}

Existen tres modos de funcionamiento: muestra individual, media aritmética
y representación gráfica. Cada uno de ellos aporta distintos resultados.


\subsection{Muestra individual}

En este modo el visor muestra el valor de la última muestra adquirida (la
frecuencia de adquisición es de 4 Hz).

Si la señal evaluada oscila a una frecuencia elevada los valores observados
cambian rápidamente por lo que no resultan nada útiles.


\subsection{Media aritmética}

En este modo el visor muestra la media aritmética de las muestras
adquiridas en un espacio de tiempo de 250 ms. El número de muestras
utilizado para realizar la media depende de la frecuencia de muestreo (la
frecuencia de adquisición puede ajustarse mediante el selector de
frecuencia).

El visor se refresca cada 250 ms cuando este modo está activado. Si la
señal estudiada oscila rápidamente en el visor puede observarse el valor
<<$0.00$>>.


\subsection{Representación gráfica}

En este modo de funcionamiento se muestra el aspecto de la señal y el de su
espectro en frecuencias. Al seleccionar el modo de funcionamiento gráfico
se inhabilita el selector de frecuencia y se fija la frecuencia de muestreo
a la máxima frecuencia de muestreo posible (100 kHz si no se han añadido
más canales al objeto dispositivo asociado a la tarjeta).

Según la configuración de control división temporal se utiliza uno de los
dos modos de representación implementados.


\subsubsection{Modo de representación disparado}

La representación de la señal se muestra fija en el centro de la ventana de
representación. Este modo es útil para observar señales que oscilan a
frecuencias por encima de los 4 Hz, el disparo falla cuando se observan
señales que oscilan a una frecuencia inferior. Permite determinar
fácilmente la amplitud y la frecuencia de oscilación de la señal.


\subsubsection{Modo de representación continuo}

Este modo de representación se activa cuando la escala temporal de la
representación (control división temporal) se ajusta por encima de los 4
ms. En este modo la representación de la señal se desplaza de derecha a
izquierda a medida que avanza el tiempo. Este modo de representación es
útil para observar la forma de señales que oscilan a frecuencias inferiores
a los 4 Hz. También puede ayudar a determinar la amplitud de una señal. Sin
embargo es difícil averiguar la frecuencia a la que oscila una señal
observando la representación obtenida al activar este modo.


\section{Salida y limpieza}

Al cerrar la aplicación se detiene el proceso de muestreo y se elimina
último canal asignado al objeto dispositivo (si antes no se ha eliminado
ningún canal).

Al terminar la aplicación es necesario eliminar el objeto dispositivo y
limpiar el área de trabajo de \matlab{}.

\begin{center}
    \begin{lstlisting}
	aux = daqfind
	delete(aux);
	clear aux;
    \end{lstlisting}
\end{center}


\addtocounter{totalpages}{\value{page}}
\addtocounter{totalpages}{-1}
\setcounter{page}{1}
\chapter{Pruebas y contenidos adicionales}

Este apéndice no debe adjuntarse en el documento final, en el se anotarán
ideas para modificar el contenido del grueso del documento y bases para
generar los contenidos adicionales.


\section{Configuración de página}

La configuración actual se ha hecho utilizando el paquete \textsf{typearea}
que forma parte del conjunto del \textsc{koma}-\textsc{s}cript. Las
opciones más importantes que deben pasarse a este paquete son el tipo de
papel (A4, A3, A5, letter,\dots) y el espacio de corrección necesario para
evitar fallos en la encuadernación (\textsc{bcor} = \texttt{magnitud} o
directamente \textsc{bcor}\texttt{magnitud}). Utilizando el paquete
\textsf{layouts} pueden crearse figuras que representen la configuración de
página actual.

\newlength{\auxmm}
\newlength{\auxin}
\newlength{\auxpt}
\setlength{\auxmm}{1mm}
\setlength{\auxin}{1in}
\setlength{\auxpt}{1pt}
\newsavebox\caja
\sbox\caja{\includegraphics{gis-pfc-ch2-02.mps}}

\begin{table}
	\centering
	\printinunitsof{mm}\pagevalues\medskip\par

	\begin{tabular}{l l}
		\toprule
		1in = \printinunitsof{mm}\prntlen{\auxin} %
		& 1pt = \printinunitsof{mm}\prntlen{\auxpt} \\
		1mm = \printinunitsof{in}\prntlen{\auxmm} %
		& 1mm = \printinunitsof{pt}\prntlen{\auxmm} \\
		Alto de \texttt{gis-ch2-02.mps} %
		& \printinunitsof{mm}\prntlen{\ht\caja} \\
		Ancho de \texttt{gis-ch2-02.mps} %
		& \printinunitsof{mm}\prntlen{\wd\caja} \\
		marginparwidth %
		= \printinunitsof{pt}\prntlen{\marginparwidth} %
		& marginparsep %
		= \printinunitsof{pt}\prntlen{\marginparsep} \\
		\bottomrule
	\end{tabular}
	\caption[Valores actuales de la distribución de página]{Valores que
	completan el diagrama representado en la \vref{fig:layouts}
	sustituir la nomenclatura de referencia por los valores
	correspondientes}
\end{table}

\begin{figure}
	\begin{center}
		\includegraphics{gis-pfc-ch2-02.mps}
	\end{center}
	\caption[Segunda figura del segundo capítulo]{Segunda figura del
	segundo capítulo.}
	\label{fig:ch102}
\end{figure}

El lector puede fijarse que se cumple la regla de construcción que aplica
el paquete \textsf{typearea} en la que el margen inferior es dos veces el
margen inferior, y que el margen interior de página ---una vez eliminado el
centímetro (\textsc{bcor}\texttt{1cm}) que se deja para compensar el
encuadernado--- es la mitad del margen exterior de página. De ese modo se
crea una distribución de página en la que el margen interior de página de
las páginas par e impar juntas es igual a cada uno de los márgenes
exteriores de ambas páginas.

\begin{figure}
	\pagediagram
	\caption{Distribución del texto en las páginas de este documento}
	\label{fig:layouts}
\end{figure}

\begin{figure}\ContinuedFloat
	\currentpage
	\pagedesign
	\caption[]{Continuación del \vref{fig:layouts}}
\end{figure}


\section{Gráficos con MetaPost}

Anotaciones destinadas a obtener gráficos escalables de calidad empleando
el paquete MetaPost para la creación de gráficos en PostScript.


\subsection{Tamaño}

Tengo que modificar el tamaño de algunas figuras al haber cambiado la
tipografía predeterminada de la \emph{Computer Roman} a \emph{Lucida}. Lo
que quiero que aparezca en este apartado es un cuadro con los distintos
tamaños de letra que aparecen en el documento.


\backmatter

\addtocounter{totalpages}{\value{page}}
\addtocounter{totalpages}{-1}
\setcounter{page}{\value{totalpages}}
\renewcommand\thepage{\arabic{page}}
\nocite{mittelbach2004lc, stutzman1997atd, garcia2000mrsr}
\bibliographystyle{bababbrv-lf}
\bibliography{gis-pfc}

\end{document}
