\chapter{Pruebas y contenidos adicionales}

Este apéndice no debe adjuntarse en el documento final, en el se anotarán ideas para modificar el contenido del grueso del documento y bases para generar los contenidos adicionales.


\section{Configuración de página}

La configuración actual se ha hecho utilizando el paquete \textsf{typearea} que forma parte del conjunto del \textsc{koma}-\textsc{s}cript. Las opciones más importantes que deben pasarse a este paquete son el tipo de papel (A4, A3, A5, letter,\dots) y el espacio de corrección necesario para evitar fallos en la encuadernación (\textsc{bcor} = \texttt{magnitud} o directamente \textsc{bcor}\texttt{magnitud}). Utilizando el paquete \textsf{layouts} pueden crearse figuras que representen la configuración de página actual.

\newlength{\auxmm}
\newlength{\auxin}
\newlength{\auxpt}
\setlength{\auxmm}{1mm}
\setlength{\auxin}{1in}
\setlength{\auxpt}{1pt}
\newsavebox\caja
\sbox\caja{\includegraphics{gis-pfc-ch2-02.mps}}

\begin{table}
	\centering
	\printinunitsof{mm}\pagevalues\medskip\par

	\begin{tabular}{l l}
		\toprule
		1in = \printinunitsof{mm}\prntlen{\auxin} & 1pt = \printinunitsof{mm}\prntlen{\auxpt} \\
		1mm = \printinunitsof{in}\prntlen{\auxmm} & 1mm = \printinunitsof{pt}\prntlen{\auxmm} \\
		Alto de \texttt{gis-ch2-02.mps} & \printinunitsof{mm}\prntlen{\ht\caja} \\
		Ancho de \texttt{gis-ch2-02.mps} & \printinunitsof{mm}\prntlen{\wd\caja} \\
		marginparwidth = \printinunitsof{pt}\prntlen{\marginparwidth} & marginparsep = \printinunitsof{pt}\prntlen{\marginparsep} \\
		\bottomrule
	\end{tabular}
	\caption[Valores actuales de la distribución de página]{Valores que completan el diagrama representado en la \vref{fig:layouts} sustituir la nomenclatura de referencia por los valores correspondientes}
\end{table}

\begin{figure}
	\begin{center}
		\includegraphics{gis-pfc-ch2-02.mps}
	\end{center}
	\caption[Segunda figura del segundo capítulo]{Segunda figura del segundo capítulo.}
	\label{fig:ch102}
\end{figure}

El lector puede fijarse que se cumple la regla de construcción que aplica el paquete \textsf{typearea} en la que el margen inferior es dos veces el margen inferior, y que el margen interior de página ---una vez eliminado el centímetro (\textsc{bcor}\texttt{1cm}) que se deja para compensar el encuadernado--- es la mitad del margen exterior de página. De ese modo se crea una distribución de página en la que el margen interior de página de las páginas par e impar juntas es igual a cada uno de los márgenes exteriores de ambas páginas.

\begin{figure}
	\pagediagram
	\caption{Distribución del texto en las páginas de este documento}
	\label{fig:layouts}
\end{figure}

\begin{figure}\ContinuedFloat
	\currentpage
	\pagedesign
	\caption[]{Continuación del \vref{fig:layouts}}
\end{figure}


\section{Gráficos con MetaPost}

Anotaciones destinadas a obtener gráficos escalables de calidad empleando el paquete MetaPost para la creación de gráficos en PostScript.


\subsection{Tamaño}

Tengo que modificar el tamaño de algunas figuras al haber cambiado la tipografía predeterminada de la \emph{Computer Roman} a \emph{Lucida}. Lo que quiero que aparezca en este apartado es un cuadro con los distintos tamaños de letra que aparecen en el documento.
