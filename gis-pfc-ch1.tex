\chapter{Subsistema de interacción con el medio físico}

\section{Introducción al sistema digital de medida}

Un sistema de medida es cualquier instrumento formado por más de un
elemento que permite a quien lo usa evaluar una determinada propiedad de un
objeto, medio o evento. Para una definición más rigurosa de un sistema
electrónico de medida puede optarse por la encontrada en
\cite{pallas2004sas}, y dice lo siguiente:

{\small\begin{quotation}
	Se denomina sistema a la combinación de dos o más elementos,
	subconjuntos y partes necesarias para realizar una o varias
	funciones. En los sistemas de medida, esta función es la asignación
	objetiva y empírica de un número a una propiedad o cualidad de un
	objeto o evento, de tal forma que la describa.
\end{quotation}}

Un sistema de medida digital está constituido habitualmente por los
elementos mostrados en el esquema de la \cref{fig:digmeasstm}. Pueden
agruparse estos elementos según la función que desempeñan en el sistema, si
se hace de ese modo se encuentran tres subsistemas: un subsistema para la
interacción con el medio físico, un subsistema de adquisición, y un
subsistema de control y presentación. Puede observarse entonces el sistema
como una pila de capas superpuestas, en el que cada capa provee de servicio
a la capa inmediatamente superior y abstrae las capas inferiores, en cada
una de las capas se situaría cada uno de los subsistemas propuestos. De ese
modo cada subsistema puede estudiarse por separado con independencia de los
demás, así se ha hecho en esta memoria. Consecuentemente cada subsistema da
lugar a cada uno de los tres capítulos que empezando por este describen el
sistema digital de medida implementado durante el curso del desarrollo de
este proyecto.

\begin{figure}
	\begin{center}
		\includegraphics{gis-pfc-ch1-01.mps}
	\end{center}
	\caption[Sistema digital de medida] {Distintos elementos
	funcionales que conforman el sistema digital de medida.}
	\label{fig:digmeasstm}
\end{figure}

Si se define una jerarquía en la que el nivel más bajo es aquel ocupado por
los elementos en contacto directo con el medio, y se tiene como nivel más
alto aquel en el que se encuentran los elementos que interactúan con el
supervisor, entonces los niveles ordenados de inferior a superior en la
jerarquía establecida se encuentran ocupados por los subsistemas precisados
anteriormente del siguiente modo: subsistema para la interacción con el
medio, subsistema de adquisición, y subsistema de control y presentación.
El orden en el que se han dispuesto los capítulos que tratan acerca de la
configuración del sistema digital de medida, los tres primeros capítulos,
coincide con el orden que guardan los distintos subsistemas en esta
jerarquía.



Para que un sistema pueda considerarse un sistema de medida debe haber una
interacción con el objeto o evento de estudio. El propósito de este sistema
es medir las propiedades de un determinado medio físico, en este caso
madera de palmera. La interacción con el medio ocurre en el primer
subsistema que se encuentra dentro de la jerarquía de capas que define el
sistema de medida. Atendiendo a su función dentro del sistema, este
subsistema recibe en esta memoria el nombre de subsistema para la
interacción con el medio físico. Es el único de los tres subsistemas que
conforman el sistema de medida que está directamente relacionado con los
ensayos no destructivos, mientras que el resto de subsistemas, el
subsistema de adquisición y el subsistema de control y presentación se
encuentran más ligados al tratamiento digital de señales. En una inspección
ultrasónica, el subsistema para la interacción con el medio físico, este
subsistema, es imprescindible y sin él no puede realizarse el ensayo; sin
embargo, los otros dos subsistemas pueden sustituirse por otros elementos
que realicen la misma función, por un osciloscopio digital, por ejemplo. El
subsistema para la interacción con el medio físico está compuesto por los
elementos presentes en la figura \cref{fig:submedium}. Los transductores
son los dispositivos que interactúan con el medio propiamente, a través de
la emisión de una señal acústica que se recibe en el otro extremo. Por su
parte los acondicionadores adecuan las señales de tal modo que, pueda haber
una comunicación entre los elementos presentes en este subsistema y los
pertenecientes a subsistemas por encima de éste.

\begin{figure}
	\begin{center}
		\includegraphics{gis-pfc-ch1-01.mps}
	\end{center}
	\caption[Subsistema para la interacción del medio físico]{Esquema
	que representa los distintos elementos o bloques presentes en el
	subsistema para la interacción con el medio físico.}
	\label{fig:submedium}
\end{figure}

Este capítulo se divide pues en tres apartados, en el primero se hace una
descripción técnica de los transductores utilizados en el proyecto, y se
hace una pequeña observación de su utilidad en relación con el tipo de
ensayo programado para el proyecto. En los dos apartados posteriores se
explica el proceso de diseño de los circuitos acondicionadores de la
sección de transmisión y de la sección de recepción respectivamente, se
exponen los requisitos que deben satisfacer y se explica brevemente el
funcionamiento de los circuitos propuestos para que el lector pueda
comprobar que satisfacen dichos requisitos. En el \cref{chap:endus} se hace
un desarrollo de la teoría de los ensayos no destructivos que puede ayudar
a comprender mejor algunos de los conceptos propuestos a lo largo de este
mismo capítulo.


\section{Transductores}
