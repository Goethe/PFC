\chapter{Subsistema de interacción con el medio físico}

\section{Introducción al sistema digital de medida}

Un sistema de medida es cualquier instrumento formado por más de un
elemento que permite a quien lo usa evaluar una determinada propiedad de un
objeto, medio o evento. Para una definición más rigurosa de un sistema
electrónico de medida puede optarse por la encontrada en
\cite{pallas2004sas}, y dice lo siguiente:

{\small\begin{quotation}
	Se denomina sistema a la combinación de dos o más elementos,
	subconjuntos y partes necesarias para realizar una o varias
	funciones. En los sistemas de medida, esta función es la asignación
	objetiva y empírica de un número a una propiedad o cualidad de un
	objeto o evento, de tal forma que la describa.
\end{quotation}}

Un sistema de medida digital está constituido habitualmente por los
elementos mostrados en el esquema de la \cref{fig:digmeasstm}. Pueden
agruparse estos elementos según la función que desempeñan en el sistema, si
se hace de ese modo se encuentran tres subsistemas: un subsistema para la
interacción con el medio físico, un subsistema de adquisición, y un
subsistema de control y presentación. Puede observarse entonces el sistema
como una pila de capas superpuestas, en el que cada capa provee de servicio
a la capa inmediatamente superior y abstrae las capas inferiores, en cada
una de las capas se situaría cada uno de los subsistemas propuestos. De ese
modo cada subsistema puede estudiarse por separado con independencia de los
demás, así se ha hecho en esta memoria. Consecuentemente cada subsistema da
lugar a cada uno de los tres capítulos que empezando por este describen el
sistema digital de medida implementado durante el curso del desarrollo de
este proyecto.

\begin{figure}
	\begin{center}
		\includegraphics{gis-pfc-ch1-01.mps}
	\end{center}
	\caption[Sistema digital de medida] {Distintos elementos
	funcionales que conforman el sistema digital de medida.}
	\label{fig:digmeasstm}
\end{figure}

Si se define una jerarquía en la que el nivel más bajo es aquel ocupado por
los elementos en contacto directo con el medio, y se tiene como nivel más
alto aquel en el que se encuentran los elementos que interactúan con el
supervisor, entonces los niveles ordenados de inferior a superior en la
jerarquía establecida se encuentran ocupados por los subsistemas precisados
anteriormente del siguiente modo: subsistema para la interacción con el
medio, subsistema de adquisición, y subsistema de control y presentación.
El orden en el que se han dispuesto los capítulos que tratan acerca de la
configuración del sistema digital de medida, los tres primeros capítulos,
coincide con el orden que guardan los distintos subsistemas en esta
jerarquía.


\section{Introducción}

El primero de los subsistemas estudiado en este documento es el subsistema
para la interacción con el medio físico. Es el subsistema que ocupa la capa
de menor nivel y agrupa los elementos que pueden observarse en la
\cref{fig:submedium}.

\begin{figure}
	\begin{center}
		\includegraphics{gis-pfc-ch1-02.mps}
	\end{center}
	\caption[Subsistema para la interacción con el medio físico]
	{Elementos del sistema digital de medida que también se incluyen en
	el subsistema para la interacción con el medio físico.}
	\label{fig:submedium}
\end{figure}

% >>|Fecha indeterminada|
%
% Anotaciones extraídas del antiguo tercer capítulo (diseño del dispositivo
% de medida)
%
%	Esta sección está dedicada al proceso de diseño que se lleva a cabo
%	para implementar mediante los recursos disponibles la
%	instrumentación electrónica que junto con la tarjeta de adquisición
%	y el software de control forman parte del aparato empleado en las
%	mediciones efectuadas durante este proyecto fin de carrera.
%
%	En apartados posteriores se observa que este primer diseño carece
%	de las propiedades requeridas en los experimentos programados, pero
%	igualmente se ha decidido incorporar a la memoria las notas
%	correspondientes a esta fase del proyecto por considerar que al
%	tratar una disciplina distinta con respecto al resto del documento
%	\mbox{---diseño} e implementación de instrumentación electrónica---
%	enriquecen el contenido del mismo.
%
%
% \subsection{Tipos de ensayo programados}
%
%	Es la naturaleza de los experimentos la que dictamina cuales deben
%	ser las características del instrumento de medida. Sin olvidar el
%	propósito primero de este proyecto ---determinar un método efectivo
%	para la detección de defectos en madera de palmera empleando para
%	ello ensayos no destructivos con señales de ultrasonidos--- se
%	programan dos tipos de prueba:
%
%	\begin{itemize}
%		\item Por un lado, pruebas de laboratorio con madera de
%		palmera talada y preparada.
%		\item Por otro, pruebas de campo con palmeras en un entorno
%		natural, no siendo necesario que las palmeras sean
%		silvestres, pudiendo encontrarse plantadas en un jardín o
%		espacio similar.
%	\end{itemize}
%
%	Se escoge la técnica de transmisión, en defecto de la técnica de
%	pulso"=eco, para la realización de los ensayos. El equipo
%	disponible al inicio del trabajo experimental varía en función de
%	la prueba: para los experimentos de laboratorio se cuenta con un
%	sistema de adquisición como el conformado por el conjunto de la
%	tarjeta de adquisición, un ordenador, el software de control y la
%	caja de conexiones; y para las pruebas de campo con un osciloscopio
%	y una fuente de alimentación, ambos alimentados con una batería. En
%	ambos tipos de prueba deben emplearse también dos transductores,
%	uno en transmisión y otro en recepción, junto con sus respectivos
%	circuitos acondicionadores.
%
%
% \subsection{Requisitos del instrumento de medida}
%
%	Para justificar cuales son los requisitos aplicables a los
%	transductores y a las etapas acondicionadoras que los preceden, se
%	recuerda aquí de forma resumida al lector una serie de conclusiones
%	a las que se ha llegado a lo largo de la sección anterior (la
%	\cref{sec:theory} con inicio en la \vpageref{sec:theory}).
%
%	Los tres parámetros cruciales en el desempeño de un \sig{endus}
%	son: la resolución, tanto axial como lateral; la profundidad de
%	penetración; y la relación señal a ruido.
%
%	\begin{itemize}
%		\item La resolución viene determinada principalmente por el
%		tamaño y el tipo de transductores empleados. Cuanto mayor
%		sea el ancho de banda de los transductores y menores sus
%		dimensiones, se alcanzarán mejores resoluciones. Asimismo,
%		una buena resolución depende de la forma y frecuencia de
%		los pulsos ultrasónicos, siendo mejor cuanto más estrecho
%		sea el pulso y mayor su frecuencia.
%		\item La profundidad de penetración, factor determinante en
%		los experimentos realizados para este proyecto, depende
%		fundamentalmente de la potencia transmitida en cada pulso y
%		de su frecuencia de oscilación, siendo mayor cuanto mayor
%		sean estos dos parámetros.% Aumenta la profundidad de
%		penetración a medida que aumenta la potencia transmitida y
%		DISMINUYE la frecuencia de oscilación del pulso.
%		\item Sin embargo, la consecución de una buena relación
%		señal a ruido está reñida con una buena profundidad de
%		penetración. El ruido estructural, principal factor
%		limitante de la relación señal a ruido, se agrava a medida
%		que aumenta la frecuencia o la potencia del pulso
%		ultrasónico. Para poder realizar ensayos capaces de
%		detectar defectos en el medio a profundidades razonables
%		sin sufrir en exceso los efectos del ruido de grano es
%		determinante aplicar un buen algoritmo para la reducción
%		del ruido.
%	\end{itemize}
%
%	Por supuesto, la calidad de un ensayo exige buenas configuraciones
%	de estos tres parámetros. Sin embargo, y dado que los ensayos
%	realizados para el proyecto no requieren la detección precisa del
%	defecto, si no sólo su detección, una buena resolución no sería
%	exigible aunque sí deseable. Por el contrario, el material objeto
%	de este estudio es la madera de palmera, puesto que desde la
%	planificación del proyecto se preven pruebas de campo en palmeras
%	vivas, y además se opta por emplear la técnica de transmisión
%	(léase el \cref{subsec:technics}), resulta imprescindible trabajar
%	con una buena profundidad de penetración, de forma que la señal de
%	ultrasonidos sea capaz, al menos, de atravesar el tronco de las
%	palmeras.% En base a los requisitos establecidos por las
% %	características de los ensayos se extraen las necesidades de
% %	potencia y frecuencia que deben cubrir los transductores y los
% %	circuitos acondicionadores, así como el resto de instrumentos
% %	utilizados.
%
%
% \paragraph{Requisitos de frecuencia.}
%
%	La frecuencia de las señales ultrasónicas empleadas en ensayos no
%	destructivos oscila generalmente entre los 20 kHz y los 25 MHz. Sin
%	embargo, existen límites prácticos que impiden trabajar en un rango
%	de frecuencias tan amplio. En primer lugar, el precio de los
%	transductores aumenta significativamente a medida que aumenta su
%	ancho de banda, la posibilidad de emitir pulsos de alta frecuencia
%	también encarece el precio de estos dispositivos\footnote{Es
%	habitual encontrar sensores que trabajan a frecuencias entorno a
%	los 40 kHz por un precio entorno a las \mbox{5 \pounds{}}, mientras
%	que sensores que funcionan a 300 kHz pueden costar aproximadamente
%	unas \mbox{300 \pounds{}} (referencia extraída de la página web
%	\href{http://uk.farnell.com}{uk.farnell.com})}. Por otro lado, la
%	banda de paso de las etapas de acondicionamiento, la frecuencia
%	máxima obtenible de la fuente de alimentación, y la frecuencia de
%	muestreo máxima del instrumento de adquisición, limitan la
%	frecuencia de la señal de ultrasonidos.
%
%	Obviamente, es el límite más restrictivo el que se aplica. En este
%	caso, el límite más restrictivo lo imponen los transductores y la
%	tarjeta de adquisición. Los transductores a los que se ha tenido
%	acceso en el desarrollo del proyecto son de bajo coste y su
%	frecuencia de oscilación natural se ubica en los 40 kHz, podría
%	decirse que en el límite inferior de la banda de frecuencias
%	empleada en los \sig{endus}. Por su parte, la tarjeta de
%	adquisición ofrece en condiciones óptimas una tasa de muestreo de
%	100 KS/s. Lo que implica que, para su correcta visualización en el
%	dispositivo de adquisición empleado en el laboratorio, la
%	frecuencia de la señal analógica que se muestrea debe encontrarse
%	alrededor de los 20 kHz.
%
%	Quizá el lector haya advertido que esta relación ---frecuencia de
%	muestreo unas cinco veces superior a la frecuencia de la señal
%	analógica--- supera con creces lo establecido por el teorema de
%	Nyquist, $f_s = 2\Delta f$. Esto es debido a que el objetivo del
%	muestreo es visualizar en el monitor del ordenador la forma de la
%	señal a partir de la reconstrucción que \matlab{} hace de las
%	muestras. Al contrario de lo que ocurre con el teorema de
%	Nyquist"=Shannon, que determina la frecuencia de muestreo mínima
%	para que una señal analógica pueda ser recuperada a partir de sus
%	muestras no cuantificadas sin pérdida de información, para percibir
%	la forma de una señal en una reconstrucción arbitraria de la misma
%	realizada partiendo de sus muestras cuantificadas la frecuencia de
%	muestreo necesaria debe ser muy superior a la indicada por el
%	teorema.
%
%
% \paragraph{Requisitos de potencia.}
%
%	Puesto que la frecuencia de oscilación de los pulsos ultrasónicos
%	está limitada a los 40 kHz por los motivos prácticos citados en el
%	apartado anterior, y dada la necesidad de una profundidad de
%	penetración notable, los requisitos en cuanto a la potencia que
%	debe manejarse en las etapas de transmisión y recepción se
%	encuentran muy restringidos.
%
%	Para suplir las carencias que los transductores tienen en
%	frecuencia es lógico recurrir a señales de mayor potencia, sin
%	embargo el precio reducido de los dispositivos empleados también
%	limita su capacidad de emitir señales de alta potencia.
%
%	Lo que en última instancia significa que, con ninguna de las
%	posibles soluciones al alcance del proyecto puede obtenerse la
%	profundidad de penetración adecuada para el tipo de experimentos
%	que se desea realizar. Léase a continuación, en la
%	\vpageref{subsec:solution}, la solución finalmente adoptada y la
%	valoración preliminar que el autor hace con antelación a la fase de
%	diseño.
%
%
% \subsection{Recursos disponibles al inicio de la etapa de diseño}
%
%	Para la implementación del instrumental electrónico se dispone
%	inicialmente de dos transductores ultrasónicos, uno destinado a
%	funcionar como transistor y otro que realiza el papel de receptor.
%	Ambos transductores manufacturados por el mismo fabricante y de una
%	misma gama, con características similares. En el
%	\vref{tab:transducers} se muestran las principales propiedades de
%	ambos transductores.
%
%	Además, el fabricante proporciona en la ficha técnica la respuesta
%	característica del transmisor a un pulso rectangular. La curva, que
%	puede observarse en la \cref{fig:response}, se asemeja a un pulso
%	gaussiano que modula a una sinusoide de frecuencia igual a la
%	frecuencia de resonancia del transductor. Esta respuesta a señales
%	pulsadas de pulsos rectangulares hace que este tipo de transductor
%	sea muy apropiado para su uso en \sig{endus}.
%
%	En cuanto a su aspecto y dimensiones, los transductores se
%	encuentran protegidos por un encapsulado plástico de color negro,
%	pudiendo distinguir el transmisor gracias a un punto blanco situado
%	en la pared exterior del encapsulado. Las dimensiones de los dos
%	transductores coinciden, y son las mostradas en la figura
%	esquemática \ref{fig:transducer}.
%
%	\begin{table}
%		\centering
%		\begin{threeparttable}
%		\begin{tabular}{l c c}
%			\toprule
%			& \multicolumn{2}{c}{Transductor} \\
%			\cmidrule(l){2-3}
%			Propiedad & Transmisor & Receptor \\
%			\midrule
%			Sensibilidad en transmisión\tnote{a} %
%			& 110 dB & -- \\
%			Sensibilidad en recepción\tnote{b} %
%			& -- & -70 dB \\
%			Frecuencia de resonancia (kHz) %
%			& $40 \pm 1$ & $40 \pm 1$ \\
%			Ancho del lóbulo principal %
%			& 60\,º & -- \\
%			Tensión de alimentación máxima\tnote{c} %
%			& 10 Vef & -- \\
%			Máxima potencia admitida\tnote{d} %
%			& 200 m\!Wef & \\
%			Impedancia\tnote{e} %
%			& $700\ \Omega$ & $30\ \text{k}\Omega$ \\
%			Capacidad (nF) %
%			& $2 \pm 20\%$ & $2 \pm 20\%$ \\
%			Tiempo de subida del pulso %
%			& $700\ \mu\text{s}$ & -- \\
%			Tensión máxima admitida en modo pulsado %
%			& 20 V & -- \\
%			Rango de temperatura de funcionamiento (ºC) %
%			& (-20, 60) & (-20, 60) \\
%			\bottomrule
%		\end{tabular}
%		\begin{TableNotes}
%			\tnotetext{a}{Decibelios medidos con respecto a %
%			una referencia de 200 nbar a 30 cm cuando el %
%			transmisor se alimenta con una tensión eficaz de %
%			10 Vef.}
%			\tnotetext{b}{Decibelios medidos con respecto a %
%			una referencia de 1 V/$\mu\text{bar}$.}
%			\tnotetext{c}{Esta es la tensión máxima que el %
%			transductor admite cuando se alimenta de forma %
%			continuada con una señal de alterna.}
% %			\tnotetext{d}{Se emplea en este documento el %
% %			símbolo Vef como sinónimo de la unidad de %
% %			potencial eléctrico en corriente alterna, %
% %			\emph{voltios eficaces}. Del mismo modo se %
% %			utiliza Wef, \emph{vatios eficaces}, como unidad %
% %			de potencia eficaz.}
%			\tnotetext{d}{Se refiere a la potencia máxima que %
%			el transductor puede disipar en modo pulsado, es %
%			decir, cuando la alimentación es interrumpida y %
%			periódica.}
%			\tnotetext{e}{En transmisión esta magnitud %
%			representa la impedancia de entrada equivalente %
%			del transductor. En recepción, por el contrario, %
%			representa la impedancia de salida del %
%			transductor.} %
%		\end{TableNotes}
%		\end{threeparttable}
%		\caption[Características de los transductores empleados %
%		en el dispositivo de medida]{Características de los %
%		transductores empleados en el dispositivo de medida (se %
%		proporcionan los valores típicos para cada propiedad).}
%		\label{tab:transducers}
%	\end{table}
%
%	\begin{figure}
%		\begin{center}
%			\includegraphics{gis-pfc-ch1-03.mps}
%		\end{center}
%		\caption[Respuesta del transmisor a un pulso rectangular]%
%		{Respuesta del transmisor a un pulso rectangular.}
%		\label{fig:response}
%	\end{figure}
%
%	\begin{figure}
%		\begin{center}
%			\includegraphics{gis-pfc-ch1-03.mps}
%		\end{center}
%		\caption[Dimensiones de los transductores de ultrasonidos]%
%		{Dimensiones de los transductores de ultrasonidos.}
%		\label{fig:transducer}
%	\end{figure}
%
%	El resto de componentes empleados en la implementación de las
%	etapas de acondicionamiento se encontraban disponibles en el
%	laboratorio de teoría de la señal en el que esta tarea se lleva a
%	cabo. Entre ellos cabe destacar el \sig{TL}082, un amplificador
%	operacional de propósito general cuyas prestaciones superan
%	ampliamente a las ofrecidas por el clásico \AOUA{} en términos de
%	banda de frecuencias amplificada. Mientras que el \AOUA{}
%	convencional mantiene su máxima ganancia para señales de entrada de
%	frecuencia alrededor de los 7 kHz, el amplificador empleado
%	consigue amplificar con su máximo factor de ganancia señales que
%	oscilan con frecuencias máximas de 90 kHz. Es imprescindible
%	disponer de un amplificador que conserve su ganancia por encima de
%	los 40 kHz, ya que esta es la frecuencia de resonancia de los
%	transductores, y la frecuencia de trabajo seleccionada para el
%	dispositivo de medida.
%
%
% \subsection{Solución adoptada}\label{subsec:solution}
%
%
% \subsubsection{Valoración preliminar}
%
%
%	A la vista de los requisitos que exige una aplicación como la
%	propuesta en este proyecto, las necesidades que surgen durante su
%	diseño y los medios y recursos disponibles, parece previsible un
%	fracaso en el desarrollo de la parte experimental de este
%	\sig{pfc}. Pese a todo, y puesto que el coste de implementar los
%	circuitos y realizar las pruebas es en la práctica suficientemente
%	reducido, se decide proseguir con los experimentos ya que el
%	verdadero fin del proyecto es obtener una serie de resultados, sean
%	estos los esperados o no, y de estos deducir una serie de
%	conclusiones.
%
%	En el momento en el que se desarrolla esta fase del proyecto se
%	contempla la posibilidad de inferir dos conclusiones distintas a
%	partir de los resultados: si los resultados no son buenos,
%	determinar cuáles son en realidad los requisitos que deben
%	cumplirse para poder aplicar los \sig{endus} de manera efectiva en
%	la detección de defectos en madera de palmera, y sabiendo esto
%	justificar la ausencia de resultados válidos; y, en el caso
%	contrario, en caso de que los resultados fueran satisfactorios,
%	verificar la validez de aplicar este método con un mismo propósito
%	y en unas condiciones similares.
%
%	Posteriormente, tal como se relata en la {\color{red}sección ??} se
%	tiene acceso a unos transductores distintos, que por su modo de
%	funcionamiento permiten llevar a cabo de modo satisfactorio las
%	pruebas programadas.
%
%
% \subsubsection{Montaje seleccionado}
%
%
%	En vista de las conclusiones extraídas en apartados anteriores, y
%	teniendo en cuenta las soluciones a las que se tiene acceso, con el
%	propósito de obtener los mejores resultados posibles, se opta por
%	un montaje que exprime al máximo el transmisor y que en la sección
%	de recepción utiliza un amplificador de instrumentación cuyo factor
%	de ganancia es muy elevado.% Para ello, de las soluciones a las
% %	que se tiene acceso
%
%	Esta solución, que en una primera instancia parece adecuada,
%	presenta no obstante importantes inconvenientes, de entre los
%	cuales destaca la baja calidad de relación señal a ruido que
%	proporciona.
%
%	En principio, el uso de señales de elevada potencia que oscilan a
%	bajas frecuencias es propicio para evitar las complicaciones que en
%	un ensayo introduce el ruido de grano. Si bien la señal que emite
%	el transductor de la sección de transmisión no es de una potencia
%	considerable, las bajas frecuencias a las que oscila garantizan que
%	se consiga en parte este objetivo.
%
%	No obstante no es el ruido de grano el más perjudicial en esta
%	configuración. Los transductores de ultrasonidos son transductores
%	piezoeléctricos y, por tanto, el sensor empleado en recepción se
%	engloba en la categoría de los sensores generadores. Es decir, que
%	en respuesta a una actividad ultrasónica genera una determinada
%	señal eléctrica. La frecuencia y amplitud de la señal eléctrica que
%	proporciona el sensor dependen de la señal acústica detectada. Y,
%	si como ocurre en este caso, la señal ultrasónica que alcanza el
%	sensor es muy tenue, la señal eléctrica correspondiente muestra un
%	recorrido de tensión muy reducido. De tal forma que incluso puede
%	ser comparable al ruido presente a la entrada del circuito de
%	adquisición o a otras señales indeseadas.
%
%	Por consiguiente, es, en la sección de recepción, el ruido
%	electrónico procedente de la etapa de acondicionamiento el más
%	significativo y el de más difícil solución. Por tres razones que se
%	detallan a continuación.
%
%
%	\begin{itemize}
%		\item En primer lugar, debido a que la procedencia del
%		ruido nada tiene que ver con la interacción de la señal
%		acústica con el medio, las técnicas de postprocesado cuyo
%		objetivo es eliminar el ruido procedente del entorno no son
%		aplicables.
%		\item En principio, es posible aplicar las técnicas
%		destinadas a reducir la presencia de ruido gaussiano vistas
%		en el \cref{subsec:noise}. Sin embargo, con dificultad
%		pueden obtenerse mejoras aceptables de la \sig{snr} en
%		circunstancias como las aquí expuestas.
%		\item Otra posible solución podría encontrarse al abordar
%		el problema desde el prisma de la teoría de la
%		instrumentación electrónica. Esta teoría ha mostrado ser
%		eficaz resolviendo problemas similares en el
%		acondicionamiento de sensores de múltiples tipos. Pueden
%		citarse como ejemplo: el uso del puente de Wheatstone en el
%		acondicionamiento de sensores resistivos, circuito que
%		elimina la componente de señal invariante que introduce el
%		divisor de tensión; o la utilización de células isotermas y
%		fuentes de corriente dependientes de la temperatura para
%		eliminar la componente de señal que generan las uniones de
%		un termopar con los cables de medida.
%		No obstante dadas las características del sensor y de la
%		señal que genera, la aplicación de una solución semejante
%		es ciertamente difícil.
%	\end{itemize}
%
%
% \subsection{Diseño de la etapa de transmisión}
% % {Requisitos concretos aplicables a la etapa de acondicionamiento en
% % transmisión}
% % [Requisitos de la etapa de transmisión]
%
%
%	El propósito del circuito acondicionador de la etapa de transmisión
%	es proporcionar una señal de alimentación que se ajuste a las
%	limitaciones del transductor y a los requisitos especificados en la
%	etapa anterior de la fase de diseño.
%
%
% \subsubsection{La señal de alimentación}
%
%
%	Las especificaciones
%
%	Los requisitos de la etapa de transmisión pueden obtenerse al
%	juzgar por un lado los límites del transductor de ultrasonidos que
%	se emplea como transmisor y por otro las especificaciones que debe
%	cumplir la señal acústica transmitida.
%
%	En \sig{endus}, la técnica de transmisión exige una señal pulsada
%	como la que proporciona el transmisor cuando es alimentado por un
%	pulso rectangular. Por otro lado, una de las condiciones de diseño
%	consiste en emitir a la máxima potencia posible. La señal acústica
%	es emitida por el transmisor de ultrasonidos, por consiguiente la
%	potencia de esta señal depende de los límites del transductor y de
%	como se encuentra alimentado.
%
% % >>|Comentario dentro de comentario, fecha indeterminada|
%
% %	Estoy hablando de las características de la señal de alimentación,
% %	porqué no hacer una subsubsección hablando del tema. Es decir, a
% %	partir de los datos que tengo razono como debe ser la señal de
% %	alimentación y luego digo como la implemento.
%
% %	Consultando el \vref{tab:transducers} puede observarse que la
% %	máxima potencia que puede disipar el transductor es de 200 m\!Wef
% %	en modo pulsado. Puesto que la impedancia de entrada típica interna
% %	del transductor es de $700\ \Omega$ pueden calcularse de forma
% %	sencilla valores para la potencia instantánea y, a partir de este
% %	último dato, el nivel de tensión máximo con el que debe alimentarse
% %	el transductor. Para realizar estos cálculos, primero debe fijarse
% %	otro parámetro: la forma de la señal de alimentación y sus
% %	características temporales.
%
% %	¿Por qué elijo una señal pulsada? Porque así se generan pulsos de
% %	forma automática sin necesidad de presionar ningún botón, ni
% %	realizar ninguna otra acción. Esto es bueno porque me ahorro
% %	interruptores que quizá deban ser reemplazados en el futuro por
% %	perder parte de sus cualidades. El periodo con el que se generan
% %	los pulsos no debería ser demasiado corto, lo que impediría
% %	explotar al máximo el transductor, y debe permitirme seguir los
% %	cambios de señal en el monitor del osciloscopio.
%
% %	Resulta conveniente elegir una señal pulsada para la alimentación,
% %	en concreto un tren de pulsos rectangulares, dado el comportamiento
% %	del transductor frente a este tipo de pulsos. En cuanto al ciclo de
% %	trabajo por conveniencia se elige, respectivamente: la duración del
% %	estado alto, 375 $\mu$s, equivalente a 15 periodos de la portadora;
% %	y una frecuencia de aproximadamente 1 Hz.% Una frecuencia tan baja
% %	permite un seguimiento visual de la señal, de ese modo es posible
% %	congelar el osciloscopio manualmente y eso simplifica el proceso de
% %	medida.
%
% %	Consultando el \vref{tab:transducers} puede observarse que la
% %	máxima tensión admitida por el transductor en modo pulsado es de 20
% %	V, y la máxima potencia que puede disipar es de 200 mW. Por lo
% %	tanto, la corriente de la señal de alimentación debe limitarse a 10
% %	mA, algo que puede lograrse mediante una carga adicional. En cuanto
% %	a la duración de los pulsos y el periodo de la señal pulsada, por
% %	conveniencia se elige respectivamente un tiempo de 375 $\mu$s,
% %	equivalente a 15 periodos de la portadora, y una frecuencia de
% %	$1/4$ Hz. Una frecuencia tan baja permite un seguimiento visual de
% %	la señal, de ese modo es posible disparar el osciloscopio
% %	manualmente y eso simplifica el proceso de medida.
%
% %	Consultando el \vref{tab:transducers} y la \cref{fig:response}
% %	puede deducirse cuales son las características de la señal de
% %	alimentación que debe proporcionar la etapa de acondicionamiento en
% %	transmisión. Al emplear la técnica de transmisión es necesario que
% %	la señal que emite el transductor sea una señal pulsada
%
% %	En las pruebas de laboratorio basta con emplear el generador de
% %	señales y, si se requiere, un pequeño amplificador de ganancia
% %	moderada. Por ejemplo, el generador dispuesto para el proyecto
% %	proporciona una señal con valor de pico a pico máximo de 10 voltios
% %	pico"=pico (en adelante Vpp), puesto que va a emplearse el
% %	transmisor en modo pulsado, la ganancia del amplificador debe ser
% %	de 2 V/V. Ganancia que puede obtenerse fácilmente con un
% %	amplificador monoetapa basado en un amplificador operacional de
% %	propósito general.
%
% %	Para las pruebas de campo se diseña un circuito capaz de emitir una
% %	señal pulsada, de pulsos rectangulares, que pueda ser alimentado
% %	con la misma fuente de alimentación con la que es alimentado el
% %	acondicionador de la sección de recepción. De ese modo se evita el
% %	consumo extra que el generador de señales hace de la batería al
% %	mismo tiempo que se transporta un equipo menos. El circuito
% %	mencionado se basa en el temporizador integrado 555, concretamente
% %	en el integrado 556 que contiene dos 555 en un mismo encapsulado, y
% %	el amplificador de uso común \AOUA{}.
%
% %	El 555 en su configuración como multivibrador astable es capaz de
% %	producir una señal periódica de pulsos rectangulares en los que la
% %	duración del estado alto y del estado bajo ---y, por tanto, del
% %	periodo--- dependen de la configuración de la red temporal a la que
% %	se conecta el circuito. Además tiene la capacidad de permanecer en
% %	reposo mientras se mantiene activada la puerta de reset (activa a
% %	nivel bajo). El ciclo de trabajo de la señal que genera el 555 es
% %	siempre superior al 50\% lo que implica que para lograr una señal
% %	como la que se necesita para alimentar al transmisor es necesario
% %	otro 555, en este caso en la configuración de multivibrador
% %	monoestable.
%
% %	En su configuración monoestable, el 555 es capaz de generar pulsos
% %	rectangulares de duración muy precisa. El inicio del pulso lo
% %	provoca una caída del nivel en la puerta de disparo por debajo del
% %	umbral de comparación que establece la tensión de alimentación, en
% %	concreto el umbral se establece en una tensión equivalente a un
% %	tercio de Vcc. Es decir que el circuito es activo a nivel bajo. Por
% %	otro lado, la duración del pulso viene definida por la red RC que
% %	se conecta a los terminales del integrado.
%
% %	Aquí tengo que hablar del amplificador que hago con el uA741, (con
% %	el BJT es imposible, Vbe es muy superior a 10 mVpp), y de como
% %	cargo la salida de este con el transductor y una resistencia
% %	adicional para que la corriente en bornes del transmisor no supere
% %	los 10 mA.
%
% %	En el párrafo a continuación tengo que decir que en el cuadro
% %	aparecen también los valores de consumo del circuito y, si acaso,
% %	debo decir porque no puedo alimentarlo con una pila.
%
% %	La configuración propuesta se muestra en la
% %	\vref{fig:traconditioner}, y los valores de tiempos y frecuencia
% %	correspondientes a ésta pueden verse a continuación en el
% %	\cref{tab:traconditioner}.
%
% %	\begin{figure}
% %		\begin{center}
% %			\includegraphics{gis-pfc-ch1-03.mps}
% %		\end{center}
% %		\caption[Acondicionador en la sección de transmisión]%
% %		{Configuración seleccionada para el acondicionador en la
% %		 sección de transmisión.}
% %		\label{fig:traconditioner}
% %	\end{figure}
%
% %	\begin{table}
% %		\centering
% %			\includegraphics{gis-pfc-ch1-03.mps}
% % %		\begin{tabular}{<+dimensions+>}
% % %			<++>
% % %		\end{tabular}
% %		\caption{Valores temporales y de frecuencia para la %
% %		configuración mostrada en la \cref{fig:traconditioner}.}
% %		\label{tab:traconditioner}
% %	\end{table}
%
%
% %	DEBERÍA HABLAR SOBRE LOS REQUISITOS QUE SE APLICAN A LOS CIRCUITOS
% %	ACONDICIONADORES TAMBIÉN
%
% %	Podría llamar a unos nuevos subapartados, requisitos concretos de
% %	la etapa de acondicionamiento en transmisión/recepción.
%
% %	La etapa de transmisión debe proporcionar una señal que alimente al
% %	transductor, señal de 40 kHz de amplitud variable, controlada por
% %	el usuario (control por tensión).
%
% %	La etapa de recepción debe cargar adecuadamente el sensor,
% %	amplificar la señal procedente de éste (por tanto, la banda
% %	amplificada debe encontrarse centrada en la frecuencia de
% %	resonancia del sensor, en los 20 kHz), y en principio debe evitarse
% %	la distorsión en la fase lo que perjudicaría los resultados de una
% %	hipotética aplicación de técnicas de postprocesado como la
% %	\sig{ssp}
%
% %	El razonamiento para obtener las condiciones de diseño de la etapa
% %	de acondicionamiento en recepción es similar si no bien el inverso.
% %	La banda de paso de amplificación debe ser la misma (desde los 10
% %	hasta aproximadamente los 40 kHz) y la ganancia la máxima que pueda
% %	obtenerse sin distorsión ---especialmente de la fase---, debido
% %	fundamentalmente a que se trabaja con señales muy atenuadas y a que
% %	se preve trabajar en la fase de post"=adquisición con técnicas de
% %	\textsc{ssp}.
% <<<


% >>|Fecha indeterminada|
%
% Anotaciones extraídas del antiguo segundo capítulo (sección de
% rendimiento)
%
%	Para poder representar fielmente una señal sólo a partir de sus
%	muestras es necesario muestrearla a una frecuencia superior a la
%	estipulada por el teorema de Nyquist. En la práctica es suficiente
%	muestrear señales periódicas y limitadas en banda a una frecuencia
%	de muestreo superior a unas cinco veces la frecuencia de oscilación
%	máxima de la señal.
%
%	Si la señal se muestrea a una tasa inferior mucha información se
%	pierde y el resultado de aplicar algoritmos de interpolación es
%	insuficiente. Eso lo que significa al final es que la señal que me
%	llega de los transductores no se verá reflejada en la
%	representación final porque mucha información se habrá perdido. ¿De
%	qué me sirve entonces tener transductores de calidad? Pues en
%	realidad si que sirve, porque a la hora de aplicar los algoritmos
%	ssp algo sacaré de tener mejores transductores. Aunque el resultado
%	final seguirá siendo malo. La clave no está en la representación si
%	no en la aplicación de los algoritmos.
%
%	El teorema de Nyquist estipula una frecuencia de muestreo mínima
%	para la cual no se produce una pérdida de información al muestrear
%	una señal periódica y limitada en banda. La frecuencia de Nyquist
%	es igual al doble de la frecuencia de oscilación máxima de la señal
%	que se preve muestrear. Si una señal se muestrea a una tasa
%	inferior se da una pérdida de información y no es posible recuperar
%	la señal original. En los \sig{endus}, la frecuencia de la señal
%	ultrasónica determina el alcance del ensayo, siendo mayor la
%	profundidad a la que penetran las ondas acústicas cuanto mayor sea
%	su frecuencia. Desde el punto de vista de la sección de recepción,
%	es necesario un transductor de alta frecuencia y gran ancho de
%	banda para poder recibir pulsos de alta frecuencia, sin embargo,
%	este traductor genera una señal analógica de alta frecuencia.
%
%	En ensayos en los que se trabaja con pulsos de alta frecuencia se
%	requieren transductores de alta frecuencia y gran ancho de banda.
%	Un transductor.
%
%	Para poder emitir y recibir pulsos de alta frecuencia son
%	necesarios transductores de alta frecuencia y gran ancho de banda,
%	sin embargo, estos transductores necesita
%
%	En los \sig{endus}, la frecuencia de la señal ultrasónica empleada
%	es uno de los factores que limitan el alcance del ensayo, siendo
%	mayor la profundidad a la que penetran las ondas cuanto mayor sea
%	su frecuencia. Los transductores de ultrasonidos utilizados en un
%	ensayo imponen un límite físico en la frecuencia de las señales
%	ultrasónicas. Como emisor, un transductor es capaz de emitir
%	señales que oscilan hasta una cierta frecuencia. Como receptor, un
%	transductor sólo puede seguir oscilaciones de presión cuando
%	ocurren con una frecuencia inferior a un cierto límite. En este
%	sentido los transductores se comportan como filtros además de como
%	transductores.
%
%	Obtener el mejor rendimiento de la \kpci{} es clave para ..
%
%	La estructura del sistema digital de medida {\color{red}(que puede
%	verse en la figura)} revela que el subsistema para la interacción
%	con el medio físico transmite una señal analógica al subsistema de
%	adquisición, este a su vez proporciona al subsistema de control y
%	presentación una versión digitalizada de esa señal para que, por
%	último, el bloque de presentación muestre al usuario supervisor la
%	forma de la señal. Para que la representación que percibe el
%	supervisor sea fiel a la señal analógica que entrega el subsistema
%	para la interacción con el medio físico, la frecuencia de
%	oscilación máxima de la señal debe estar por debajo del límite
%	configurado por el subsistema de adquisición. O lo que es lo mismo,
%	desde la perspectiva del modelo de capas, la señal analógica que
%	proporciona el subsistema para la interacción con el medio físico
%	debe cumplir los requisitos que le impone el subsistema que se
%	encuentra en la capa inmediatamente superior, el subsistema de
%	adquisición.
%
%	En los \sig{endus}, la frecuencia de la señal ultrasónica empleada
%	es uno de los factores que limitan el alcance del ensayo, siendo
%	mayor la profundidad a la que penetran las ondas cuanto mayor sea
%	su frecuencia. Los transductores de ultrasonidos utilizados en un
%	ensayo imponen un límite físico en la frecuencia de las señales
%	ultrasónicas. Como emisor, un transductor es capaz de emitir
%	señales que oscilan hasta una cierta frecuencia. Como receptor, un
%	transductor sólo puede seguir oscilaciones de presión cuando
%	ocurren con una frecuencia inferior a un cierto límite. En este
%	sentido los transductores se comportan como filtros además de como
%	transductores. No obstante, en un sistema digital de medida como el
%	propuesto existen otros límites que afectan a la frecuencia de las
%	señales de ultrasonidos con las que se trabaja.% Este párrafo
% %	debería aparecer en el primer capítulo
%
%	Uno de los principios en los que se sostiene el modelo de capas en
%	base al que se ha configurado el sistema digital de medida es que
%	cada capa provee de servicio a la capa que se encuentra
%	inmediatamente por encima. Consecuentemente, el servicio del que
%	provee una capa debe adecuarse a las necesidades de la capa que
%	recibe el servicio. Si se observan los subsistemas de adquisición y
%	para la interacción con el medio físico (en el que se ubican los
%	transductores), el subsistema para la interacción con el medio
%	físico se encarga ---en la sección de recepción--- de convertir una
%	señal acústica en una señal eléctrica y de acondicionar esta señal
%	eléctrica para que se ajuste a los requisitos del subsistema de
%	adquisición.
%
%	Es bien sabido, a través del teorema de Nyquist, que la frecuencia
%	de muestreo con la que debe muestrearse una señal analógica para
%	que esta pueda ser recuperada a partir de la versión digital
%	obtenida en el proceso de muestreo, debe ser mayor o igual al doble
%	de la mayor de las frecuencias con las que oscila la señal
%	original, siempre que esta sea periódica y se encuentre limitada en
%	banda. No obstante, para representar una señal directamente a
%	partir de sus muestras con cierta fidelidad es necesario un volumen
%	de muestras de la señal por unidad de tiempo mayor que el que se
%	obtiene a la mínima frecuencia de muestreo estipulada por el
%	teorema de Nyquist. En la práctica, una frecuencia de muestreo unas
%	cinco veces el límite superior de frecuencias correspondiente a la
%	señal que se preve representar suele ser suficiente.
%
%	Ahora convendría decir que como el subsistema de adquisición limita
%	la frecuencia de oscilación de la señal que entrega el subsistema
%	para la interacción con el medio físico también limita de forma
%	indirecta la frecuencia de la señal de ultrasonidos y por tanto la
%	profundidad de los ensayos. Limita la frecuencia de la señal de
%	ultrasonidos porque si funciono a una frecuencia superior que la
%	frecuencia que me dice el subsistema de adquisición al muestrear
%	pierdo mucha información y se desperdicia.
%
%	Por tanto, si como ocurre en este caso, el subsistema de
%	adquisición impone un requisito a la frecuencia de oscilación de la
%	señal que entrega el subsistema
%
%	Para este caso, puede considerarse que la señal eléctrica que
%	genera el receptor es idéntica a la señal acústica que percibe
%	salvo por dos hechos: la magnitud que varía es el voltaje y no la
%	presión; y, al operar el transductor como un filtro además de como
%	un conversor, la señal eléctrica está filtrada.
%
%	Un sistema con la capacidad de representar la forma de una señal,
%	un sistema como el propuesto por este proyecto, se ve limitado en
%	su funcionamiento a tratar con señales que no superen los límites
%	impuestos por su subsistema de adquisición. En otras palabras, el
%	sistema de medida requiere, para su correcto funcionamiento, que la
%	frecuencia de oscilación de las señales con las que trabaja no
%	supere un quinto de su frecuencia de muestreo máxima. Esta
%	condición repercute en directamente
% <<<
