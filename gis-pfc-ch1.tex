\chapter{Subsistema de interacción con el medio físico}

\section{Introducción al sistema digital de medida}

Un sistema de medida es cualquier instrumento formado por más de un
elemento que permite a quien lo usa evaluar una determinada propiedad de un
objeto, medio o evento. Para una definición más rigurosa de un sistema
electrónico de medida puede optarse por la encontrada en
\cite{pallas2004sas}, y dice lo siguiente:

{\small\begin{quotation}
	Se denomina sistema a la combinación de dos o más elementos,
	subconjuntos y partes necesarias para realizar una o varias
	funciones. En los sistemas de medida, esta función es la asignación
	objetiva y empírica de un número a una propiedad o cualidad de un
	objeto o evento, de tal forma que la describa.
\end{quotation}}

Un sistema de medida digital está constituido habitualmente por los
elementos mostrados en el esquema de la \cref{fig:digmeasstm}. Pueden
agruparse estos elementos según la función que desempeñan en el sistema, si
se hace de ese modo se encuentran tres subsistemas: un subsistema para la
interacción con el medio físico, un subsistema de adquisición, y un
subsistema de control y presentación. Puede observarse entonces el sistema
como una pila de capas superpuestas, en el que cada capa provee de servicio
a la capa inmediatamente superior y abstrae las capas inferiores, en cada
una de las capas se situaría cada uno de los subsistemas propuestos. De ese
modo cada subsistema puede estudiarse por separado con independencia de los
demás, así se ha hecho en esta memoria. Consecuentemente cada subsistema da
lugar a cada uno de los tres capítulos que empezando por este describen el
sistema digital de medida implementado durante el curso del desarrollo de
este proyecto.

\begin{figure}
	\begin{center}
		\includegraphics{gis-pfc-ch1-01.mps}
	\end{center}
	\caption[Sistema digital de medida] {Distintos elementos
	funcionales que conforman el sistema digital de medida.}
	\label{fig:digmeasstm}
\end{figure}

Si se define una jerarquía en la que el nivel más bajo es aquel ocupado por
los elementos en contacto directo con el medio, y se tiene como nivel más
alto aquel en el que se encuentran los elementos que interactúan con el
supervisor, entonces los niveles ordenados de inferior a superior en la
jerarquía establecida se encuentran ocupados por los subsistemas precisados
anteriormente del siguiente modo: subsistema para la interacción con el
medio, subsistema de adquisición, y subsistema de control y presentación.
El orden en el que se han dispuesto los capítulos que tratan acerca de la
configuración del sistema digital de medida, los tres primeros capítulos,
coincide con el orden que guardan los distintos subsistemas en esta
jerarquía.



Para que un sistema pueda considerarse un sistema de medida debe haber una
interacción con el objeto o evento de estudio. El propósito de este sistema
es medir las propiedades de un determinado medio físico, en este caso
madera de palmera. La interacción con el medio ocurre en el primer
subsistema que se encuentra dentro de la jerarquía de capas que define el
sistema de medida. Atendiendo a su función dentro del sistema, este
subsistema recibe en esta memoria el nombre de subsistema para la
interacción con el medio físico. Es el único de los tres subsistemas que
conforman el sistema de medida que está directamente relacionado con los
ensayos no destructivos, mientras que el resto de subsistemas, el
subsistema de adquisición y el subsistema de control y presentación se
encuentran más ligados al tratamiento digital de señales. En una inspección
ultrasónica, el subsistema para la interacción con el medio físico, este
subsistema, es imprescindible y sin él no puede realizarse el ensayo; sin
embargo, los otros dos subsistemas pueden sustituirse por otros elementos
que realicen la misma función, por un osciloscopio digital, por ejemplo. El
subsistema para la interacción con el medio físico está compuesto por los
elementos presentes en la figura \cref{fig:submedium}. Los transductores
son los dispositivos que realmente interactúan con el medio, a través de la
emisión de una señal acústica que se recibe en el otro extremo. Por su
parte los acondicionadores adecuan las señales de tal modo que, pueda haber
una comunicación entre los elementos presentes en este subsistema y los
pertenecientes a subsistemas por encima de éste.

\begin{figure}
	\begin{center}
		\includegraphics{gis-pfc-ch1-01.mps}
	\end{center}
	\caption[Subsistema para la interacción del medio físico]{Esquema
	que representa los distintos elementos o bloques presentes en el
	subsistema para la interacción con el medio físico.}
	\label{fig:submedium}
\end{figure}

Este capítulo se divide pues en tres apartados, en el primero se hace una
descripción técnica de los transductores utilizados en el proyecto, y se
hace una pequeña observación de su utilidad en relación con el tipo de
ensayo programado para el proyecto. En los dos apartados posteriores se
explica el proceso de diseño de los circuitos acondicionadores de la
sección de transmisión y de la sección de recepción respectivamente, se
exponen los requisitos que deben satisfacer y se explica brevemente el
funcionamiento de los circuitos propuestos para que el lector pueda
comprobar que satisfacen dichos requisitos. En el \cref{chap:endus} se hace
un desarrollo de la teoría de los ensayos no destructivos que puede ayudar
a comprender mejor algunos de los conceptos propuestos a lo largo de este
capítulo.


\section{Transductores de ultrasonidos}\label{sec:transducers}

Los transductores de ultrasonidos son transductores piezoeléctricos y, por
tanto, entran en la categoría de transductores generadores. La principal
característica de un sensor generador es que al ser expuesto a la magnitud
no eléctrica de interés genera una diferencia de potencial eléctrico
proporcional sin necesidad de una alimentación eléctrica. Este efecto es
reversible y puede utilizarse para provocar una respuesta no eléctrica a
una variación en una corriente eléctrica por medio de un actuador. Los
actuadores y los sensores generadores suelen presentar características
similares, sobretodo debido a este motivo.  En determinados materiales
aparece una polarización eléctrica si se somete el material a un esfuerzo
que provoca deformación, este fenómeno se conoce como efecto piezoeléctrico
y es un efecto reversible. Si se aplica una diferencia de potencial
eléctrico a dos de las caras de un material piezoeléctrico, en éste aparece
una deformación.  Un actuador de ultrasonidos vibra cuando se excita con
una onda eléctrica y esta vibración se transmite al medio en forma de onda
de presión. La señal ultrasónica se propaga por el medio y finalmente
incide en el sensor, ejerciendo una presión sobre su superficie que el
transductor traduce en una señal eléctrica.

La principal diferencia entre los transductores de ultrasonidos y los
transductores acústicos convencionales es que los transductores de
ultrasonidos trabajan a frecuencias superiores a los 20 kHz donde la onda
acústica se hace inaudible para el oído humano, de esta manera es posible
transmitir la energía acústica en pulsos estrechos. Los transductores de
ultrasonidos pueden ser de dos tipos dependiendo de como se construyan,
abiertos o cerrados. Los transductores de tipo abierto están formados por
una pantalla protectora y la pieza de material piezoeléctrico montada sobre
un cono metálico, suelen ser más eficientes y presentan un mayor ancho de
banda, no obstante, están menos preparados para trabajar en condiciones
duras y su frecuencia de resonancia se determina con menor precisión. En
transductores de diseño cerrado la pieza resonante está en contacto directo
con la carcasa protectora, y ésta se ajusta durante el proceso de
fabricación para que el conjunto vibre a la frecuencia de resonancia
elegida, estos transductores son menos eficientes pero más resistentes y
más precisos. Los transductores empleados en este \sig{pfc} son de tipo
cerrado. En la \vref{fig:transducers} se representan varias planos acotados
de los transductores de forma que puedan apreciarse sus dimensiones. En el
encapsulado del actuador existe una marca en forma de punto blanco que lo
identifica y que sirve para distinguirlo del sensor, es la única diferencia
entre ambos. El resto de características relevantes vienen recogidas en el
\cref{tab:transducers}.

% Las principales características de los transductores
% utilizados en el proyecto vienen recogidas en el \cref{tab:transducers}.

% Y su aspecto viene reflejado en la representación que realiza la figura
% 1.3
% Más que: Y su aspecto..., cuando termino de explicar que diferencias
% existen entre un transductor de tipo abierto y uno de tipo cerrado puedo
% decir algo como: los transductores utilizados en este PFC son de tipo
% cerrado, en la figura tal puede observarse su aspecto, se proporcionan
% varias vistas y se han anotado cotas para las principales dimensiones. El
% resto de características (de mayor relevancia, que presentan los
% transductores) vienen recogidas en el cuadro.

\begin{figure}
	\begin{center}
		\includegraphics{gis-pfc-ch1-05.mps}
	\end{center}
	\caption[Dimensiones de los transductores de
	ultrasonidos]{Dimensiones y forma de los transductores de
	ultrasonidos.}
	\label{fig:transducers}
\end{figure}

\begin{sidewaystable}
	\centering
	\begin{threeparttable}
	\begin{tabular}{l c c}
		\toprule
		& \multicolumn{2}{c}{Transductor} \\
		\cmidrule(l){2-3}
		Propiedad & Transmisor & Receptor \\
		\midrule
		Nivel de presión acústica transmitida & 110 dB & --- \\
		Sensibilidad del receptor & --- & -70 dB \\
		Frecuencia fundamental de resonancia
		& $40 \text{kHz} \pm 1 \text{kHz}$
		& $40 \text{kHz} \pm 1 \text{kHz}$ \\
		Ancho de banda a -6 dB & $2.5$ kHz & $3.0$ kHz \\
		Ancho del lóbulo principal & 60º & --- \\
		Máxima potencia disipable & 200 mWef & --- \\
		Impedancia & $700\ \Omega$ & $30\ \text{k}\Omega$ \\
		Capacidad & $2 \text{nF} \pm 20\%$
		& $2 \text{nF} \pm 20\%$ \\
		Tiempo de subida & $700\ \mu\text{s}$ & --- \\
		Rango de temperatura de funcionamiento %
		& (-20, 60) [ºC] & (-20, 60) [ºC] \\
		\bottomrule
	\end{tabular}
	\begin{TableNotes}
		\tnotetext{*}{Todas las magnitudes de la tabla se dan con
		respecto a la frecuencia natural de resonancia de los
		transductores.}
	\end{TableNotes}
	\end{threeparttable}
	\caption[Características de los transductores]{Características de
	los transductores empleados en el sistema de medida.}
	\label{tab:transducers}
\end{sidewaystable}

El actuador de ultrasonidos puede funcionar de dos formas diferentes: si es
alimentado con una sinusoide emite una onda acústica continua al medio; por
el contrario, si se alimenta con un pulso rectangular (o de continua),
emite un tono modulado por un pulso gaussiano de la misma duración. En
cualquier caso la potencia disipada por el actuador no puede exceder el
límite de 20 mW eficaces especificado por el fabricante. Dependiendo del
tipo de alimentación es posible adoptar una u otra estrategia para acotar
la potencia suministrada al sensor. Si se alimenta el sensor con una onda
sinusoidal puede, o bien limitarse la tensión de alimentación, o bien
utilizarse una sinusoide pulsada, empleando una configuración de ciclo de
trabajo que garantice que no se supera el límite de potencia soportado. En
caso de utilizar un pulso rectangular, o un tren de pulsos rectangulares,
las opciones son similares, reducir el nivel de tensión o configurar el
ciclo de trabajo de la señal. El actuador de ultrasonidos carga el circuito
de alimentación con una impedancia que presenta un carácter variable con
respecto a la frecuencia de la onda eléctrica y que debe tenerse en cuenta
para calcular la transferencia de potencia. La impedancia equivalente
presenta un comportamiento capacitivo a frecuencias por debajo de la
frecuencia fundamental de oscilación del actuador, carácter resistivo a
frecuencias en torno a la frecuencia de resonancia y carácter inductivo a
frecuencias por encima de esta frecuencia. Por tanto, la máxima
transferencia de potencia ocurre en frecuencias cercanas a la frecuencia
natural de oscilación del actuador, donde éste se comporta como una
impedancia resistiva de un valor típico que ronda los 700 $\Omega$. La
potencia suministrada al actuador se reparte entre la frecuencia
fundamental de resonancia y los distintos armónicos a frecuencias
superiores. El actuador siempre vibra a su frecuencia fundamental de
resonancia o a frecuencias múltiplo de ésta, independientemente de la
frecuencia de la onda eléctrica utilizada para el suministro de potencia.
Si bien esto es cierto, para evitar excitar armónicos secundarios distintos
de la frecuencia fundamental de resonancia es aconsejable que la frecuencia
de la onda eléctrica sea lo más próxima a la frecuencia de resonancia del
actuador, además esto garantiza la máxima transferencia de potencia.

En ensayos no destructivos mediante ultrasonidos es imprescindible utilizar
pulsos acústicos como el representado en la \cref{fig:pulse}, para poder
evaluar las propiedades del medio de propagación empleando una de las dos
técnicas expuestas en el \cref{chap:endus} en la \cref{sec:technics}. Como
se explica en dicho capítulo, el actuador continua vibrando a su frecuencia
fundamental de resonancia incluso después de que la onda de alimentación
pase a un estado bajo, durante un tiempo determinado durante el proceso de
fabricación del dispositivo ---esto es algo que no se aprecia en la
figura---, dando lugar a lo que en teoría de \sig{endus} se conoce como
<<zona ciega>> o <<zona muerta>>, este fenómeno se explica con mayor
detalle en la \cref{sec:field}.

\begin{figure}
	\begin{center}
		\includegraphics{gis-pfc-ch1-04.mps}
	\end{center}
	\caption[Pulso acústico generado por el actuador de
	ultrasonidos]{Pulso acústico que se obtiene del actuador de
	ultrasonidos cuando se alimenta con un pulso rectangular.}
	\label{fig:pulse}
\end{figure}

El parámetro más importante asociado al actuador de ultrasonidos es el
nivel de presión acústica transmitida (\emph{Transmitting Sound Pressure
Level}, o \psig{spl}), representa la presión que la onda acústica ejerce
sobre la presión estática del aire a una distancia determinada del actuador
cuando éste se alimenta con una determinada tensión eficaz. Proporciona una
medida de la eficiencia del sensor y puede utilizarse conjuntamente con la
sensibilidad del sensor para calcular de forma teórica cual es el alcance
máximo de un experimento de ultrasonidos que emplea estos transductores. El
\sig{spl} es una magnitud que varía con la frecuencia de la onda eléctrica,
y alcanza su máximo en la frecuencia natural de resonancia del actuador. El
fabricante suele proporcionar, o bien el valor típico del \sig{spl} a la
frecuencia de resonancia del actuador, o bien un perfil de la magnitud
frente a la frecuencia de la onda eléctrica. Con independencia del tipo de
valor proporcionado, éste suele darse en escala logarítmica utilizando una
para ello una referencia de presión que también se especifica. Para
calcular el \sig{spl} que se da en condiciones distintas de aquellas en que
se miden las especificaciones, debe calcularse la variación en escala
lineal y después trasladar este resultado a escala logarítmica, antes de
poder operar con el \sig{spl}. El sensor de ultrasonidos presenta unas
propiedades muy similares a las propiedades del actuador, si bien se
caracteriza por presentar una impedancia característica bastante mayor. En
el sensor la conversión se da en sentido contrario, las ondas acústicas de
presión que inciden en la cara externa del sensor se traducen en
variaciones de la tensión eléctrica que existe en bornes del dispositivo.
Si la propiedad más característica del actuador de ultrasonidos es el
\sig{spl}, la propiedad más característica del sensor es la sensibilidad.
La sensibilidad suele proporcionarse también en escala logarítmica, igual
que ocurre con el \sig{spl}, suele darse un valor típico o una
característica de la propiedad frente a la frecuencia que, en ocasiones se
superpone con la gráfica del \sig{spl} si los transductores pertenecen a
una misma serie. La magnitud lineal es la relación entre la presión
detectada y la diferencia de potencial eléctrico que se crea en bornes del
sensor y, en este caso, representa una medida de la eficiencia del sensor.

La madera de palmera es un medio peculiar y la atenuación que introduce en
la señal acústica es mucho mayor que la que introduce un medio como el
aire. Trabajar con transductores cuya frecuencia de resonancia se encuentra
localizada en las frecuencias bajas de la región ultrasónica presenta una
serie de ventajas y desventajas. Desde el punto de vista del sistema de
medida digital, la frecuencia de la señal eléctrica que ataca al subsistema
de adquisición no puede superar los 50 kHz, de lo contrario, con una
frecuencia de muestreo de 10 \kms{}, la señal digitalizada se vería
afectada por el aliasing. Por otro lado, atendiendo a las propiedades de
propagación de las ondas acústicas, la dispersión afecta en menor medida a
las bajas frecuencias (vid. \cref{chap:endus}), por consiguiente resulta
beneficioso trabajar en el límite inferior de las frecuencias ultrasónicas.
No obstante, las ondas acústicas de baja frecuencia tienen una tendencia
mayor a propagarse por la superficie de un medio, lo cual puede provocar
ecos indeseados que se suman a la onda acústica recibida y que dificultan
la realización de los ensayos. El verdadero problema de emplear
transductores de gama baja en un endus radica en que emiten una energía
ultrasónica muy débil, insuficiente si lo que se pretende es atravesar el
tronco de una palmera. Para poder utilizar la técnica de transmisión es
imprescindible que la onda acústica atraviese el medio, de lo contrario
únicamente puede recurrirse a la técnica de pulso"=eco. Sin embargo, es
difícil conseguir buenos resultados utilizando la técnica de pulso"=eco con
transductores de gama baja. La técnica de pulso"=eco está pensada para ser
aplicada con un único transductor que debe funcionar secuencialmente como
emisor y como receptor, y utilizar dos transductores dificulta la obtención
de buenos resultados. La teoría sostiene pues que es imposible obtener
resultados útiles en un ensayo no destructivo con ultrasonidos en el que se
emplean transductores de gama baja. Posteriormente, en el
\cref{chap:part1conclusions} esta hipótesis se confirma, por lo que ha
resultado necesario recurrir a otro tipo de transductores para poder
efectuar los ensayos previstos, este tema vuelve a tratarse más adelante
durante la segunda parte de la memoria.


\section{Acondicionador de la sección de emisión}

Los circuitos acondicionadores a menudo se relacionan únicamente con
sensores y no con actuadores. No obstante, en el marco que define el
sistema de medida digital no puede entenderse el circuito que precede al
actuador de un modo distinto a como se entiende conceptualmente un circuito
acondicionador, aunque el flujo de información circule en sentido
contrario. Conceptualmente, un circuito que acondiciona un sensor traslada
las variaciones que se producen en cierta propiedad del sensor a un
instrumento estándar por medio de una señal eléctrica. El circuito que se
inserta entre la fuente de alimentación y el actuador genera, a partir de
la alimentación que suministra la fuente, una señal adecuada para excitar
el actuador. El tipo de alimentación que precisa el actuador para emitir
pulsos acústicos como los que requiere el experimento de ultrasonidos es
muy concreta, y en cierto sentido transporta una determinada información,
por lo que no puede obtenerse directamente a partir de un instrumento
estándar como es en este caso la fuente de alimentación. Este circuito
proporciona, además, adaptación de impedancias entre la fuente de
alimentación y el actuador. Estas dos son, sin duda, competencias
atribuidas a los circuitos acondicionadores, teniendo esto en consideración
y atendiendo a la definición de circuito acondicionador que se hace en el
\cref{sec:rxco}, puede decirse que, efectivamente, el circuito que acompaña
al actuador en el sistema de medida responde a la definición formal de
circuito acondicionador y puede considerarse como tal.

Aclarado ésto, el primer paso en el diseño de un circuito acondicionador es
determinar los requisitos que dicho circuito debe satisfacer. En este caso,
el principal propósito del circuito acondicionador es proporcionar un
suministro eléctrico que reúna unas condiciones determinadas. Para la
realización de un \sig{endus} es necesario que el actuador de ultrasonidos
emita un pulso acústico, no una onda acústica continua. De lo contrario no
se pueden percibir parámetros en la señal eléctrica necesarios para
determinar algunas de las propiedades del medio de propagación. El actuador
de ultrasonidos utilizado en el proyecto puede emitir este tipo de pulsos
acústicos, para emitir un pulso acústico debe excitarse con un pulso
eléctrico rectangular. Por tanto, uno de los requisitos que el circuito
acondicionador debe satisfacer es que a su salida debe generar un tren de
pulsos eléctricos rectangulares. Con cada pulso rectangular el actuador
emite un pulso acústico, como requisito opcional puede considerarse la
incorporación de un interruptor de inicio con el que controlar el proceso.
El otro requisito fundamental que debe cumplir el circuito es un requisito
de potencia. El pulso acústico que emite el actuador debe ser capaz de
atravesar el tronco de una palmera, para conseguirlo una buena estrategia
consiste en emitir la máxima energía ultrasónica posible. Esto, a su vez,
puede conseguirse si se suministra al actuador la máxima potencia que puede
soportar. El límite de potencia que el actuador puede disipar es de 200 mW
eficaces, considerando máxima transferencia de potencia, la tensión eficaz
de la onda eléctrica con la que se alimenta el actuador debe encontrarse en
torno a los 10 V eficaces. Sin embargo, la hipótesis de que existe máxima
transferencia de potencia cuando se alimenta el sensor con un tren de
pulsos rectangulares no es correcta. De hecho, es difícil determinar cual
es el factor de transferencia de potencia en estas circunstancias. Para
favorecer el máximo tránsito de corriente a través de la carga una solución
sencilla consiste en diseñar el circuito acondicionador para que su
impedancia de salida sea mínima.

El diseño propuesto se muestra en la \cref{fig:txconditioner}, está basado
en dos temporizadores 555 ---el \sig{ne556} de Texas Instruments es un
circuito integrado que contiene dos 555 dentro de un mismo encapsulado--- y
lo completa un amplificador operacional. El 555 es un temporizador
integrado de gran precisión capaz de proporcionar pulsos rectangulares de
12 V de amplitud con tiempos de subida y de bajada contenidos entre los 300
y los 100 ns. Es un integrado muy conocido, es muy versátil, y es empleado
en gran variedad de aplicaciones. Habitualmente se utiliza en sus
configuraciones de multivibrador astable y multivibrador monoestable. En su
configuración astable el 555 genera un tren de pulsos rectangulares
periódico en el que el ciclo de trabajo puede ajustarse a voluntad para
adoptar un valor entre el 50\% y el 100\%. Por el contrario, en su
configuración monoestable el circuito genera un pulso de duración muy
precisa bajo un estímulo externo. La configuración monoestable responde a
un estímulo por estado bajo, por tanto, es fácil observar la relación que
existe entre las dos configuraciones. Utilizando dos 555 en cascada
(alimentando uno con la salida del anterior), el primero en su
configuración astable y el segundo configurado como monoestable puede
obtenerse un tren de pulsos rectangulares en el que el ciclo de trabajo no
se encuentra limitado a ningún rango de valores. En una configuración en
cascada como la propuesta, el ciclo de trabajo está determinado por la
relación entre el periodo del tren de pulsos, que únicamente depende de la
configuración del astable, y la duración de los pulsos, que a su vez
depende de la configuración del monoestable; de ahí que pueda evitarse la
restricción que se aplica al ciclo de trabajo cuando sólo se utiliza un
astable.

\begin{sidewaysfigure}
	\begin{center}
		\includegraphics{gis-pfc-ch1-02.mps}
	\end{center}
	\caption[Circuito acondicionador de la sección de emisión]{Circuito
	propuesto para acondicionar el actuador de ultrasonidos.}
	\label{fig:txconditioner}
\end{sidewaysfigure}

Ambas configuraciones, astable y monoestable, se caracterizan por la
presencia de una red temporal en el circuito. Otra propiedad característica
de estos circuitos es que el terminal de control debe encontrarse acoplado
a tierra por medio de un condensador. La red temporal es un circuito
compuesto por resistencias y condensadores que se encuentra conectado a los
terminales de disparo, umbral y descarga del 555 y que dependiendo de la
distribución, número y valor de sus componentes provoca un determinado
comportamiento en el 555. En un astable la red temporal está compuesta por
un condensador conectado por un lado a los terminales de umbral y disparo y
por otro a masa, y dos resistencias: una situada entre la tensión de
alimentación y el terminal de descarga del 555; y otra que está conectada
al primer terminal del condensador y en su otro extremo al terminal de
descarga. El periodo y el ciclo de trabajo del tren de pulsos que puede
observarse a la salida de un astable están determinados por las siguientes
ecuaciones.

\begin{align}
	T_H &= ln(2)(R_1 + R_2)C_1 \simeq 0,693(R_1 + R_2)C_1 \\
	T_L &= ln(2)(R_2)C_1 \simeq 0,693\cdot C_1R_2 \\
	T &= T_H + T_L = ln(2)(R_1 + 2R_2)C_1 = \frac{1}{f} \\
	D &= \frac{T_H}{T} = \frac{R_1 + R_2}{R_1 + 2R_2}
	\label{eq:astable}
\end{align}

Por su parte, la red temporal de un circuito monoestable está compuesta por
una resistencia ubicada entre la alimentación y los terminales de descarga
y umbral, y un condensador conectado al segundo terminal de la resistencia
y a masa. En este caso la red temporal determina la duración del pulso que
el monoestable genera cuando recibe un estímulo, la ecuación es la
siguiente.

\begin{equation}
	T_H = ln(3)C_2R_3 \simeq 1,1\cdot C_2R_3
	\label{eq:monostable}
\end{equation}

El terminal de disparo de un monoestable ha de ser conectado a la señal de
estímulo. La señal de estímulo, que en la configuración en cascada de dos
555 es la señal a la salida del astable, debe caracterizarse por permanecer
siempre en un estado estable en el que la tensión debe ser próxima a la
alimentación, y en el momento en el que deba generarse el pulso cambiar a
un estado bajo que no debe durar más de lo que dura el pulso de salida. Por
tanto, resulta comprensible que el ciclo de trabajo de un tren de pulsos
generado por un astable con el propósito de servir de señal de estímulo de
un 555 monoestable debe rondar valores elevados y, por tanto, $R_2 \lll
R_1$.

Como se mencionaba anteriormente el circuito acondicionador lo completa un
amplificador operacional configurado con realimentación negativa. El
propósito de este circuito es amplificar la señal de salida del 556 al
tiempo que limita el valor de la impedancia de salida del conjunto. La
señal de salida resultante, al emplear los componentes listados en el
cuadro (??) puede observarse en la \cref{fig:txacvo}. De acuerdo con lo
observado en la \cref{fig:pulse} en el anterior apartado, cuando uno de los
pulsos rectangulares de los que se compone el tren llega al actuador se
genera un pulso acústico. Por razones de comodidad puede instalarse un
interruptor en el terminal de \func{reset} del 555 monoestable para poder
detener o iniciar el proceso, aunque desde el punto de vista del
experimento basta con seleccionar una de los pulsos eléctricos generados.

\begin{figure}
	\begin{center}
		\includegraphics{gis-pfc-ch1-03.mps}
	\end{center}
	\caption[Señal a la salida del amplificador \textsc{ne5534},
	$u(t)$]{Señal generada por el acondicionador de transmisión medida
	a la salida del amplificador \textsc{ne5534}, $u(t)$.}
	\label{fig:txacvo}
\end{figure}


\section{Acondicionador de la sección de recepción}\label{sec:rxco}
