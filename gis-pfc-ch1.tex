\chapter{Subsistema de interacción con el medio físico}

Para que un sistema pueda considerarse un sistema de medida debe haber una
interacción con el objeto o evento de estudio. El propósito de este sistema
es medir las propiedades de un determinado medio físico, en este caso
madera de palmera. La interacción con el medio ocurre en el primer
subsistema que se encuentra dentro de la jerarquía de capas que define el
sistema de medida. Atendiendo a su función dentro del sistema, este
subsistema recibe en esta memoria el nombre de subsistema para la
interacción con el medio físico. Es el único de los tres subsistemas que
conforman el sistema de medida que está directamente relacionado con los
ensayos no destructivos, mientras que el resto de subsistemas ---el
subsistema de adquisición y el subsistema de control y presentación--- se
encuentran más ligados al tratamiento digital de señales. En otras
palabras, para la realización de un ensayo el subsistema para la
interacción con el medio físico es imprescindible, el resto no, y pueden
ser sustituidos por otros elementos que suplan su función en el sistema.

El subsistema para la interacción con el medio físico está compuesto por
los elementos presentes en la figura \cref{fig:submedium}. Como puede verse
en el esquema, los transductores son dispositivos que interactúan con el
medio (por medio de la onda acústica). El acondicionador conectado al
sensor adecua la señal eléctrica que el dispositivo genera de modo que la
información con transporta pueda transmitirse correctamente a etapas
posteriores del sistema. El acondicionador que acompaña al actuador genera
una señal de características determinadas con la que éste es alimentado. El
objeto de alimentar el transductor con una señal eléctrica de propiedades
tan concretas es el de obtener del actuador una respuesta también concreta.

\begin{figure}
	\begin{center}
		\includegraphics{gis-pfc-ch1-01.mps}
	\end{center}
	\caption[Subsistema para la interacción del medio físico]{Esquema
	que representa los distintos elementos o bloques presentes en el
	subsistema para la interacción con el medio físico.}
	\label{fig:submedium}
\end{figure}

Este capítulo se divide pues en tres apartados, en el primero se describen
los transductores empleados durante el proyecto, y se hace una pequeña
observación de su utilidad en relación con el tipo de ensayo
programado. Los dos apartados posteriores describen por su parte los
circuitos acondicionadores presentes en el subsistema.

\section{Transductores de ultrasonidos}\label{sec:transducers}

Los transductores de ultrasonidos son transductores piezoeléctricos y, por
tanto, entran en la categoría de transductores generadores. La principal
característica de un sensor generador es que al ser expuesto a la magnitud
no eléctrica de interés genera una diferencia de potencial eléctrico
proporcional, sin necesidad de una alimentación eléctrica. Este efecto es
reversible y puede utilizarse para provocar una respuesta no eléctrica a
una variación en una corriente eléctrica por medio de un actuador. Los
actuadores y los sensores generadores suelen presentar características
similares por este motivo.

En determinados materiales aparece una polarización eléctrica si se somete
el material a un esfuerzo que provoca deformación, este fenómeno se conoce
como efecto piezoeléctrico y es un efecto reversible. Si se aplica una
diferencia de potencial eléctrico a dos de las caras de un material
piezoeléctrico en éste aparece una deformación. Un actuador de ultrasonidos
vibra cuando se excita con una onda eléctrica y esta vibración se transmite
al medio en forma de onda de presión. La señal ultrasónica se propaga por
el medio y finalmente incide en el sensor, ejerciendo una presión sobre su
superficie que el transductor traduce en una señal eléctrica.

La principal diferencia entre los transductores de ultrasonidos y los
transductores acústicos convencionales es que los transductores de
ultrasonidos trabajan a frecuencias superiores a los 20 kHz donde la onda
acústica se hace inaudible para el oído humano, de esta manera es posible
transmitir la energía acústica en pulsos estrechos. Los transductores de
ultrasonidos pueden ser de dos tipos dependiendo de como se construyan,
abiertos o cerrados. Los transductores de tipo abierto están formados por
una pantalla protectora y la pieza de material piezoeléctrico montada sobre
un cono metálico, suelen ser más eficientes y presentan un mayor ancho de
banda, no obstante, están menos preparados para trabajar en condiciones
adversas y presentan una frecuencia de resonancia menos precisa. En
transductores de diseño cerrado la pieza resonante está en contacto directo
con la carcasa protectora, y ésta se ajusta durante el proceso de
fabricación para que el conjunto vibre a la frecuencia de resonancia
elegida, estos transductores son menos eficientes pero más resistentes y
más precisos. Los transductores empleados en este \sig{pfc} son de tipo
cerrado. La \vref{fig:transducers} muestra una representación del
encapsulado que reviste los transductores utilizados en el proyecto, en la
que pueden apreciarse las dimensiones del dispositivo. En el encapsulado
del actuador existe una marca en forma de punto blanco que lo identifica y
que sirve para distinguirlo del sensor, es la única diferencia aparente
entre ambos. Las características técnicas de los transductores vienen
recogidas en el \cref{tab:transducers}.

\begin{figure}
	\begin{center}
		\includegraphics{gis-pfc-ch1-05.mps}
	\end{center}
	\caption[Dimensiones de los transductores de
	ultrasonidos]{Dimensiones y forma de los transductores de
	ultrasonidos.}
	\label{fig:transducers}
\end{figure}

\begin{sidewaystable}
	\centering
	\begin{threeparttable}
	\begin{tabular}{l c c}
		\toprule
		& \multicolumn{2}{c}{Valor especificado\tnote{*}} \\
		\cmidrule(l){2-3}
		\multicolumn{1}{c}{Propiedad} & Transmisor & Receptor \\
		\midrule
		Nivel de presión acústica transmitida & 110 dB & --- \\
		Sensibilidad & --- & -70 dB \\
		Frecuencia fundamental de resonancia
		& $40 \text{kHz} \pm 1 \text{kHz}$
		& $40 \text{kHz} \pm 1 \text{kHz}$ \\
		Ancho de banda a -6 dB & $2.5$ kHz & $3.0$ kHz \\
		Ancho del lóbulo principal & 60º & --- \\
		Máxima potencia disipable & 200 mW (rms) & --- \\
		Impedancia & $700\ \Omega$ & $30\ \text{k}\Omega$ \\
		Capacidad & $2 \text{nF} \pm 20\%$
		& $2 \text{nF} \pm 20\%$ \\
		Tiempo de subida & $700\ \mu\text{s}$ & --- \\
		Rango de temperatura de funcionamiento %
		& (-20, 60) [ºC] & (-20, 60) [ºC] \\
		\bottomrule
	\end{tabular}
	\begin{TableNotes}
		\tnotetext{*}{Todas las magnitudes de la tabla se dan con
		respecto a la frecuencia natural de resonancia de los
		transductores.}
	\end{TableNotes}
	\end{threeparttable}
	\caption[Características de los transductores]{Características de
	los transductores empleados en el sistema de medida.}
	\label{tab:transducers}
\end{sidewaystable}


\subsection{El actuador de ultrasonidos}

El actuador de ultrasonidos puede funcionar de dos formas diferentes: si es
alimentado con una sinusoide emite una onda acústica continua al medio; por
el contrario, si se alimenta con un pulso rectangular (o de continua),
emite un tono modulado por un pulso gaussiano de la misma duración. En
cualquier caso la potencia disipada por el actuador no puede exceder el
límite de 20 mW eficaces especificado por el fabricante. Dependiendo del
tipo de alimentación es posible adoptar una u otra estrategia para acotar
la potencia suministrada al sensor. Si se alimenta el sensor con una onda
sinusoidal puede, o bien limitarse la amplitud de la onda, o bien
utilizarse una sinusoide pulsada, empleando una configuración de ciclo de
trabajo que garantice que no se supera el límite de potencia soportado. En
caso de utilizar un pulso rectangular, o un tren de pulsos rectangulares,
las opciones son similares, reducir el nivel de tensión de salida en el
estado alto o configurar el ciclo de trabajo de la señal apropiadamente.

El actuador de ultrasonidos carga el circuito de alimentación con una
impedancia que presenta carácter variable, la componente reactiva de la
impedancia varía con la frecuencia de la onda empleada para excitar el
transductor. La potencia transferida al actuador depende del valor que
adopta esta impedancia. A frecuencias por debajo de la frecuencia natural
de resonancia del actuador muestra un comportamiento capacitivo, en torno a
la frecuencia de resonancia la reactancia es prácticamente nula, y a
frecuencias superiores adopta valores inductivos. Por tanto, resulta más
fácil encontrar una configuración en la que se obtenga máxima transferencia
de potencia a frecuencias en torno a la frecuencia natural de resonancia
del actuador. Por otro lado, la potencia suministrada al actuador se
reparte entre la frecuencia fundamental de resonancia y los distintos
armónicos a frecuencias superiores. El actuador siempre vibra a su
frecuencia fundamental de resonancia o a frecuencias múltiplo de ésta,
independientemente de la frecuencia de la onda eléctrica utilizada para el
suministro de potencia. Si bien esto es cierto, para evitar excitar
armónicos secundarios distintos de la frecuencia fundamental de resonancia
es aconsejable que la frecuencia de la onda eléctrica sea lo más próxima a
la frecuencia de resonancia del actuador, además esto garantiza la máxima
transferencia de potencia.

En ensayos no destructivos mediante ultrasonidos es imprescindible trabajar
con pulsos acústicos como el representado en la \cref{fig:pulse}, para
poder evaluar las propiedades del medio de propagación empleando una de las
dos técnicas expuestas en el \cref{chap:endus} en la \cref{sec:technics}.
Como se explica en dicho capítulo, el actuador continua vibrando a su
frecuencia fundamental de resonancia incluso después de que la onda de
alimentación pase a un estado bajo, durante un tiempo determinado durante
el proceso de fabricación del dispositivo ---esto es algo que no se aprecia
en la figura---, dando lugar a lo que en teoría de \sig{endus} se conoce
como <<zona ciega>> o <<zona muerta>>, este fenómeno se explica con mayor
detalle en la \cref{sec:field}.

\begin{figure}
	\begin{center}
		\includegraphics{gis-pfc-ch1-04.mps}
	\end{center}
	\caption[Pulso acústico generado por el actuador de
	ultrasonidos]{Pulso acústico que se obtiene del actuador de
	ultrasonidos cuando se alimenta con un pulso rectangular.}
	\label{fig:pulse}
\end{figure}

El parámetro más importante asociado al actuador de ultrasonidos es el
nivel de presión acústica transmitida (\emph{Transmitting Sound Pressure
Level}, o \psig{spl}), representa la presión que la onda acústica ejerce
sobre la presión estática del aire a una distancia determinada del actuador
cuando éste se alimenta con una señal eléctrica de una determinada
intensidad (nivel de tensión eficaz). Proporciona una medida de la
eficiencia del actuador y puede utilizarse conjuntamente con la
sensibilidad del sensor para calcular de forma teórica cual es el alcance
máximo de un experimento de ultrasonidos que emplea dichos transductores.
El \sig{spl} es una magnitud que varía con la frecuencia de la onda
eléctrica utilizada para alimentar el actuador y alcanza su máximo a la
frecuencia natural de resonancia de éste. El fabricante suele proporcionar,
o bien el valor típico del \sig{spl} a la frecuencia de resonancia, o bien
un perfil de la magnitud frente a la frecuencia de la onda eléctrica. Con
independencia de que el \sig{spl} se presente mediante un valor numérico o
mediante un gráfico suele proporcionarse en unidades en base a una escala
logarítmica que toma como referencia un valor de presión que, generalmente,
se especifica también. Para calcular el \sig{spl} que se da en condiciones
distintas de aquellas en que se miden las especificaciones debe calcularse
la variación en escala lineal y después trasladar este resultado a escala
logarítmica antes de poder operar con el \sig{spl}.


\subsection{Sensor de ultrasonidos}

El sensor de ultrasonidos presenta unas propiedades muy similares a las
propiedades del actuador, si bien se caracteriza por presentar una
impedancia característica bastante mayor. En el sensor la conversión se da
en sentido contrario, las ondas acústicas de presión que inciden en la cara
externa del sensor se traducen en variaciones de la tensión eléctrica que
existe en bornes del dispositivo. Si la propiedad más característica del
actuador de ultrasonidos es el \sig{spl}, la propiedad más característica
del sensor es la sensibilidad. La sensibilidad suele proporcionarse también
en escala logarítmica, igual que ocurre con el \sig{spl}, suele darse un
valor típico o una característica de la propiedad frente a la frecuencia
que, en ocasiones, se superpone con la gráfica del \sig{spl} si los
transductores pertenecen a una misma serie. La magnitud lineal es la
relación entre la presión detectada y la diferencia de potencial eléctrico
que se crea en bornes del sensor y, en este caso, representa una medida de
la eficiencia del sensor.


\subsection{Observaciones sobre los transductores}

La madera de palmera es un medio peculiar y la atenuación que introduce en
la señal acústica es mucho mayor que la que introduce un medio como el
aire. Trabajar con transductores cuya frecuencia de resonancia se encuentra
localizada en las frecuencias bajas de la región ultrasónica presenta una
serie de ventajas y desventajas. Desde el punto de vista del sistema de
medida digital, la frecuencia de la señal eléctrica que ataca al subsistema
de adquisición no puede superar los 50 kHz, de lo contrario, con una
frecuencia de muestreo de 10 \kms{} la señal digitalizada se vería afectada
por el aliasing. Por otro lado, atendiendo a las propiedades de propagación
de las ondas acústicas, la dispersión afecta en menor medida a las bajas
frecuencias (vid. \cref{chap:endus}), por consiguiente resulta beneficioso
trabajar en el límite inferior de las frecuencias ultrasónicas. No
obstante, las ondas acústicas de baja frecuencia tienen una tendencia mayor
a propagarse por la superficie de un medio, lo cual puede provocar ecos
indeseados que se suman a la onda acústica recibida y que dificultan la
realización de los ensayos. El verdadero problema de emplear transductores
de gama baja en un \sig{endus} radica en que emiten una energía ultrasónica
muy débil, insuficiente si lo que se pretende es atravesar el tronco de una
palmera. Para poder utilizar la técnica de transmisión es imprescindible
que la onda acústica atraviese el medio, de lo contrario únicamente puede
recurrirse a la técnica de pulso"=eco. Sin embargo, es difícil conseguir
buenos resultados utilizando la técnica de pulso"=eco con transductores de
gama baja. La técnica de pulso"=eco está pensada para ser aplicada con un
único transductor que debe funcionar secuencialmente como emisor y como
receptor, y utilizar dos transductores dificulta la obtención de buenos
resultados. La teoría sostiene pues que es imposible obtener resultados
útiles en un ensayo no destructivo con ultrasonidos en el que se emplean
transductores de gama baja. Posteriormente, en el
\cref{chap:part1conclusions} esta hipótesis se confirma, por lo que ha
resultado necesario recurrir a otro tipo de transductores para poder
efectuar los ensayos previstos, este tema vuelve a tratarse más adelante
durante la segunda parte de la memoria.


\section{Acondicionador de la sección de recepción}\label{sec:rxco}

Una transición directa de la señal del sensor, donde es generada, al
sistema de medida conlleva en la mayoría de los casos una notable pérdida
de información causada por diversos fenómenos. En un sentido amplio, un
acondicionador es un circuito que se conecta a un transductor y cuya
finalidad no es otra que la de impedir, en la medida de lo posible, que la
información que transporta la señal que procede del sensor se pierda en el
mencionado proceso de transición. En otras palabras, el propósito en
esencia de un circuito acondicionador es el de realzar las componentes de
la señal que acarrean la información, al tiempo que atenúa las señales
espurias que se mezclan con la señal de interés y dificultan la lectura de
dicha información y crea las condiciones adecuadas para que la señal sea
transferida adecuadamente al sistema. Para aproximarse al máximo a este
objetivo el circuito acondicionador transforma la señal para que se acomode
a las características de la subsiguiente etapa en el sistema. No obstante,
la labor del acondicionador no se limita exclusivamente a aplicar una
transformación sobre la señal, a continuación se enumeran las principales
acciones que desempeña un circuito de estas características en un sistema
de medida.

\begin{itemize}
	\item Acota el recorrido de la señal y lo confinar en unos márgenes
		de tensión apropiados, evitando una posible saturación de
		etapas posteriores.
	\item Adapta la impedancia de salida del sensor a la impedancia de
		entrada de la etapa posterior al circuito, de ese modo la
		transmisión es máxima.
	\item A menudo los sensores proveen una señal de amplitud
		comparable al ruido, el circuito acondicionador es
		responsable de amplificar la señal manteniendo ---o
		mejorando si es posible--- el nivel de \sig{snr}. Los
		circuitos acondicionadores están integrados ---en caso de
		que sea necesario--- por amplificadores de instrumentación,
		útiles por sus altos índices de \sig{cmrr}.
	\item El acondicionador de un sistema de medida filtra componentes
		de señal indeseadas susceptibles de causar algún tipo de
		distorsión en los resultados de una prueba.
\end{itemize}


\subsection{Diseño del circuito acondicionador}

El acondicionador es un circuito complejo que requiere de un diseño
elaborado. En definitiva la función que ejerce el circuito acondicionador
en el sistema de medida es muy concreta: a partir de la señal generada por
el sensor ofrece una señal que puede ser procesada adecuadamente por un
dispositivo estándar. El diseño del circuito depende, entonces,
exclusivamente de tres aspectos: el sensor utilizado, el dispositivo
estándar que sirve de interfaz para con el resto del sistema, y las
condiciones de funcionamiento.

Del sensor es preciso determinar el tipo de señal que genera, su amplitud y
frecuencia, en ocasiones energía o densidad espectral de potencia; y su
impedancia equivalente.

Los dispositivos estándar son aparatos comunes a la mayoría de sistemas
capaces de desempeñar una función de procesamiento, almacenamiento y/o
presentación en multitud de situaciones diferentes. Están preparados para
trabajar con señales de características dispares, y están pensados para
comunicarse con una gran diversidad de instrumentos adicionales. Sus
características están bien definidas y son ampliamente conocidas, además de
ser muy versátiles hacen gala de una elevada precisión. Por todo ello,
resulta práctico utilizarlos pues simplifican el diseño de un sistema de
medida y, por extensión, de un circuito acondicionador. Suelen, asimismo,
ser dispositivos caros, y difíciles de reemplazar ---representan el
estándar, no es fácil encontrar alternativas ni práctico buscarlas---. Por
lo general, cuando el sistema de medida deja de satisfacer su propósito por
cualquier motivo y es necesario introducir en él alguna modificación es
preferible sustituir el sensor y/o el circuito acondicionador y diseñar la
nueva etapa para que se adapte a las características del dispositivo
estándar.

Las funciones de procesamiento, almacenamiento y presentación de la señal
se reparten en el esquema propuesto en este proyecto en dos entidades
físicas distintas: la tarjeta de adquisición digital y el ordenador, que
asimismo gobierna el funcionamiento del sistema. Ambas entidades pueden ser
consideradas dispositivos estándar de acuerdo a la definición dada en el
párrafo anterior. De cara al diseño del circuito acondicionador ha de
recalcarse que éste únicamente interactúa de forma directa con la tarjeta,
por lo que es dicha tarjeta la que ha de tenerse en cuenta durante el
proceso de diseño. La tarjeta \kpci{} se encarga concretamente de la
conversión de analógico a digital de la señal, de la reordenación y
manipulación de los datos y del almacenamiento intermedio de los mismos. De
momento no es necesario conocer con mucho detalle cual es el funcionamiento
de la tarjeta, tema sobre el cual se profundiza en el
\cref{chap:acquisition}.

En cuanto a las condiciones de trabajo: temperatura ambiente, presencia de
radiación, presencia de impulsos eléctricos\dots; pueden ser muy variadas
pero se asume que se trabaja en condiciones estándar y no se profundizará
más en el tema.

Recapitulando, en los \cref{tab:sensprop,tab:cardprop} se encuentran
recogidos los parámetros relevantes en el diseño de este circuito
acondicionador y los valores que adquieren en el caso particular de este
sistema de medida (recuérdese que la configuración del subsistema, en una
división que responde a la función de los distintos elementos que lo
integran, puede verse en la \cref{fig:submedium}).

\begin{table}
	\centering
	\begin{tabular}{l r@{\ }l}
		\toprule
		Propiedad & \multicolumn{2}{c}{Valor} \\
		\midrule
		\\
		Tipo de señal generada & \multicolumn{2}{c}{$x(t) =
		A\sin(\omega t)\cdot\text{e}^{-\alpha t^2}$} \\\\
		Densidad espectral de potencia & \multicolumn{2}{c}{$X(F) =
		(A/2)\sqrt{\pi/\alpha}\cdot\text{e}^{-(\pi F)^2/\alpha}$}\\
		\\
		Frecuencia de resonancia & $20$ & $[\text{kHz}]$ \\
		Ancho de banda (aprox.) & $20$ & $[\text{kHz}]$ \\
		Resistencia & \hspace*{4.8em}$2$ & $[\text{k}\Omega]$ \\
		Capacitancia & \hspace*{4.8em}$1.9$ & $[\text{nF}]$ \\
		\bottomrule
	\end{tabular}
	\caption[Propiedades relevantes del sensor]{Propiedades del sensor
	relevantes para el diseño del circuito acondicionador.}
	\label{tab:sensprop}
\end{table}

\begin{table}
	\centering
	\begin{tabular}{l@{\hspace*{3em}}r@{\ }l}
		\toprule
		Propiedad & \multicolumn{2}{c}{Valor} \\
		\midrule
		Rango de tensiones aceptado & $(-10, 10)$ & $[\text{V}]$ \\
		Resistencia & $1$ & $[\text{G}\Omega]$ \\
		Capacitancia & $C < 200$ & $[\text{pF}]$ \\
		\bottomrule
	\end{tabular}
	\caption[Propiedades relevantes de la tarjeta]{Propiedades de la
	tarjeta de adquisición \kpci{} (vid. cáp. \ref{chap:acquisition})
	relevantes para el diseño del circuito acondicionador.}
	\label{tab:cardprop}
\end{table}

% Todo ello nos dice (de acuerdo con las necesidades del experimento de
% endus que se desea realizar)\dots

%\subsection{Circuito propuesto}


\section{Acondicionador de la sección de emisión}

Los circuitos acondicionadores a menudo se relacionan únicamente con
sensores y no con actuadores. No obstante, en el marco que define el
sistema de medida digital no puede entenderse el circuito que precede al
actuador de un modo distinto a como se entiende conceptualmente un circuito
acondicionador, aunque el flujo de información circule en sentido
contrario. Conceptualmente, un circuito que acondiciona un sensor traslada
las variaciones que se producen en cierta propiedad del sensor a un
instrumento estándar por medio de una señal eléctrica. El circuito que se
inserta entre la fuente de alimentación y el actuador genera, a partir de
la alimentación que suministra la fuente, una señal adecuada para excitar
el actuador. El tipo de alimentación que precisa el actuador para emitir
pulsos acústicos como los que requiere el experimento de ultrasonidos es
muy concreta, y en cierto sentido transporta una determinada información,
por lo que no puede obtenerse directamente a partir de un instrumento
estándar como es en este caso la fuente de alimentación. Este circuito
proporciona, además, adaptación de impedancias entre la fuente de
alimentación y el actuador. Estas dos son, sin duda, competencias
atribuidas a los circuitos acondicionadores, teniendo esto en consideración
y atendiendo a la definición de circuito acondicionador que se hace en el
\cref{sec:rxco}, puede decirse que, efectivamente, el circuito que acompaña
al actuador en el sistema de medida responde a la definición formal de
circuito acondicionador y puede considerarse como tal.


\subsection[Requisitos de diseño del acondicionador]{Requisitos de diseño
que debe satisfacer el circuito acondicionador}

Aclarado ésto, el primer paso en el diseño de un circuito acondicionador es
determinar los requisitos que dicho circuito debe satisfacer. Como se
observa del párrafo anterior, el acondicionador debe proporcionar un
suministro eléctrico que reúna unas condiciones determinadas. En
definitiva, para determinar que tipo de señal debe generar el
acondicionador es preciso estudiar el tipo de experimento realizado y las
características del actuador empleado en dicho experimento. Para la
realización de un \sig{endus} es necesario que el actuador de ultrasonidos
emita un pulso acústico, no una onda acústica continua. De lo contrario no
se pueden percibir parámetros en la señal eléctrica necesarios para
determinar algunas de las propiedades del medio de propagación. El actuador
de ultrasonidos utilizado en el proyecto puede emitir este tipo de pulsos
acústicos, para emitir un pulso acústico debe excitarse con un pulso
eléctrico rectangular. Por tanto, uno de los requisitos que el circuito
acondicionador debe satisfacer es que a su salida debe generar un tren de
pulsos eléctricos rectangulares. Con cada pulso rectangular el actuador
emite un pulso acústico, como requisito opcional puede considerarse la
incorporación de un interruptor de inicio con el que controlar el proceso.

El otro requisito fundamental que debe cumplir el circuito es un requisito
de potencia. El pulso acústico que emite el actuador debe ser capaz de
atravesar el tronco de una palmera, para conseguirlo una buena estrategia
consiste en emitir la máxima energía ultrasónica posible. Esto, a su vez,
puede conseguirse si se suministra al actuador la máxima potencia que puede
soportar. El límite de potencia que el actuador puede disipar es de 200 mW
eficaces, considerando máxima transferencia de potencia, la tensión eficaz
de la onda eléctrica con la que se alimenta el actuador debe encontrarse en
torno a los 10 V eficaces. La impedancia con la que el actuador carga el
circuito debe, sin embargo, tenerse en cuenta a la hora de calcular la
transferencia de potencia que existe entre el acondicionador y el actuador.
Cuando es alimentado por un tren de pulsos rectangulares, el conjunto
formado por el acondicionador y el actuador se comporta como un circuito
\sig{RC} que se encuentra en régimen transitorio. De forma que para
averiguar cuales son los niveles de tensión y corriente que se observan en
bornes del actuador es menester calcular la constante de tiempos del
circuito ($\tau = C\cdot R_\text{eq}$) y observar el comportamiento del
circuito en el tiempo. Tomando como hipótesis que la impedancia de salida
del acondicionador es reducida ---hipótesis que se cumple en el circuito
propuesto (\cref{fig:txconditioner}), dado que la etapa de salida de dicho
circuito la constituye un amplificador operacional cuya impedancia de
salida se aproxima a cero---, y teniendo en cuenta que la parte resistiva
de la impedancia atribuida al actuador es también pequeña
(\cref{tab:transducers}), puede deducirse que la constante de tiempos
resultante es también moderada. Es posible obtener un valor aproximado a
partir de cálculos sencillos.

\begin{equation}\label{eq:timec}
	\begin{split}
		\tau &= C_\text{L}\cdot R_\text{eq}
		\Big|_{R_\text{in}\simeq\infty} =
		C_\text{L}\left(R_\text{L} + R_0\|(R_\text{a} +
		R_\text{b})\right)\Big|_{R_0 \simeq 0} =
		C_\text{L}\cdot R_\text{L} \\
		&= 700\cdot2.2\cdot 10^{-9} = 1.54\mu s
	\end{split}
\end{equation}

Donde $R_\text{in}$ y $R_\text{0}$ representan respectivamente la
impedancia de entrada y la impedancia de salida del amplificador
operacional. Finalmente se observa que, con los valores de periodo, tensión
en el estado alto y ciclo de trabajo de la señal que proporciona el
acondicionador (\cref{fig:txacvo}), y considerando una constante de tiempos
como la obtenida en la \cref{eq:timec}, la respuesta del actuador es muy
rápida y la señal en sus bornes sigue con bastante fidelidad la señal que
puede medirse con el circuito en abierto.


\subsection[Circuito acondicionador propuesto]{Circuito acondicionador
propuesto, el 555}

El diseño propuesto se muestra en la \cref{fig:txconditioner}, está basado
en dos temporizadores 555 ---a nivel comercial se encuentran disponibles
encapsulados que en su interior contienen dos 555, generalmente se conocen
bajo el número de referencia 556--- y lo completa un amplificador
operacional. El 555 es un temporizador integrado de gran precisión capaz de
proporcionar pulsos rectangulares de 12 V de amplitud con tiempos de subida
y de bajada contenidos entre los 300 y los 100 ns. Es un integrado muy
conocido, es muy versátil, y es empleado en gran variedad de aplicaciones.
Habitualmente se utiliza en sus configuraciones de multivibrador astable y
multivibrador monoestable. En su configuración astable el 555 genera un
tren de pulsos rectangulares periódico en el que el ciclo de trabajo puede
ajustarse a voluntad para adoptar un valor entre el 50\% y el 100\%. Por el
contrario, en su configuración monoestable el circuito genera un pulso de
duración muy precisa bajo un estímulo externo. La configuración monoestable
responde a un estímulo por estado bajo, por tanto, es fácil observar la
relación que existe entre las dos configuraciones. Utilizando dos 555 en
cascada (alimentando uno con la salida del anterior), el primero en su
configuración astable y el segundo configurado como monoestable puede
obtenerse un tren de pulsos rectangulares en el que el ciclo de trabajo no
se encuentra limitado a ningún rango de valores. En una configuración en
cascada como la propuesta, el ciclo de trabajo está determinado por la
relación entre el periodo del tren de pulsos, que únicamente depende de la
configuración del astable, y la duración de los pulsos, que a su vez
depende de la configuración del monoestable; de ahí que pueda evitarse la
restricción que se aplica al ciclo de trabajo cuando sólo se utiliza un
astable.

\begin{sidewaysfigure}
	\begin{center}
		\includegraphics{gis-pfc-ch1-02.mps}
	\end{center}
	\caption[Circuito acondicionador de la sección de emisión]{Circuito
	propuesto para acondicionar el actuador de ultrasonidos. Para
	elaborar la figura se ha utilizado el \sig{ne556} y el \sig{ne5534}
	de Texas Instruments.}
	\label{fig:txconditioner}
\end{sidewaysfigure}

Ambas configuraciones, astable y monoestable, se caracterizan por la
presencia de una red temporal en el circuito. Otra propiedad característica
de estos circuitos es que el terminal de control debe encontrarse acoplado
a tierra por medio de un condensador. La red temporal es un circuito
compuesto por resistencias y condensadores que se encuentra conectado a los
terminales de disparo, umbral y descarga del 555 y que dependiendo de la
distribución, número y valor de sus componentes provoca un determinado
comportamiento en el 555. En un astable la red temporal está compuesta por
un condensador conectado por un lado a los terminales de umbral y disparo y
por otro a masa, y dos resistencias: una situada entre la tensión de
alimentación y el terminal de descarga del 555; y otra que está conectada
al primer terminal del condensador y en su otro extremo al terminal de
descarga. El periodo y el ciclo de trabajo del tren de pulsos que puede
observarse a la salida de un astable están determinados por las siguientes
ecuaciones.

\begin{align}
	T_H &= ln(2)(R_1 + R_2)C_1 \simeq 0,693(R_1 + R_2)C_1 \\
	T_L &= ln(2)(R_2)C_1 \simeq 0,693\cdot C_1R_2 \\
	T &= T_H + T_L = ln(2)(R_1 + 2R_2)C_1 = \frac{1}{f} \\
	D &= \frac{T_H}{T} = \frac{R_1 + R_2}{R_1 + 2R_2}
	\label{eq:astable}
\end{align}

Por su parte, la red temporal de un circuito monoestable está compuesta por
una resistencia ubicada entre la alimentación y los terminales de descarga
y umbral, y un condensador conectado al segundo terminal de la resistencia
y a masa. En este caso la red temporal determina la duración del pulso que
el monoestable genera cuando recibe un estímulo, la ecuación es la
siguiente.

\begin{equation}
	T_H = ln(3)C_2R_3 \simeq 1,1\cdot C_2R_3
	\label{eq:monostable}
\end{equation}

El terminal de disparo de un monoestable ha de ser conectado a la señal de
estímulo. La señal de estímulo, que en la configuración en cascada de dos
555 es la señal a la salida del astable, debe caracterizarse por permanecer
siempre en un estado estable en el que la tensión debe ser próxima a la
tensión de alimentación, y en el momento en el que deba generarse el pulso
cambiar a un estado bajo que no debe durar más de lo que dura el pulso de
salida. Por tanto, resulta comprensible que el ciclo de trabajo de un tren
de pulsos generado por un astable con el propósito de servir de señal de
estímulo de un 555 monoestable debe rondar valores elevados y, por tanto,
$R_2 \lll R_1$.

Como se mencionaba al inicio del apartado el circuito acondicionador lo
completa un amplificador operacional configurado con realimentación
negativa. El propósito de este circuito es amplificar la señal de salida
del 556 al tiempo que limita el valor de la impedancia de salida del
conjunto. La señal de salida resultante, al emplear los componentes
listados en el \cref{tab:txcondcomp} puede observarse en la
\cref{fig:txacvo}. De acuerdo con lo observado en la \cref{fig:pulse} en la
sección anterior, cuando uno de los pulsos rectangulares de los que se
compone el tren llega al actuador se genera un pulso acústico. Por razones
de comodidad puede instalarse un interruptor en el terminal de \func{reset}
del 555 monoestable para poder detener o iniciar el proceso, aunque desde
el punto de vista del experimento basta con pausar la representación de los
datos en el momento deseado. Puesto que con la configuración propuesta la
imagen ha de refrescarse cada segundo es fácil para un ser humano
determinar en que momento resulta adecuado congelar la prueba.

\begin{table}
	\centering
	\begin{threeparttable}
	% Columnas en modo matemático
	\begin{tabular}{>{\centering\(}p{.3\textwidth}<{\)} D{.}{\ }{3.4}
		>{\(}c<{\)}}
		\toprule
		\multicolumn{1}{c}{Componente} &
			\multicolumn{1}{c}{Valor nominal} &
			\multicolumn{1}{c}{Coste
			aproximado\tnote{a}\hspace{.8em}(\texteuro)} \\
		\midrule
		R_1 & 1.{\text{k}\Omega} & 0.20 \\
		R_2 & 10.{\text{M}\Omega} & 0.20 \\
		R_3 & 510.{\Omega} & 0.20 \\
		C\hspace*{.4em} & 100.\text{nF} & 0.20 \\
		C_1 & 150.\text{nF} & 0.20 \\
		C_2 & 680.\text{nF} & 0.20 \\
		\multicolumn{1}{c}{\textsc{ne/se} \(556\)} &
		\multicolumn{1}{c}{\parbox{1.5em}{---}} & 0.40 \\
		\multicolumn{1}{c}{\textsc{ne} \(5534\)} &
		\multicolumn{1}{c}{\parbox{1.5em}{---}} & 0.70 \\
		\bottomrule
	\end{tabular}
	\begin{TableNotes}
		\tnotetext{a}{Los datos correspondientes a esta columna del
		cuadro se han elaborado a partir de los precios que pueden
		consultarse en el catálogo de
		\href{http://es.farnell.com}{Farnell}.}
	\end{TableNotes}
	\end{threeparttable}
	\caption[Componentes del acondicionador en emisión]{Componentes
	seleccionados para implementar el circuito acondicionador que
	aparece en la \cref{fig:txconditioner}.}
	\label{tab:txcondcomp}
\end{table}

\begin{figure}
	\begin{center}
		\includegraphics{gis-pfc-ch1-03.mps}
	\end{center}
	\caption[Señal a la salida del amplificador \textsc{ne5534},
	$u(t)$]{Señal generada por el acondicionador de transmisión medida
	a la salida del amplificador \textsc{ne5534}, $u(t)$.}
	\label{fig:txacvo}
\end{figure}
