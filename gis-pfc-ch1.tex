\chapter{Subsistema de interacción con el medio físico}

\section{Introducción al sistema digital de medida}

Un sistema de medida es cualquier instrumento formado por más de un
elemento que permite a quien lo usa evaluar una determinada propiedad de un
objeto, medio o evento. Para una definición más rigurosa de un sistema
electrónico de medida puede optarse por la encontrada en
\cite{pallas2004sas}, y dice lo siguiente:

{\small\begin{quotation}
	Se denomina sistema a la combinación de dos o más elementos,
	subconjuntos y partes necesarias para realizar una o varias
	funciones. En los sistemas de medida, esta función es la asignación
	objetiva y empírica de un número a una propiedad o cualidad de un
	objeto o evento, de tal forma que la describa.
\end{quotation}}

Un sistema de medida digital está constituido habitualmente por los
elementos mostrados en el esquema de la \cref{fig:digmeasstm}. Pueden
agruparse estos elementos según la función que desempeñan en el sistema, si
se hace de ese modo se encuentran tres subsistemas: un subsistema para la
interacción con el medio físico, un subsistema de adquisición, y un
subsistema de control y presentación. Puede observarse entonces el sistema
como una pila de capas superpuestas, en el que cada capa provee de servicio
a la capa inmediatamente superior y abstrae las capas inferiores, en cada
una de las capas se situaría cada uno de los subsistemas propuestos. De ese
modo cada subsistema puede estudiarse por separado con independencia de los
demás, así se ha hecho en esta memoria. Consecuentemente cada subsistema da
lugar a cada uno de los tres capítulos que empezando por este describen el
sistema digital de medida implementado durante el curso del desarrollo de
este proyecto.

\begin{figure}
	\begin{center}
		\includegraphics{gis-pfc-ch1-01.mps}
	\end{center}
	\caption[Sistema digital de medida] {Distintos elementos
	funcionales que conforman el sistema digital de medida.}
	\label{fig:digmeasstm}
\end{figure}

Si se define una jerarquía en la que el nivel más bajo es aquel ocupado por
los elementos en contacto directo con el medio, y se tiene como nivel más
alto aquel en el que se encuentran los elementos que interactúan con el
supervisor, entonces los niveles ordenados de inferior a superior en la
jerarquía establecida se encuentran ocupados por los subsistemas precisados
anteriormente del siguiente modo: subsistema para la interacción con el
medio, subsistema de adquisición, y subsistema de control y presentación.
El orden en el que se han dispuesto los capítulos que tratan acerca de la
configuración del sistema digital de medida, los tres primeros capítulos,
coincide con el orden que guardan los distintos subsistemas en esta
jerarquía.


\section{Introducción}

El primero de los subsistemas estudiado en este documento es el subsistema
para la interacción con el medio físico. Es el subsistema que ocupa la capa
de menor nivel y agrupa los elementos que pueden observarse en la
\cref{fig:submedium}.

\begin{figure}
	\begin{center}
		\includegraphics{gis-pfc-ch1-02.mps}
	\end{center}
	\caption[Subsistema para la interacción con el medio físico]
	{Elementos del sistema digital de medida que también se incluyen en
	el subsistema para la interacción con el medio físico.}
	\label{fig:submedium}
\end{figure}



