\chapter{Subsistema de interacción con el medio físico}

\section{Introducción al sistema digital de medida}

Un sistema de medida es cualquier instrumento formado por más de un
elemento que permite a quien lo usa evaluar una determinada propiedad de un
objeto, medio o evento. Para una definición más rigurosa de un sistema
electrónico de medida puede optarse por la encontrada en
\cite{pallas2004sas}, y dice lo siguiente:

{\small\begin{quotation}
	Se denomina sistema a la combinación de dos o más elementos,
	subconjuntos y partes necesarias para realizar una o varias
	funciones. En los sistemas de medida, esta función es la asignación
	objetiva y empírica de un número a una propiedad o cualidad de un
	objeto o evento, de tal forma que la describa.
\end{quotation}}

Un sistema de medida digital está constituido habitualmente por los
elementos mostrados en el esquema de la \cref{fig:digmeasstm}. Pueden
agruparse estos elementos según la función que desempeñan en el sistema, si
se hace de ese modo se encuentran tres subsistemas: un subsistema para la
interacción con el medio físico, un subsistema de adquisición, y un
subsistema de control y presentación. Puede observarse entonces el sistema
como una pila de capas superpuestas, en el que cada capa provee de servicio
a la capa inmediatamente superior y abstrae las capas inferiores, en cada
una de las capas se situaría cada uno de los subsistemas propuestos. De ese
modo cada subsistema puede estudiarse por separado con independencia de los
demás, así se ha hecho en esta memoria. Consecuentemente cada subsistema da
lugar a cada uno de los tres capítulos que empezando por este describen el
sistema digital de medida implementado durante el curso del desarrollo de
este proyecto.

\begin{figure}
	\begin{center}
		\includegraphics{gis-pfc-ch1-01.mps}
	\end{center}
	\caption[Sistema digital de medida] {Distintos elementos
	funcionales que conforman el sistema digital de medida.}
	\label{fig:digmeasstm}
\end{figure}

Si se define una jerarquía en la que el nivel más bajo es aquel ocupado por
los elementos en contacto directo con el medio, y se tiene como nivel más
alto aquel en el que se encuentran los elementos que interactúan con el
supervisor, entonces los niveles ordenados de inferior a superior en la
jerarquía establecida se encuentran ocupados por los subsistemas precisados
anteriormente del siguiente modo: subsistema para la interacción con el
medio, subsistema de adquisición, y subsistema de control y presentación.
El orden en el que se han dispuesto los capítulos que tratan acerca de la
configuración del sistema digital de medida, los tres primeros capítulos,
coincide con el orden que guardan los distintos subsistemas en esta
jerarquía.



Para que un sistema pueda considerarse un sistema de medida debe haber una
interacción con el objeto o evento de estudio. El propósito de este sistema
es medir las propiedades de un determinado medio físico, en este caso
madera de palmera. La interacción con el medio ocurre en el primer
subsistema que se encuentra dentro de la jerarquía de capas que define el
sistema de medida. Atendiendo a su función dentro del sistema, este
subsistema recibe en esta memoria el nombre de subsistema para la
interacción con el medio físico. Es el único de los tres subsistemas que
conforman el sistema de medida que está directamente relacionado con los
ensayos no destructivos, mientras que el resto de subsistemas, el
subsistema de adquisición y el subsistema de control y presentación se
encuentran más ligados al tratamiento digital de señales. En una inspección
ultrasónica, el subsistema para la interacción con el medio físico, este
subsistema, es imprescindible y sin él no puede realizarse el ensayo; sin
embargo, los otros dos subsistemas pueden sustituirse por otros elementos
que realicen la misma función, por un osciloscopio digital, por ejemplo. El
subsistema para la interacción con el medio físico está compuesto por los
elementos presentes en la figura \cref{fig:submedium}. Los transductores
son los dispositivos que interactúan con el medio propiamente dicho, a
través de la emisión de una señal acústica que se recibe en el otro
extremo. Por su parte los acondicionadores adecuan las señales de tal modo
que, pueda haber una comunicación entre los elementos presentes en este
subsistema y los pertenecientes a subsistemas por encima de éste.

\begin{figure}
	\begin{center}
		\includegraphics{gis-pfc-ch1-01.mps}
	\end{center}
	\caption[Subsistema para la interacción del medio físico]{Esquema
	que representa los distintos elementos o bloques presentes en el
	subsistema para la interacción con el medio físico.}
	\label{fig:submedium}
\end{figure}

Este capítulo se divide pues en tres apartados, en el primero se hace una
descripción técnica de los transductores utilizados en el proyecto, y se
hace una pequeña observación de su utilidad en relación con el tipo de
ensayo programado para el proyecto. En los dos apartados posteriores se
explica el proceso de diseño de los circuitos acondicionadores de la
sección de transmisión y de la sección de recepción respectivamente, se
exponen los requisitos que deben satisfacer y se explica brevemente el
funcionamiento de los circuitos propuestos para que el lector pueda
comprobar que satisfacen dichos requisitos. En el \cref{chap:endus} se hace
un desarrollo de la teoría de los ensayos no destructivos que puede ayudar
a comprender mejor algunos de los conceptos propuestos a lo largo de este
capítulo.


\section{Transductores de ultrasonidos}

Los transductores de ultrasonidos son transductores piezoeléctricos y, por
tanto, entran en la categoría de transductores generadores. La principal
característica de un sensor generador es que al ser expuesto a la magnitud
no eléctrica de interés se crea en bornes del sensor una diferencia de
potencial eléctrico proporcional al valor que adopta dicha magnitud sin
necesidad de que el sensor sea alimentado de forma externa. Este efecto es
reversible y puede utilizarse para provocar una respuesta no eléctrica a
una variación en una corriente eléctrica por medio de un actuador. Los
actuadores y los sensores generadores suelen presentar características
similares, sobretodo debido a este motivo. El efecto piezoeléctrico es un
efecto por el cual en un material que presenta una estructura cristalina se
crea una diferencia de potencial eléctrico al aplicarse sobre él una
deformación mecánica. El efecto contrario también recibe el mismo nombre,
al aplicar una diferencia de potencial sobre un material este se deforma
por acción del potencial comprimiéndose o expandiéndose. Las contracciones
provocadas en un actuador de ultrasonidos al excitarlo con una señal
eléctrica se transforman en ondas de presión, concretamente en ondas
acústicas, que se propagan por el medio. De igual manera la onda acústica
actúa sobre el sensor, ejerciendo una presión sobre su superficie que el
sensor traduce en una señal eléctrica. Las principales características de
los transductores utilizados en el proyecto vienen recogidas en el
\cref{tab:transducers}.

\begin{table}
	\centering
	\begin{threeparttable}
	\begin{tabular}{l c c}
		\toprule
		& \multicolumn{2}{c}{Transductor} \\
		\cmidrule(l){2-3}
		Propiedad & Transmisor & Receptor \\
		\midrule
		Nivel de presión acústica transmitida & 110 dB & --- \\
		Sensibilidad del receptor & --- & -70 dB \\
		Frecuencia fundamental de resonancia
		& $40 \text{kHz} \pm 1 \text{kHz}$
		& $40 \text{kHz} \pm 1 \text{kHz}$ \\
		Ancho de banda a -6 dB & 2.5 kHz & 3.0 kHz \\
		Ancho del lóbulo principal & 60º & \\
		Máxima potencia disipable & 200 mWef & \\
		Impedancia & $700\ \Omega$ & $30\ \text{k}\Omega$ \\
		Capacidad & $2 \text{nF} \pm 20\%$
		& $2 \text{nF} \pm 20\%$ \\
		Tiempo de subida & $700\ \mu\text{s}$ & \\
		Rango de temperatura de funcionamiento %
		& (-20, 60) [ºC] & (-20, 60) [ºC] \\
		\bottomrule
	\end{tabular}
	\begin{TableNotes}
		\tnotetext{*}{Todas las magnitudes de la tabla se dan con
		respecto a la frecuencia natural de resonancia de los
		transductores.}
	\end{TableNotes}
	\end{threeparttable}
	\caption[Características de los transductores empleados en el
	sistema de medida]{Características de los transductores empleados
	en el sistema de medida.}
	\label{tab:transducers}
\end{table}

El actuador de ultrasonidos puede funcionar de dos formas diferentes: si es
alimentado con una sinusoide emite una onda acústica continua al medio; por
el contrario, si se alimenta con un pulso rectangular (o de continua),
emite un tono modulado por un pulso gaussiano de la misma duración. En
cualquier caso la potencia disipada por el actuador no puede exceder el
límite de 20 mW eficaces especificado por el fabricante. Dependiendo del
tipo de alimentación es posible adoptar una u otra estrategia para acotar
la potencia suministrada al sensor. Si se alimenta el sensor con una onda
sinusoidal puede, o bien limitarse la tensión de alimentación, o bien
utilizarse una sinusoide pulsada, empleando una configuración de ciclo de
trabajo que garantice que no se supera el límite de potencia soportado. En
caso de utilizar un pulso rectangular, o un tren de pulsos rectangulares,
las opciones son similares, reducir el nivel de tensión o configurar el
ciclo de trabajo de la señal. El actuador de ultrasonidos carga el circuito
de alimentación con una impedancia que presenta un carácter variable con
respecto a la frecuencia de la onda eléctrica y que debe tenerse en cuenta
para calcular la transferencia de potencia. La impedancia equivalente
presenta un comportamiento capacitivo a frecuencias por debajo de la
frecuencia fundamental de oscilación del actuador, carácter resistivo a
frecuencias en torno a la frecuencia de resonancia y carácter inductivo a
frecuencias por encima de esta frecuencia. Por tanto, la máxima
transferencia de potencia ocurre en frecuencias cercanas a la frecuencia
natural de oscilación del actuador, donde éste se comporta como una
impedancia resistiva de un valor típico que ronda los 700 $\Omega$. La
potencia suministrada al actuador se reparte entre la frecuencia
fundamental de resonancia y los distintos armónicos a frecuencias
superiores. El actuador siempre vibra a su frecuencia fundamental de
resonancia o a frecuencias múltiplo de ésta independientemente de la
frecuencia de la onda eléctrica utilizada para el suministro de potencia.
Si bien esto es cierto, para evitar excitar armónicos distintos de la
frecuencia fundamental de resonancia es aconsejable que la frecuencia de la
onda eléctrica sea lo más próxima a la frecuencia de resonancia del
actuador, además esto garantiza la máxima transferencia de potencia.

En ensayos no destructivos mediante ultrasonidos es imprescindible utilizar
pulsos acústicos como el representado en la \cref{fig:pulse}, para poder
evaluar las propiedades del medio de propagación empleando una de las dos
técnicas expuestas en el \cref{chap:endus} en la \cref{sec:technics}. Como
se explica en dicho capítulo, el actuador continua vibrando a su frecuencia
fundamental de resonancia incluso después de que la onda de alimentación
pase a un estado bajo durante un tiempo determinado durante el proceso de
fabricación del dispositivo ---esto es algo que no se aprecia en la
figura---, dando lugar a lo que en teoría de \sig{endus} se conoce como
<<zona ciega>> o <<zona muerta>>, este fenómeno se explica con mayor
detalle en la \cref{sec:field}.

\begin{figure}
	\begin{center}
		\includegraphics{gis-pfc-ch1-04.mps}
	\end{center}
	\caption[Pulso acústico generado por el actuador de
	ultrasonidos]{Pulso acústico que se obtiene del actuador de
	ultrasonidos cuando se alimenta con un pulso rectangular.}
	\label{fig:pulse}
\end{figure}

El parámetro más importante asociado al actuador de ultrasonidos es el
nivel de presión acústica transmitida (\emph{Transmitting Sound Pressure
Level}, o \psig{spl}), representa la presión que la onda acústica ejerce
sobre la presión estática del aire a una distancia determinada del actuador
cuando éste se alimenta con una determinada tensión eficaz. Proporciona una
medida de la eficiencia del sensor y puede utilizarse conjuntamente con la
sensibilidad del sensor para calcular de forma teórica cual es el alcance
máximo de un experimento de ultrasonidos que emplea estos transductores. El
\sig{spl} es una magnitud que varía con la frecuencia de la onda eléctrica,
y alcanza su máximo en la frecuencia natural de resonancia del actuador. El
fabricante suele proporcionar, o bien el valor típico del \sig{spl} a la
frecuencia de resonancia del actuador, o bien un perfil de la magnitud
frente a la frecuencia de la onda eléctrica. Con independencia del tipo de
valor proporcionado, éste suele darse en escala logarítmica utilizando una
para ello una referencia de presión que también se especifica. Para
calcular el \sig{spl} que se da en condiciones distintas de aquellas en que
se miden las especificaciones, debe calcularse la variación en escala
lineal y después trasladar este resultado a escala logarítmica, antes de
poder operar con el \sig{spl}. El sensor de ultrasonidos presenta unas
propiedades muy similares a las propiedades del actuador, si bien se
caracteriza por una impedancia bastante mayor. En el sensor la conversión
se da en sentido contrario, las ondas acústicas de presión que inciden en
la cara externa del sensor se traducen en variaciones de la tensión
eléctrica que existe en bornes del dispositivo. Si la propiedad más
característica del actuador de ultrasonidos es el \sig{spl}, la propiedad
más característica del sensor es la sensibilidad. La sensibilidad suele
proporcionarse también en escala logarítmica, igual que el \sig{spl} suele
darse un valor típico o una característica de la propiedad frente a la
frecuencia que, en ocasiones se superpone con la gráfica del \sig{spl} si
los transductores pertenecen a una misma serie. La magnitud lineal es la
relación entre la presión detectada y la diferencia de potencial eléctrico
que se crea en bornes del sensor y, en este caso, representa una medida de
la eficiencia del sensor.
