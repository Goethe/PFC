\chapter{Subsistema de interacción con el medio físico}

Para que un sistema pueda considerarse un sistema de medida debe haber una
interacción con el objeto o evento de estudio. El propósito de este sistema
es medir las propiedades de un determinado medio físico, la madera de
palmera. Atendiendo a su función dentro del sistema este subsistema recibe
en esta memoria el nombre de subsistema para la interacción con el medio
físico. Es el único de los tres subsistemas que conforman el sistema de
medida que está directamente relacionado con los ensayos no destructivos,
mientras que el resto de subsistemas, el subsistema de adquisición y el
subsistema de control y presentación, se encuentran más ligados al
tratamiento digital de señales. En otras palabras, en la ejecución de un
\sig{endus} el subsistema para la interacción con el medio físico es
imprescindible, el resto de elementos del sistema no lo son, y pueden ser
sustituidos por otros que los suplan.

El subsistema para la interacción con el medio físico está compuesto por
los elementos presentes en la \cref{fig:submedium}. Como puede verse en el
esquema, los transductores son dispositivos que interactúan con el medio
(por medio de una onda acústica). El acondicionador conectado al sensor
adecua la señal eléctrica que el dispositivo genera de modo que la
información que transporta pueda transmitirse correctamente a etapas
posteriores del sistema. El acondicionador que acompaña al actuador genera
una señal de características determinadas con la que el dispositivo es
alimentado de forma que su respuesta pueda predecirse.

\begin{figure}
    \begin{center}
	\includegraphics{gis-pfc-ch1-01.pdf}
    \end{center}
    \caption[Subsistema para la interacción con el medio físico]{Esquema
    que representa los distintos elementos o bloques presentes en el
    subsistema para la interacción con el medio físico.}
    \label{fig:submedium}
\end{figure}

Este capítulo se divide pues en tres apartados, en el primero se describen
los transductores empleados durante el proyecto, y se hace una pequeña
observación de su utilidad en relación con el tipo de ensayo
programado. Los dos apartados posteriores describen por su parte los
circuitos acondicionadores presentes en el subsistema.

\section{Transductores de ultrasonidos}\label{sec:transducers}

Los transductores de ultrasonidos son transductores piezoeléctricos y, por
tanto, entran en la categoría de transductores generadores. La principal
característica de un sensor generador es que al ser expuesto a la magnitud
no eléctrica de interés genera una diferencia de potencial eléctrico
proporcional sin necesidad de una alimentación eléctrica. Este efecto es
reversible y puede utilizarse para provocar una respuesta no eléctrica a
una variación en una corriente eléctrica, para hacerlo hay que emplear un
actuador. Los actuadores y los sensores generadores suelen presentar
características similares debido a la reversibilidad del efecto que
gobierna su funcionamiento.

En determinados materiales aparece una polarización eléctrica si son
sometidos a un esfuerzo que provoca deformación, este fenómeno se conoce
como efecto piezoeléctrico y es un efecto reversible. Si se aplica una
diferencia de potencial eléctrico a dos de las caras de un material
piezoeléctrico en éste aparece una deformación. Un actuador de ultrasonidos
vibra cuando se excita con una onda eléctrica y esta vibración se transmite
al medio en forma de onda de presión. La señal ultrasónica se propaga por
el medio y finalmente incide en el sensor, ejerciendo una presión sobre su
superficie que es traducida en una señal eléctrica.

La principal diferencia entre los transductores de ultrasonidos y los
transductores acústicos convencionales es que los transductores de
ultrasonidos trabajan a frecuencias superiores a los 20 kHz donde la onda
acústica se hace inaudible para el oído humano, de esta manera es posible
transmitir la energía acústica en pulsos estrechos. Los transductores de
ultrasonidos pueden ser de dos tipos dependiendo de como se construyan,
abiertos o cerrados. Los transductores de tipo abierto están formados por
una pantalla protectora y la pieza de material piezoeléctrico está montada
sobre un cono metálico, suelen ser más eficientes y presentan un mayor
ancho de banda. No obstante, están menos preparados para trabajar en
condiciones adversas y presentan una frecuencia de resonancia menos
precisa. En transductores de diseño cerrado la pieza resonante está en
contacto directo con la carcasa protectora y ésta se ajusta durante el
proceso de fabricación para que el conjunto vibre a la frecuencia de
resonancia elegida, estos transductores son menos eficientes pero más
resistentes y más precisos. Los transductores empleados en este \sig{pfc}
son de tipo cerrado. La \vref{fig:transducers} muestra una representación
del encapsulado que reviste los transductores utilizados en el proyecto en
la que pueden apreciarse las dimensiones del dispositivo. En el encapsulado
del actuador existe una marca en forma de punto blanco que lo identifica y
que sirve para distinguirlo del sensor, es la única diferencia aparente
entre ambos. Las características técnicas de los transductores vienen
recogidas en el \cref{tab:transducers}.

\begin{figure}
    \begin{center}
	\includegraphics{gis-pfc-ch1-02.pdf}
    \end{center}
    \caption[Dimensiones de los transductores de
    ultrasonidos]{Dimensiones y forma de los transductores de
    ultrasonidos.}
    \label{fig:transducers}
\end{figure}

\begin{sidewaystable}
    \centering
    \begin{threeparttable}
	\begin{tabular}{l c c}
	    \toprule
	    & \multicolumn{2}{c}{Valor especificado\tnote{*}} \\
	    \cmidrule(l){2-3}
	    \multicolumn{1}{c}{Propiedad} & Transmisor & Receptor \\
	    \midrule
	    Nivel de presión acústica transmitida & 110 dB & --- \\
	    Sensibilidad & --- & -70 dB \\
	    Frecuencia fundamental de resonancia
	    & $40 \text{kHz} \pm 1 \text{kHz}$
	    & $40 \text{kHz} \pm 1 \text{kHz}$ \\
	    Ancho de banda a -6 dB & $2.5$ kHz & $3.0$ kHz \\
	    Ancho del lóbulo principal & 60º & --- \\
	    Máxima potencia disipable & 200 mW (rms) & --- \\
	    Impedancia & $700\ \Omega$ & $30\ \text{k}\Omega$ \\
	    Capacidad & $2 \text{nF} \pm 20\%$
	    & $2 \text{nF} \pm 20\%$ \\
	    Tiempo de subida & $700\ \mu\text{s}$ & --- \\
	    Rango de temperatura de funcionamiento %
	    & (-20, 60) [ºC] & (-20, 60) [ºC] \\
	    \bottomrule
	\end{tabular}
	\begin{TableNotes}
	    \tnotetext{*}{Todas las magnitudes del cuadro se dan con
	    respecto a la frecuencia natural de resonancia de los
	    transductores.}
	\end{TableNotes}
    \end{threeparttable}
    \caption[Características de los transductores]{Características de
    los transductores empleados en el sistema de medida.}
    \label{tab:transducers}
\end{sidewaystable}


\subsection{El actuador de ultrasonidos}

En general los actuadores de ultrasonidos no son muy exigentes y admiten
ser alimentados para su funcionamiento con señales de formas muy diversas
como, por ejemplo, sinusoides u ondas rectangulares. Sin embargo, aunque
oscilan siempre a su frecuencia natural de resonancia, los actuadores
alimentados con señales de forma no sinusoidal emiten más intensamente
armónicos secundarios, comportamiento que puede ser indeseado. Pueden
utilizarse también para el propósito de alimentar un actuador señales
moduladas (pulsadas) a fin de introducir un ciclo de trabajo y conseguir
señales de menor potencia, al alimentar el actuador con una señal pulsada
no se modifica su comportamiento y puede servir para limitar la potencia
que disipa (que no puede superar el límite establecido por el fabricante,
en este caso 200 mW). Si se desea trabajar con pulsos acústicos, para
conseguir que el actuador los emita debe alimentarse con un pulso
rectangular o una portadora modulada por un pulso de este tipo (véase la
\cref{fig:pulse}).

\sshortpage{}

\begin{figure}
	\begin{center}
		\includegraphics{gis-pfc-ch1-03.pdf}
	\end{center}
	\caption[Pulso acústico generado por el actuador de
	ultrasonidos]{Pulso acústico que se obtiene del actuador de
	ultrasonidos cuando se alimenta con un pulso rectangular.}
	\label{fig:pulse}
\end{figure}

El actuador de ultrasonidos carga el circuito de alimentación con una
impedancia que presenta carácter variable, la componente reactiva de la
impedancia varía con la frecuencia de la onda empleada para excitar el
transductor. La potencia transferida al actuador depende del valor que
adopta esta impedancia. A frecuencias por debajo de la frecuencia natural
de resonancia del actuador muestra un comportamiento capacitivo, en torno a
la frecuencia de resonancia la reactancia es prácticamente nula, y a
frecuencias superiores adopta valores inductivos. Por tanto, resulta más
fácil encontrar una configuración en la que se obtenga máxima transferencia
de potencia a frecuencias en torno a la frecuencia natural de resonancia
del actuador. Por otro lado, la potencia suministrada al actuador se
reparte entre la frecuencia fundamental de resonancia y los distintos
armónicos a frecuencias superiores. El actuador siempre vibra a su
frecuencia fundamental de resonancia o a frecuencias múltiplo de ésta,
independientemente de la frecuencia o la forma de la onda eléctrica
utilizada para su suministro de potencia. Si bien esto es cierto, para
evitar excitar armónicos secundarios distintos de la frecuencia fundamental
de resonancia es aconsejable que la frecuencia de la onda eléctrica sea lo
más próxima posible a la frecuencia de resonancia del actuador, además esto
garantiza la máxima transferencia de potencia.

\sshortpage{}

El parámetro más importante asociado al actuador de ultrasonidos es el
nivel de presión acústica transmitida (\emph{Transmitting Sound Pressure
Level}, o \psig{spl}), representa la presión que la onda acústica ejerce
sobre la presión estática del aire a una distancia determinada del actuador
cuando éste se alimenta con una señal eléctrica de una determinada
intensidad (nivel de tensión eficaz). Proporciona una medida de la
eficiencia del actuador y puede utilizarse conjuntamente con la
sensibilidad del sensor para calcular de forma teórica cual es el alcance
máximo de un experimento de ultrasonidos que emplea dichos transductores.
El \sig{spl} es una magnitud que varía con la frecuencia de la onda
eléctrica utilizada para alimentar el actuador y alcanza su máximo a la
frecuencia natural de resonancia de éste. El fabricante suele proporcionar,
o bien el valor típico del \sig{spl} a la frecuencia de resonancia, o bien
un perfil de la magnitud frente a la frecuencia de la onda eléctrica. Con
independencia de que el \sig{spl} se presente mediante un valor numérico o
mediante un gráfico, suele proporcionarse en unidades en base a una escala
logarítmica que toma como referencia un valor de presión que se especifica
también. Para calcular el \sig{spl} que se da en condiciones distintas de
aquellas en que se miden las especificaciones debe calcularse la variación
en escala lineal y después trasladar este resultado a escala logarítmica.


\subsection{Sensor de ultrasonidos}

El sensor de ultrasonidos presenta unas propiedades muy similares a las
propiedades del actuador si bien se caracteriza por presentar una
impedancia característica bastante mayor. En el sensor la conversión se da
en sentido contrario, las ondas acústicas de presión que inciden en la cara
externa del sensor se traducen en variaciones de la tensión eléctrica que
existe en bornes del dispositivo. Si la propiedad más característica del
actuador de ultrasonidos es el \sig{spl}, la propiedad más característica
del sensor es la sensibilidad. La sensibilidad suele proporcionarse también
en escala logarítmica, igual que ocurre con el \sig{spl} suele darse un
valor típico o una característica de la propiedad frente a la frecuencia
que en ocasiones se superpone con la gráfica del \sig{spl} si los
transductores pertenecen a una misma serie. La magnitud lineal es la
relación entre la presión detectada y la diferencia de potencial eléctrico
que se crea en bornes del sensor y constituye una medida de la eficiencia
del sensor.


\subsection{Observaciones sobre los transductores}

La madera de palmera es un medio peculiar y la atenuación que introduce en
la señal acústica es mucho mayor que la que introducen otras maderas. El
problema de emplear transductores de gama baja en un \sig{endus} radica en
que emiten una energía ultrasónica muy débil, insuficiente si lo que se
pretende es atravesar el tronco de una palmera. Para poder utilizar la
técnica de transmisión es imprescindible que la onda acústica atraviese el
medio, de lo contrario únicamente puede recurrirse a la técnica de
pulso"=eco. La técnica de pulso"=eco está pensada para ser aplicada con un
único transductor que debe funcionar secuencialmente como emisor y como
receptor, utilizar dos transductores dificulta ya de por sí la obtención de
buenos resultados, pero si además éstos son de gama baja las posibilidades
de obtener medidas de una calidad mínima son escasas. La teoría sostiene
pues que es imposible obtener resultados útiles en un ensayo no destructivo
con ultrasonidos en el que se emplean transductores de gama baja.
Posteriormente, en el \cref{subsec:transducerconclusions} esta hipótesis se
confirma, por lo que ha resultado necesario recurrir a otro tipo de
transductores para poder efectuar los ensayos previstos, este tema vuelve a
tratarse más adelante durante la segunda parte de la memoria.


\section{Circuito de acondicionamiento del sensor}\label{sec:rxco}

Generalmente, el núcleo de un sistema de medida ---su unidad de
procesamiento, almacenamiento y/o presentación--- está formado por uno o
varios dispositivos estándar como son un osciloscopio o un analizador de
espectros. Los dispositivos de este tipo son eficaces en el ejercicio de
sus funciones y dada su versatilidad y precisión cumplen las expectativas
de los usuarios en la gran mayoría de situaciones, por lo que son
ampliamente utilizados. Sus características son bien conocidas, de modo que
hacer uso de dispositivos estándar simplifica el diseño de un sistema de
medida. En el caso de optar por el uso de una alternativa a los
dispositivos estándar en la configuración de un sistema de medida ello no
suele estar motivado por una razón de necesidad si no en la mayoría de los
casos por un interés particular en aplicaciones concretas y funcionalidades
añadidas.

No obstante, la señal que genera un sensor no es adecuada para ser
procesada directamente por un dispositivo estándar, es necesario
acondicionar previamente dicha señal; de no hacerlo se perdería la mayor
parte de la información que transmite. Con el término acondicionador se
designa el circuito electrónico que se interpone entre el sensor y el
dispositivo estándar y cuya finalidad es adecuar la señal generada por el
sensor a las características de la unidad de procesado del sistema. Con
éste propósito un circuito acondicionador puede ser diseñado y construido,
según la situación lo requiera, para desempeñar una o varias de las
funciones listadas a continuación.

\begin{itemize}
	\item Acotar el recorrido de la señal y confinarlo en unos márgenes
	    de tensión apropiados, evitando una posible saturación y
	    eventual daño de etapas posteriores.
	\item Proporcionar adaptación de impedancias entre el sensor y el
	    dispositivo estándar de modo que la transferencia de señal sea
	    máxima.
	\item A menudo los sensores proveen una señal de amplitud
	    comparable al ruido, el circuito acondicionador es responsable
	    de amplificar la señal manteniendo ---o mejorando si es
	    posible--- el nivel de \sig{snr}. Los circuitos
	    acondicionadores están integrados, en caso de que sea
	    necesario, por amplificadores de instrumentación útiles por sus
	    altos índices de \sig{cmrr}.
	\item El acondicionador de un sistema de medida ha de intentar
	    filtrar todas las componentes de señal indeseadas susceptibles
	    de causar algún tipo de distorsión en los resultados de una
	    prueba.
	\item Ha de modular o demodular la señal de acuerdo a las
	    necesidades de la etapa subsiguiente.
\end{itemize}


\subsection[Requisitos de diseño del circuito acondicionador]{Requisitos de
diseño que debe satisfacer el circuito acondicionador}

El acondicionador es un circuito complejo que requiere de un diseño
elaborado. La función que ejerce el circuito acondicionador en el sistema
de medida es, en definitiva, muy concreta: a partir de la señal generada
por el sensor ofrece una señal que puede ser procesada adecuadamente por un
dispositivo estándar. El diseño del circuito depende entonces
exclusivamente de tres aspectos: el sensor utilizado, el dispositivo
estándar que sirve de interfaz para con el resto del sistema, y las
condiciones de funcionamiento.

De lado del sensor es preciso determinar: la impedancia equivalente del
sensor; el tipo de señal que genera (su amplitud y frecuencia, su densidad
espectral de potencia, y si se encuentra o no modulada); por último, la
presencia de ruido o señales interferentes y el tipo de los mismos.

Las funciones de procesamiento, almacenamiento y presentación de la señal
se reparten en el sistema de medida que ocupa a este proyecto
(\vref{fig:digmeasstm}) en dos entidades físicas distintas: la tarjeta de
adquisición digital\footnote{Concretamente la tarjeta de adquisición
realiza las funciones de: conversión de analógico a digital de la señal;
reordenación y manipulación de los datos digitales; y almacenamiento
intermedio de las muestras antes de servirlas al ordenador anfitrión,
evitando fallos en la transmisión y garantizando que los datos se reciben
en el mismo orden en el que se envían. En el \cref{chap:acquisition} se
profundiza en la descripción del funcionamiento de la tarjeta \kpci{}.} y
el ordenador, que además gobierna el funcionamiento del sistema. Ambas
entidades pueden ser consideradas dispositivos estándar de acuerdo a la
definición dada en la introducción a esta sección. De ambos dispositivos,
configurados en cascada, el acondicionador interactúa de manera directa
sólo con la tarjeta de adquisición. Este hecho es relevante de cara al
diseño del circuito, pues significa que únicamente depende de las
características de la tarjeta y no de las del ordenador. Al interponerse la
tarjeta de adquisición entre el acondicionador y el ordenador, realizando
gran parte del procesado de la señal, la posible repercusión que el diseño
del acondicionador pudiese tener en los datos que recibe el ordenador es
despreciable. Siendo así, es preciso tener en consideración las
características de la tarjeta que influyen en la transmisión de la
información a lo largo del sistema, éstas son: impedancia de entrada, y
rango de tensiones límite a partir del cual se produce saturación del
dispositivo.

En cuanto a las condiciones de trabajo: temperatura ambiente, presencia de
radiación, presencia de impulsos eléctricos\dots; pueden ser muy variadas
pero se asume que se trabaja en condiciones estándar y no se profundizará
más en el tema.


\subsection{Circuito propuesto}

La \cref{fig:sensor-conditioner} muestra una representación esquemática del
circuito propuesto para el acondicionamiento del sensor de ultrasonidos.
Está compuesto por una única etapa basada en un amplificador de
instrumentación, el \sig{ina155} de Texas Instruments. El \sig{ina155} es
un amplificador de tecnología \sig{cmos}, bajo coste y bajo consumo, capaz
de funcionar polarizado con una única fuente de alimentación o batería (a
partir de aquí se utilizarán indistintamente los términos fuente de
alimentación o batería para referirse a la referencia de tensión utilizada
para polarizar los circuitos acondicionadores), presenta una ganancia
máxima de 50 V/V que se mantiene plana más allá de los 100 kHz (ganancias
inferiores se mantienen por encima de este margen de frecuencias).

\begin{figure}
    \begin{center}
	\includegraphics{gis-pfc-ch1-04.pdf}
    \end{center}
    \caption[Circuito de acondicionamiento del sensor
    piezoeléctrico]{Circuito propuesto para el acondicionamiento del sensor
    de ultrasonidos.}
    \label{fig:sensor-conditioner}
\end{figure}

Las ventajas de utilizar un amplificador de instrumentación son varias, de
las que destacan dos, que al estar íntimamente relacionadas pueden
considerarse como una sola: la configuración del amplificador de
instrumentación permite conectar el sensor de ultrasonidos a sus dos
entradas como si se tratase de una fuente o referencia de tensión flotante,
es decir, sin necesidad de una tercera referencia de tensión; al conectar
el transductor de esta forma al amplificador, el alto rechazo del modo
común de éste garantiza que cualquier componente en modo común generada por
el sensor piezoeléctrico sea en su mayor parte atenuada. Destaca también la
alta impedancia de entrada vista en ambos puertos de entrada del
amplificador, además, la configuración del \sig{ina155} ---al igual que
ocurre con otros amplificadores de instrumentación--- impide que el sensor
se vea involucrado en la red o redes de realimentación presentes en el
circuito, lo cual simplifica en gran medida la predicción de la señal
observada a la salida del integrado.

Por otra parte, el hecho de utilizar un amplificador de bajo
consumo\footnote{Un amplificador es considerado de bajo consumo si la
corriente que consume de la batería en estado de reposo (esto es, cuando la
entrada se cortocircuita a tierra) es muy pequeña. El valor típico de
consumo de corriente en reposo del \sig{ina155} es de 1,7 mA.} y capaz de
funcionar polarizado con una única batería (el \sig{ina155} está
especialmente diseñado para trabajar en estas condiciones) presenta la
ventaja de un consumo reducido que posibilita el uso de la aplicación en
pruebas de campo, bastaría con alimentar el circuito mediante una batería
(en este caso concreto, el circuito propuesto puede funcionar
ininterrumpidamente durante un espacio de tiempo de aproximadamente unas
cinco horas y media con una pila de petaca alcalina de 9 V).

La elección del \sig{ina155} se ha hecho teniendo en cuenta también que la
respuesta en frecuencia del amplificador es plana en torno a la frecuencia
de resonancia del sensor. En los \sig{endus} la onda ultrasónica suele
verse afectada por la dispersión y se desplaza en frecuencia, por lo que un
amplificador cuya respuesta es plana en torno a la frecuencia de trabajo es
especialmente apropiado para la realización de este tipo de ensayos.


\subsubsection{Funcionamiento del circuito}

Al polarizarse el circuito con una única fuente de alimentación
($V^+_{cc}$) es necesario introducir una componente en continua en la señal
de entrada para evitar que el acondicionador recorte el ciclo negativo de
la señal. Con el mismo propósito se polariza la entrada de referencia del
integrado. Para ello se utiliza una tensión de valor igual a $V^+_{cc}/2$,
que garantiza el mayor recorrido de la señal de salida sin que ninguno de
los dos ciclos de la señal sature el amplificador. Obviamente, al
configurar el circuito de esta manera la señal de salida oscilará en torno
a la tensión de referencia y no respecto al nivel de cero.


\section{Circuito de acondicionamiento del actuador}

\sshortpage{}

A menudo el concepto de circuito acondicionador está ligado únicamente al
concepto de sensor. En un sistema de medida, por el contrario, el circuito
que acompaña al actuador también se conoce como circuito acondicionador.
Ésto se debe a que en cierto modo el comportamiento de éste circuito es
similar al de un circuito acondicionador como se entiende normalmente.

Para extraer conclusiones a partir de la señal obtenida de un sensor es
necesario conocer cual es la señal que la origina. Resulta pues
comprensible pensar que la señal transmitida contiene también información
importante. Siguiendo esta línea de pensamiento se llega a la conclusión de
que no es útil alimentar un actuador con cualquier señal, ya que la señal
que emite debe conocerse. En ocasiones no es posible o práctico averiguar
directamente cómo es la señal que emite un actuador, sin embargo,
conociendo la respuesta del actuador al impulso es fácil determinar que
señal emite si se conoce la señal con la que se alimenta. Podría afirmarse
también entonces que la señal que alimenta al actuador contiene información
valiosa. Si la actividad del circuito de alimentación depende del sistema,
puede decirse para terminar que el acondicionador recibe información
procedente del sistema. En otras palabras, el circuito de acondicionamiento
del actuador transforma la información que recibe del sistema en
información útil para el actuador. En conjunto, si se observa el sistema
como un conducto por el que circula información, el circuito acondicionador
de la sección de emisión ocupa la posición contraria que el circuito
acondicionador de la sección de recepción, realizando la misma tarea
(conversión del formato de la información) pero en sentido contrario (del
sistema al transductor). Además, teniendo en cuenta que el circuito
acondicionador que acompaña al actuador realiza otras funciones que
permiten que la información se transmita al actuador, puede considerarse
este circuito como un verdadero circuito acondicionador.


\subsection[Requisitos de diseño del circuito acondicionador]{Requisitos de
diseño que debe satisfacer el circuito acondicionador}

Aclarado ésto se procede a determinar que parámetros han de tenerse en
cuenta en el diseño del circuito acondicionador. En \sig{endus} el actuador
debe emitir al medio un pulso o una señal pulsada. Para conseguir que el
actuador emita al medio un pulso acústico debe recibir un pulso eléctrico
rectangular o una señal modulada por un pulso rectangular. Para controlar
mínimamente la potencia de los armónicos secundarios es preferible
administrar al actuador una señal modulada por un pulso de duración
limitada. En principio, se puede pensar que los sensores pueden llegar a
utilizarse en la realización de pruebas de campo. Para no tener que
depender de una fuente de alimentación, se fija una condición de diseño:
que el circuito pueda funcionar con una batería. Es difícil obtener una
señal sinusoidal mediante un pequeño circuito alimentado con una batería,
es más práctico obtener una señal rectangular. La duración del pulso (del
pulso acústico y, en consecuencia, del pulso eléctrico que alimenta al
actuador) la fija el tipo de experimento que se va a realizar. Como a
priori no se conocen las necesidades concretas del experimento puede
fijarse como condición de diseño del circuito acondicionador que sea capaz
de generar una señal modulada por un pulso cuya duración se pueda
seleccionar.

Los ensayos se van a realizar de forma manual y no automática, es decir,
cada vez que se desea transmitir un pulso acústico al medio es un operador
el que en realidad acciona el circuito acondicionador. Por eso, del lado
del sistema el único requisito de diseño es que el circuito presente un
interruptor o cualquier otro mecanismo que sirva para accionarlo.

En \sig{endus} efectuados sobre madera de palmera la onda acústica se ve
particularmente afectada por la dispersión. Por este motivo es importante
transmitir pulsos acústicos de la mayor potencia posible. Ésto condiciona a
su vez el circuito de acondicionamiento del actuador, que debe transmitirle
una señal lo más intensa posible, garantizando una buena transferencia de
potencia (o lo que es lo mismo, debe existir entre acondicionador y
actuador adaptación de impedancias). La máxima potencia que puede disipar
el sensor es de 200 mW eficaces, lo que significa que en régimen pulsado,
de forma instantánea, puede disipar una potencia mayor. Para el propósito
de este proyecto se decide que la señal destinada a alimentar el sensor
debe alcanzar al menos una amplitud de 15 Vpp (obviamente debe suministrar
la intensidad que corresponde a esos niveles de tensión). Por otra parte,
para conseguir la máxima transferencia de posible se fija también como
condición de diseño que la señal generada por el acondicionador ha de
oscilar a la frecuencia natural de resonancia del actuador, de modo que la
impedancia observada por el circuito sea de carácter resistivo y su valor
sea cercano a los 500 $\Omega$. Siendo así es posible garantizar la máxima
transferencia de potencia siempre y cuando la impedancia de salida del
osciloscopio sea cercana a 0.


\subsection{Circuito propuesto}

Dispuestas las condiciones de diseño expuestas en el apartado anterior se
propone el circuito representado en la \cref{fig:actuator-conditioner} para
el acondicionamiento del actuador. Este circuito está constituido por
cuatro etapas:

\begin{itemize}
    \item Una primera etapa que cuenta con un interruptor para que el
	operario pueda accionar el circuito y generar el pulso acústico.
    \item Una segunda etapa cuyo elemento central es un temporizador en
	configuración monoestable.
    \item La tercera etapa está formada también en base a un temporizador
	pero en este caso configurado en su configuración astable.
    \item Por último el circuito dispone de una etapa de salida que amplia
	la variación de tensión hasta el nivel requerido.
\end{itemize}

\begin{sidewaysfigure}[p]
	\begin{center}
		\includegraphics{gis-pfc-ch1-05.pdf}
	\end{center}
	\caption[Circuito de acondicionamiento del actuador]{Circuito
	propuesto para el acondicionamiento del actuador de ultrasonidos.}
	\label{fig:actuator-conditioner}
\end{sidewaysfigure}

El \sig{tlc555}, utilizado en este circuito, es un temporizador integrado
del tipo 555 fabricado en tecnología \sig{cmos}. Se trata de un
temporizador de bajo consumo y elevada precisión diseñado para funcionar
alimentado con una batería. El \sig{cd4049ub} por su parte es un \sig{ic}
en el que se integran seis inversores, también es capaz de funcionar a
partir del suministro eléctrico que entrega una batería.


\subsubsection{Funcionamiento del circuito}

Un 555 en su configuración monoestable genera, al percibir en su entrada de
disparo (\sig{trig}) una tensión por debajo de un umbral de $V^+_{cc}/3$,
un pulso de una duración determinada por la red $RC$ que lo acompaña. Si el
estado bajo persiste tras transcurrir un tiempo igual a la duración del
pulso el comportamiento de la tensión de salida del circuito es
impredecible. La primera etapa del circuito mantiene la tensión que entra
al temporizador de la segunda, por su puerta de disparo, por encima del
nivel de umbral. En el instante en el que el operario acciona el pulsador
la tensión de salida de la etapa pasa a un estado bajo cercano a cero, tras
el cual empieza a crecer exponencialmente (a medida que el condensador se
carga) hasta alcanzar su valor estable $V^+_{cc}$. La curva de crecimiento
exponencial que sigue la tensión está determinada por la resistencia y el
condensador conectados directamente a la entrada de disparo del 555.

La segunda etapa genera un pulso rectangular de 300 $\mu\text{s}$ (el
equivalente a 12 periodos de una señal que oscila a 40 kHz) que sirve para
modular la salida de la tercera etapa. La entrada de \sig{reset} de un 555
es activa a estado bajo, es decir, que mientras la salida del primer 555
permanece en el estado estable (en el que adquiere un valor próximo a cero)
el segundo 555 permanece inactivo. En el momento en el que detecta el pulso
generado por el primer temporizador, el segundo 555 empieza a generar un
tren de pulsos rectangulares. La frecuencia de oscilación de la señal
rectangular observada a la salida de la tercera etapa está determinada por
la red de resistencias y condensadores conectada a las entradas de
descarga, umbral y disparo del temporizador. Esta red está formada por un
condensador y dos resistencias cuyos valores se ha seleccionado para que
la señal que genera el circuito oscile a 40 kHz.

La señal que ve la cuarta etapa a su entrada es pues una señal rectangular
que oscila a 40 kHz entre 0 y $V^+_{cc}$ modulada por un pulso de 300
$\mu\text{s}$ de duración. La última etapa del circuito duplica la amplitud
de pico a pico de la señal de modo que el actuador es finalmente alimentado
por una señal cuya amplitud es igual a $2 V^+_{cc}$ Vpp. Como puede
observarse en el esquemático cada extremo del actuador está conectado a un
condensador que a su vez está conectado a la salida de dos de los
inversores integrados en el \sig{cd4049ub}. Al cortocircuitar la salida de
dos inversores se suministra la corriente suficiente para alimentar el
transductor, por su parte la función de los condensadores es la de filtrar
la componente en continua de la señal e impedir que ésta llegue al
dispositivo de ultrasonidos.
