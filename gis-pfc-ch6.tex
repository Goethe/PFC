\chapter{Caracterización de la madera de palmera en ultrasonidos}

El interés de la madera como material de construcción, o materia prima para
la fabricación de herramientas, instrumentos musicales, mobiliario o
carpintería naval ---por citar algunos ejemplos de la multitud de usos que
recibe la madera en la actualidad--- ha originado desde principios de siglo
un interés por el estudio de sus propiedades físicas. Recientemente los
ensayos no destructivos por ultrasonidos han tenido un papel importante en
el estudio de las características estructurales de la madera, especialmente
en la detección de discontinuidades como grietas, presencia de
acebolladuras o nudos. El interés por las propiedades estructurales de la
madera se justifica asimismo puesto que en muchas de sus aplicaciones de su
integridad depende la seguridad de sus usuarios, viendo como ejemplo más
claro de ello el uso de madera en la construcción.

Además, existe un interés patente por otras propiedades de la madera, como
sus propiedades mecánicas, al conocer las propiedades mecánicas de
distintos tipos de madera en distintas circunstancias es posible determinar
el mejor uso para cada tipo; o acústicas, la madera es útil en la creación
de cavidades resonantes o como aislante acústico pues muestra una alta
capacidad para atenuar el ruido. Los \sig{endus} han demostrado ser un
método efectivo para determinar las propiedades mecánicas y acústicas de la
madera.

Las palmeras, aunque también útiles para la construcción por sus
propiedades (al ser una madera blanda y ligera, la madera palmera se
utiliza sobre todo en la construcción de tejados y tejadillos), han pasado
a ser un importante elemento decorativo en los núcleos urbanos, e incluso
en ciertas localizaciones constituyen un elemento importante del patrimonio
cultural y medioambiental.

Como parte de la decoración urbana la palmera está sometida a la acción de
un medio hostil, en el que se dan grandes temperaturas y en el que sufre la
acción de agentes externos que en otros entornos no existen o muestran
menor actividad. Como género de plantas apreciado, existe un interés por
saber como afecta a la palmera su entorno, conocimiento que puede
alcanzarse contrastando las propiedades físicas de la madera de palmera
cuando está sujeta a la acción de distintos medios. Así mismo, la madera de
palmera defectuosa también puede ser causa de accidentes ---por ejemplo,
los jardineros debe subirse a las copas de las palmeras para su
mantenimiento, las palmeras que presentan deficiencias estructurales
constituyen un riesgo para su integridad física--- por lo que el
conocimiento de las características estructurales de un ejemplar,
especialmente de los ejemplares in vivo, adquiere vital importancia. Las
aplicaciones basadas en ultrasonidos parecen especialmente adecuadas al
tratar de deducir las propiedades físicas de palmeras vivas, pues cuentan
con la ventaja de no dañar el ejemplar.


\section{Objetivos del experimento}

La madera de palmera es desde el punto de vista de su estudio como medio de
propagación acústico dentro del conjunto de las distintas maderas cuanto
menos un medio peculiar podría decirse incluso, sin temor a equivocarse,
que es atípico. Ello se debe a las características propias del género; la
palmera, a pesar de ser una planta de tipo arbustivo crece ---la mayor
parte de las veces motivada por la acción del ser humano--- en forma
arborescente. Su aspecto semejante a un árbol no debe dar lugar a duda, su
tallo arborescente no muestra formación de madera secundaria ---ya que los
haces vasculares (fibras) que lo integran son de tipo colateral cerrado, lo
que significa que no conservan el procambium (tejido celular presente en
los árboles responsable de su crecimiento) después de alcanzar el estado
adulto, y ello impide un crecimiento ulterior--- y carece de un verdadero
tronco\footnote{En realidad el tallo de una palmera es leñoso debido a que
contiene esclerénquima fibroso xilemático que le sirve de sostén.}.

De cara a su estudio, el tallo de una palmera presenta una estela (sección
transversal) en la que los haces vasculares formados en su mayor parte por
fibras xilemáticas y vasos liberianos (aquellos que transportan nutrientes
desde la parte autótrofa de la planta a sus partes basales subterráneas),
no se agrupan en ninguna estructura reconocible, si no que están dispersos,
formando espacios entre sí rellenos con material blando y húmedo, y aire.
La gran proporción de humedad, el espacio vacío, y la distribución
heterogénea de las fibras, dificulta enormemente la propagación de ondas de
media y alta frecuencia a través de un tallo de palmera, algo que no ocurre
en otros tipos de madera.

Actualmente existen muy pocas referencias bibliográficas que documenten que
tipo de información, parámetros o propiedades físicas pueden extraerse de
un ensayo por ultrasonidos en madera de palmera. Menos aún si se trata de
ensayos realizados con transductores de impacto. En general, es muy difícil
encontrar información que relacione los parámetros físicos deducibles
mediante técnicas no destructivas basadas en ultrasonidos y las
características microestructurales inducidas por condiciones adversas en
madera de palmera.
