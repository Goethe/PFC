\chapter{Caracterización de la madera de palmera en ultrasonidos}

El interés de la madera como material de construcción, o materia prima para
la fabricación de herramientas, instrumentos musicales, mobiliario o
carpintería naval ---por citar algunos ejemplos de la multitud de usos que
recibe la madera en la actualidad--- ha originado desde principios de siglo
un interés por el estudio de sus propiedades físicas. Recientemente los
ensayos no destructivos por ultrasonidos han tenido un papel importante en
el estudio de las características estructurales de la madera, especialmente
en la detección de discontinuidades como grietas, presencia de
acebolladuras o nudos. El interés por las propiedades estructurales de la
madera se justifica asimismo puesto que en muchas de sus aplicaciones de su
integridad depende la seguridad de sus usuarios, viendo como ejemplo más
claro de ello el uso de madera en la construcción.

Además, existe un interés patente por otras propiedades de la madera, como
sus propiedades mecánicas, al conocer las propiedades mecánicas de
distintos tipos de madera en distintas circunstancias es posible determinar
el mejor uso para cada tipo; o acústicas, la madera es útil en la creación
de cavidades resonantes o como aislante acústico pues muestra una alta
capacidad para atenuar el ruido. Los \sig{endus} han demostrado ser un
método efectivo para determinar las propiedades mecánicas y acústicas de la
madera.

Las palmeras, aunque también útiles para la construcción por sus
propiedades (al ser una madera blanda y ligera, la madera palmera se
utiliza sobre todo en la construcción de tejados y tejadillos), han pasado
a ser un importante elemento decorativo en los núcleos urbanos, e incluso
en ciertas localizaciones constituyen un elemento importante del patrimonio
cultural y medioambiental.

Como parte de la decoración urbana la palmera está sometida a la acción de
un medio hostil, en el que se dan grandes temperaturas y en el que sufre la
acción de agentes externos que en otros entornos no existen o muestran
menor actividad. Como género de plantas apreciado, existe un interés por
saber como afecta a la palmera su entorno, conocimiento que puede
alcanzarse contrastando las propiedades físicas de la madera de palmera
cuando está sujeta a la acción de distintos medios. Así mismo, la madera de
palmera defectuosa también puede ser causa de accidentes ---por ejemplo,
los jardineros debe subirse a las copas de las palmeras para su
mantenimiento, las palmeras que presentan deficiencias estructurales
constituyen un riesgo para su integridad física--- por lo que el
conocimiento de las características estructurales de un ejemplar,
especialmente de los ejemplares in vivo, adquiere vital importancia. Las
aplicaciones basadas en ultrasonidos parecen especialmente adecuadas al
tratar de deducir las propiedades físicas de palmeras vivas, pues cuentan
con la ventaja de no dañar el ejemplar.


\section{Objetivos del experimento}

La madera de palmera es desde el punto de vista de su estudio como medio de
propagación acústico dentro del conjunto de las distintas maderas cuanto
menos un medio peculiar podría decirse incluso, sin temor a equivocarse,
que es atípico. Ello se debe a las características propias del género; la
palmera, a pesar de ser una planta de tipo arbustivo crece ---la mayor
parte de las veces motivada por la acción del ser humano--- en forma
arborescente.

Ello se debe a las características propias del género, la palmera, a pesar
de su crecimiento en forma arborescente ---la mayor parte de las veces
motivado por la acción del ser humano--- es una planta de tipo arbustivo.
Es por ello que su tallo a pesar de ser leñoso muestra una composición
distinta a la madera de los árboles, no presenta madera secundaria pues sus
fibras no conservan procambium\footnote{Las fibras o haces vasculares de
las palmeras se denominan colaterales cerrados debido a esta propiedad. En
realidad, el tallo de una palmera es leñoso debido a que contiene
esclerénquima fibroso xilemático, un tejido celular elástico que le sirve
de sostén.} (tejido celular presente en los árboles responsable de su
crecimiento) con lo que carece de verdadero tronco.

De cara a su estudio, el tallo de una palmera presenta una estela (sección
transversal) en la que los haces vasculares formados en su mayor parte por
fibras xilemáticas y vasos liberianos (aquellos que transportan nutrientes
desde la parte autótrofa de la planta a sus partes basales subterráneas) no
se agrupan en ninguna estructura reconocible, si no que están dispersos
formando espacios entre sí rellenos con material blando y húmedo, y aire.
La gran proporción de humedad, el espacio vacío, y la distribución
heterogénea de las fibras dificulta enormemente la propagación de ondas de
media y alta frecuencia a través de un tallo de palmera, algo que no ocurre
en otros tipos de madera.

Actualmente existen muy pocas referencias bibliográficas que documenten que
tipo de información, parámetros o propiedades físicas pueden extraerse de
un ensayo por ultrasonidos en madera de palmera. En general, es muy difícil
encontrar información que relacione los parámetros físicos deducibles
mediante técnicas no destructivas basadas en ultrasonidos y las
características microestructurales inducidas por condiciones adversas en
madera de palmera. Menos aún si se trata de ensayos realizados con
transductores de impacto, pruebas realizadas con esta tecnología son muy
escasas.

Al disponer en el laboratorio de unos transductores de impacto,
recientemente obtenidos, se plantea como parte del proyecto fin de carrera
la realización de un estudio preliminar cuyos resultados puedan justificar
el interés en abordar un proyecto mucho más ambicioso en este campo. Cabe
remarcar aquí que como estudio preliminar este apartado del proyecto no
pretende aportar más que una serie de resultados que deberán valorarse
única y exclusivamente con el propósito aquí especificado.


\subsection{Elección de bajas frecuencias}

Es bien sabido (y se expone con detalle en el \vref{chap:endus} de esta
memoria) que las ondas ultrasónicas presentan un comportamiento variable
con respecto a la frecuencia. En este subapartado se justifica la elección
de las bajas frecuencias para el desarrollo de esta aplicación poniendo de
manifiesto cuales son las ventajas y desventajas que presentan en este tipo
de ensayos.


\subsubsection{Ventajas}

Como se observó en el capítulo anterior, uno de los principales
inconvenientes en los \sig{endus} es la aparición de ruido de grano. En
resumidas cuentas el ruido de grano es una señal originada en base al mismo
fenómeno que genera el eco que determina la presencia de defectos en el
material de estudio, la dispersión. Por ello es difícil diferenciar y
extraer el ruido de la señal recibida en un ensayo, lo que perjudica la
relación señal a ruido (por duplicado, ya que la dispersión forma parte de
la atenuación que afecta a la señal que se propaga) y dificulta la
obtención de resultados válidos. Las bajas frecuencias, mayores longitudes
de onda, no interactúan directamente con las partículas más pequeñas de un
medio por lo que presentan mayor inmunidad a este fenómeno y se ven
afectadas en menor medida por el ruido estructural.

Al verse menos afectadas por el efecto de la dispersión las ondas de baja
frecuencia se propagan satisfactoriamente a una mayor distancia permitiendo
el análisis de muestras a una mayor profundidad. Esta propiedad de las
ondas de baja frecuencia resulta útil cuando se desea explorar muestras de
un espesor notable o voluminosas como en este caso el tallo de un ejemplar
vivo de palmera. Además, en el caso de la madera de palmera, por su
composición, es preciso realizar los ensayos con ondas de baja frecuencia
ya que al beneficiarse de una mayor inmunidad frente a la dispersión son
las únicas capaces de propagarse a través de un medio como éste, realmente
hostil, hecho que por sí mismo justifica ampliamente su uso.


\subsubsection{Inconvenientes}

La principal desventaja de las bajas frecuencias frente a las medias y
altas frecuencias radica en que a medida que disminuye la frecuencia de la
onda empleada disminuye la bondad del ensayo, al menos en términos de
resolución. Esto puede no ser, sin embargo, demasiado importante
dependiendo de la aplicación, como es en el caso de este proyecto. Véase un
ejemplo explicativo: si lo que se desea es únicamente detectar un defecto y
no determinar su posición y forma exactas resulta más importante que la
onda ultrasónica alcance el sensor que que los resultados gocen de una
buena resolución.

Otro gran problema de las bajas frecuencias es que son más susceptibles de
viajar por la superficie de un material en lugar de atravesarlo.
Dependiendo del tipo de muestra y de sensor ésto puede resultar
inconveniente. En muestras en las que una de las dimensiones transversales
es muy pequeña la onda de superficie alcanza el sensor prácticamente al
mismo tiempo que la onda de propagación longitudinal, por lo que ambas
contribuciones se suman en el receptor actuando la onda superficial como
onda interferente. Sin embargo, en las muestras in vivo la distancia
recorrida por la onda superficial es prácticamente nula, tanto por el
diámetro de la muestra como por el relieve de su superficie, lo que
garantiza la viabilidad de los \sig{endus} en este sentido.


\section{Equipo utilizado}

Para la realización de las distintas pruebas que forman parte de este
estudio preliminar se ha utilizado un equipo determinado, de entre el que
merece la pena destacar los siguientes dispositivos y herramientas.

Los transductores de impacto o sondas piezoeléctricas de impacto son
transductores específicos para aplicaciones que implican la realización de
ensayos no destructivos en ejemplares de árboles in vivo. En este caso
resultan útiles también para evaluar las propiedades de las palmeras,
puesto que éstas crecen de forma arborescente. Están formados por un punzón
que se inserta levemente en el tallo del ejemplar que se desea estudiar sin
que ello le cause ningún daño, quedando la sonda bien sujeta. El punzón
canaliza eficazmente la onda acústica transmitida al tronco del árbol,
igualmente sirve para capturar la máxima energía que proviene del impulso
ultrasónico que se propaga por el medio. Cuentan también con una cabeza
metálica que recubre el propio transductor, dicha cabeza metálica está
diseñada para poder amartillarse y es capaz simultáneamente de resistir el
impacto, protegiendo la parte activa de la sonda, y de transmitir toda la
fuerza del golpe al árbol a través de la onda acústica. Finalmente, las
sondas de impacto disponen de un cable coaxial con un conector de rosca
compatible con osciloscopios y por el que transmiten la onda acústica.

Este tipo de transductores puede utilizarse indistintamente como actuador o
como sensor ya que es la estructura metálica externa la que estructura
metálica del transductor la que, al recibir el golpe del martillo, genera y
transmite la onda acústica. Por su parte, el componente activo de la sonda
realiza la transformación de onda de presión a onda eléctrica y únicamente
en este sentido\footnote{Es fácil observar que el sentido de la conversión
es único, puesto que la sonda coaxial que conecta el sensor con el
osciloscopio únicamente transmite información, en forma de señal eléctrica,
en el sentido transductor osciloscopio, pues los osciloscopios no generan
señales.}. Si el transductor actúa como sensor envía una señal eléctrica
correspondiente con la onda acústica recibida, por el contrario, si sirve
como actuador transmite la correspondiente a la onda transmitida; de ese
modo es fácil registrar ambas señales.

En concreto los transductores utilizados en este proyecto fin de carrera
han sido fabricados por el fabricante húngaro de equipos para la
realización de pruebas no invasivas
\href{www.fakopp.com/site/piezo}{fakopp}. Se trata de transductores que,
pese a estar diseñados para resistir el impacto de un martillo, ofrecen una
buena relación señal a ruido. Funcionan a una frecuencia de resonancia
propia de 45 kHz (es menester recordar que la banda de actuación de los
ultrasonidos parte de los 20 kHz y llega a los 1000 MHz, por lo que una
frecuencia de 45 kHz es en este contexto una frecuencia relativamente
baja).

Calibre de factura propia


\section{Metodología seguida durante las pruebas}

Las pruebas realizadas persiguen demostrar que existe una relación entre
los distintos parámetros de la señal recibida y las características del
medio en que se propaga. Si así fuese quedaría justificado un estudio más
ambicioso cuyo objetivo sería el de determinar de forma inequívoca dicha
relación. Los parámetros que se pretende evaluar para intentar dictaminar
que dicha relación en efecto existe son los listados a continuación:

\begin{itemize}
    \item Amplitud de la señal, para determinar en que medida se ha visto
	atenuada la onda sónica al propagarse.
    \item Aparición de dispersión, se evalúa si hay signos visibles del
	efecto de la dispersión en la onda recibida.
    \item Velocidad de propagación, el cociente del diámetro del ejemplar y
	el tiempo de vuelo ayudan a determinar la velocidad de propagación
	de la onda.
    \item Desplazamiento en frecuencia, se observa la posición del máximo
	del espectro frecuencial de la onda ultrasónica.
    \item Curva de atenuación, o forma de la señal, se evalúa si la caída
	de la señal es más o menos suave o abrupta.
    \item Aparición de secundarios, debidos a efectos varios como la
	propagación de onda superficial.
\end{itemize}

Estos parámetros se evalúan en los resultados recogidos al realizar
diferentes ensayos en muestras de distintos tipos, como son: palmeras
sanas, se toman muestras de palmeras ubicadas en distintas localizaciones,
realizando ensayos en ejemplares pertenecientes a múltiples variedades y
que presentan asimismo características variadas como grosor o tipo de
corteza; además se evalúa palmeras muertas o ejemplares que aparentan un
mal estado de salud, eucaliptos, pinos y, finalmente, postes de telefonía.
De ese modo es posible realizar una comparación posterior para tratar de
identificar una relación entre los parámetros estimados y las condiciones o
tipo de la muestra.

El lector ha de notar que no es posible extraer resultados válidos de
muestras de palmera en laboratorio. Ello es debido a que el tallo de la
palmera pierde sus propiedades originales cuando el ejemplar muere.
Mientras vive una palmera intercambia nutrientes y agua entre su copa y sus
partes basales a través de vasos que integran su tallo, principalmente
utilizando un mecanismo basado en la capilaridad. Al morir el individuo
este proceso se detiene y el tallo va perdiendo su composición húmeda
regularmente hasta que finalmente se seca por completo. Debido a esta otra
singularidad que presentan las palmeras, los resultados obtenidos en
muestras extraídas de los restos de un ejemplar no son confiables y no
aportan información válida para la finalidad de este estudio. Es por ello
por lo que los ensayos que aquí se han realizado en ejemplares, palmeras,
in vivo.

La relación entre las propiedades de la madera de palmera y los parámetros
de la señal evaluados es, como se ha afirmado anteriormente, desconocida;
pero no sólo eso, el modo en que las condiciones externas influyen en el
medio es también desconocido. Por último cabe destacar aquí, en parte para
el posterior entendimiento de las conclusiones y resultados expuestos en
este documento que: debido al motivo que se ha manifestado, carece de
sentido comparar los datos obtenidos de las ondas recibidas al realizar
ensayos en las distintas muestras siempre que no se haga en el contexto
---esto es, mirando también las características--- de la onda que parte del
emisor durante la realización de cada prueba.


\subsection{Procedimiento de medida}

El procedimiento de medida ha resultado tedioso y complejo por distintos
motivos. En primer lugar, al realizarse medidas de campo, es preciso
trasladarse a la ubicación en la que se encuentra localizada la muestra
para poder realizar las medidas. Las muestras potenciales, por su
naturaleza, se encuentran en terreno de difícil acceso como zanjas o
huertos, a los que es preciso trasladar el voluminoso equipo para poder
realizar las medidas. En cada ocasión se ha de desplazar el equipo hasta
las proximidades de la muestra y después dicho equipo debe disponerse de
modo que el ensayo pueda ser realizado. Tarea que no siempre es fácil,
debido sobre todo a la naturaleza del terreno que frecuentemente resulta
irregular y accidentado, con lo cual la simple acción de asentar una mesa
se antoja complicada.

Por otro lado, gran parte del equipo utilizado requiere de alimentación
eléctrica para funcionar, al operar en lugares apartados se ha precisado de
una batería para alimentar los distintos dispositivos utilizados. La
capacidad de la batería es limitada lo que limita la duración de una sesión
de toma de muestras. Además, al agotarse es preciso detener las pruebas
durante el periodo de recarga de la batería que dura aproximadamente un día
antes de reemprender los ensayos.

A todo ello debe añadirse que tanto el director como el proyectista en este
proyecto fin de carrera desconocían previamente a los ensayos el manejo de
los transductores de impacto que se han empleado para la realización de las
pruebas, lo cual ha dificultado, más aún si cabe, la ejecución de los
ensayos..
