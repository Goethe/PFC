\chapter{Caracterización de la madera de palmera en ultrasonidos}

El interés de la madera como material de construcción, o materia prima para
la fabricación de herramientas, instrumentos musicales, mobiliario o
carpintería naval ---por citar algunos ejemplos de la multitud de usos que
recibe la madera en la actualidad--- ha originado desde principios de siglo
un interés por el estudio de sus propiedades físicas. Recientemente los
ensayos no destructivos por ultrasonidos han tenido un papel importante en
el estudio de las características estructurales de la madera, especialmente
en la detección de discontinuidades como grietas, presencia de
acebolladuras o nudos. El interés por las propiedades estructurales de la
madera se justifica asimismo puesto que en muchas de sus aplicaciones de su
integridad depende la seguridad de sus usuarios, viendo como ejemplo más
claro de ello el uso de madera en la construcción.

Además, existe un interés patente por otras propiedades de la madera, como
sus propiedades mecánicas, al conocer las propiedades mecánicas de
distintos tipos de madera en distintas circunstancias es posible determinar
el mejor uso para cada tipo; o acústicas, la madera es útil en la creación
de cavidades resonantes o como aislante acústico pues muestra una alta
capacidad para atenuar el ruido. Los \sig{endus} han demostrado ser un
método efectivo para determinar las propiedades mecánicas y acústicas de la
madera.

Las palmeras, aunque también útiles para la construcción por sus
propiedades (al ser una madera blanda y ligera, la madera palmera se
utiliza sobre todo en la construcción de tejados y tejadillos), han pasado
a ser un importante elemento decorativo en los núcleos urbanos, e incluso
en ciertas localizaciones constituyen un elemento importante del patrimonio
cultural y medioambiental.

Como parte de la decoración urbana la palmera está sometida a la acción de
un medio hostil, en el que se dan grandes temperaturas y en el que sufre la
acción de agentes externos que en otros entornos no existen o muestran
menor actividad. Como género de plantas apreciado, existe un interés por
saber como afecta a la palmera su entorno, conocimiento que puede
alcanzarse contrastando las propiedades físicas de la madera de palmera
cuando está sujeta a la acción de distintos medios. Así mismo, la madera de
palmera defectuosa también puede ser causa de accidentes ---por ejemplo,
los jardineros debe subirse a las copas de las palmeras para su
mantenimiento, las palmeras que presentan deficiencias estructurales
constituyen un riesgo para su integridad física--- por lo que el
conocimiento de las características estructurales de un ejemplar,
especialmente de los ejemplares in vivo, adquiere vital importancia. Las
aplicaciones basadas en ultrasonidos parecen especialmente adecuadas al
tratar de deducir las propiedades físicas de palmeras vivas, pues cuentan
con la ventaja de no dañar el ejemplar.


\section{Objetivos del experimento}

La madera de palmera es desde el punto de vista de su estudio como medio de
propagación acústico dentro del conjunto de las distintas maderas cuanto
menos un medio peculiar podría decirse incluso, sin temor a equivocarse,
que es atípico. Ello se debe a las características propias del género; la
palmera, a pesar de ser una planta de tipo arbustivo crece ---la mayor
parte de las veces motivada por la acción del ser humano--- en forma
arborescente.

Ello se debe a las características propias del género, la palmera, a pesar
de su crecimiento en forma arborescente ---la mayor parte de las veces
motivado por la acción del ser humano--- es una planta de tipo arbustivo.
Es por ello que su tallo a pesar de ser leñoso muestra una composición
distinta a la madera de los árboles, no presenta madera secundaria pues sus
fibras no conservan procambium\footnote{Las fibras o haces vasculares de
las palmeras se denominan colaterales cerrados debido a esta propiedad. En
realidad, el tallo de una palmera es leñoso debido a que contiene
esclerénquima fibroso xilemático, un tejido celular elástico que le sirve
de sostén.} (tejido celular presente en los árboles responsable de su
crecimiento) con lo que carece de verdadero tronco.

De cara a su estudio, el tallo de una palmera presenta una estela (sección
transversal) en la que los haces vasculares formados en su mayor parte por
fibras xilemáticas y vasos liberianos (aquellos que transportan nutrientes
desde la parte autótrofa de la planta a sus partes basales subterráneas) no
se agrupan en ninguna estructura reconocible, si no que están dispersos
formando espacios entre sí rellenos con material blando y húmedo, y aire.
La gran proporción de humedad, el espacio vacío, y la distribución
heterogénea de las fibras dificulta enormemente la propagación de ondas de
media y alta frecuencia a través de un tallo de palmera, algo que no ocurre
en otros tipos de madera.

Actualmente existen muy pocas referencias bibliográficas que documenten que
tipo de información, parámetros o propiedades físicas pueden extraerse de
un ensayo por ultrasonidos en madera de palmera. En general, es muy difícil
encontrar información que relacione los parámetros físicos deducibles
mediante técnicas no destructivas basadas en ultrasonidos y las
características microestructurales inducidas por condiciones adversas en
madera de palmera. Menos aún si se trata de ensayos realizados con
transductores de impacto, pruebas realizadas con esta tecnología son muy
escasas.

Al disponer en el laboratorio de unos transductores de impacto,
recientemente obtenidos, se plantea como parte del proyecto fin de carrera
la realización de un estudio preliminar cuyos resultados puedan justificar
el interés en abordar un proyecto mucho más ambicioso en este campo. Cabe
remarcar aquí que como estudio preliminar este apartado del proyecto no
pretende aportar más que una serie de resultados que deberán valorarse
única y exclusivamente con el propósito aquí especificado.


\subsection{Elección de bajas frecuencias}

Es bien sabido (y se expone con detalle en el \vref{chap:endus} de esta
memoria) que las ondas ultrasónicas presentan un comportamiento variable
con respecto a la frecuencia. En este subapartado se justifica la elección
de las bajas frecuencias para el desarrollo de esta aplicación poniendo de
manifiesto cuales son las ventajas y desventajas que presentan en este tipo
de ensayos.


\subsubsection{Ventajas}

Como se observó en el capítulo anterior, uno de los principales
inconvenientes en los \sig{endus} es la aparición de ruido de grano. En
resumidas cuentas el ruido de grano es una señal originada en base al mismo
fenómeno que genera el eco que determina la presencia de defectos en el
material de estudio, la dispersión. Por ello es difícil diferenciar y
extraer el ruido de la señal recibida en un ensayo, lo que perjudica la
relación señal a ruido (por duplicado, ya que la dispersión forma parte de
la atenuación que afecta a la señal que se propaga) y dificulta la
obtención de resultados válidos. Las bajas frecuencias, mayores longitudes
de onda, no interactúan directamente con las partículas más pequeñas de un
medio por lo que presentan mayor inmunidad a este fenómeno y se ven
afectadas en menor medida por el ruido estructural. Todo lo cual significa
que las ondas de baja frecuencia se propagan satisfactoriamente a una mayor
distancia permitiendo el análisis de muestras de mayor grosor.

Es interesante subrayar que existen aplicaciones destinadas a caracterizar
materiales en las que el ruido de grano se toma como señal de interés. Si
bien esto es cierto, la madera de palmera es un medio hostil para las ondas
acústicas, hecho que sumado a las grandes dimensiones de los troncos de los
ejemplares tomados como muestra ha determinado la elección de señales de
baja frecuencia como medio para el estudio del material.


\subsubsection{Inconvenientes}

La principal desventaja de las bajas frecuencias frente a las medias y
altas frecuencias radica en que a medida que disminuye la frecuencia de la
onda empleada disminuye la bondad del ensayo, al menos en términos de
resolución. Esto puede no ser, sin embargo, demasiado importante
dependiendo de la aplicación, como es en el caso de este proyecto. Véase un
ejemplo explicativo: si lo que se desea es únicamente detectar un defecto y
no determinar su posición y forma exactas resulta más importante que la
onda ultrasónica alcance el sensor que que los resultados gocen de una
buena resolución.

Otro gran problema de las bajas frecuencias es que son más susceptibles de
viajar por la superficie de un material en lugar de atravesarlo.
Dependiendo del tipo de muestra y de sensor ésto puede resultar
inconveniente. En muestras en las que una de las dimensiones transversales
es muy pequeña la onda de superficie alcanza el sensor prácticamente al
mismo tiempo que la onda de propagación longitudinal, por lo que ambas
contribuciones se suman en el receptor actuando la onda superficial como
onda interferente. Sin embargo, en las muestras in vivo la distancia
recorrida por la onda superficial es prácticamente nula, tanto por el
diámetro de la muestra como por el relieve de su superficie, lo que
garantiza la viabilidad de los \sig{endus} en este sentido.


\section{Equipo utilizado}
