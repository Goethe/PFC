\chapter*{Introducción}

\section*{Organización de la segunda parte}

La segunda parte está dividida en dos capítulos: el quinto capítulo que
trata los fundamentos teóricos de los \sig{endus}; y el sexto en el que se
exponen los resultados extraídos de los ensayos, las conclusiones a las que
se ha llegado, así como las líneas de trabajo que se preven para futuros
proyectos relacionados con la materia.% , descripción del medio,
% resultados y conclusiones.


\begin{description}
	\item[Quinto capítulo] En el primer capítulo se describen de forma
		superficial los distintos elementos que intervienen en un
		\sig{endus} desde un punto de vista teórico. De las
		distintas técnicas existentes para combatir el ruido
		estructural se dan detalles sobre las técnicas de procesado
		por partición del espectro que son las utilizadas en este
		proyecto.
	% \item[Sexto capítulo] El sexto capítulo trata sobre el medio
	% en el que se realizan las pruebas experimentales, la madera de
	% palmera. Se proporciona una descripción teórica del material de
	% acuerdo con la documentación consultada. Se caracteriza el
	% material para un posterior análisis de los resultados encontrados
	% en las pruebas experimentales.
	\item[Sexto capítulo] El último capítulo recoge los resultados
		extraídos de las pruebas en forma de grafos y tablas
		comentados. Después se dan las conclusiones a las que se ha
		llegado a partir de estos resultados y finalmente se
		realizan una serie de comentarios sobre nuevas líneas de
		investigación que pueden seguirse de este proyecto
\end{description}
