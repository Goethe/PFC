\chapter*{Introducción}

\section*{Organización de la segunda parte}

La segunda parte está dividida en dos capítulos: el quinto capítulo, que
trata los fundamentos teóricos de los \sig{endus}; y el sexto, en el que se
exponen los resultados extraídos de los ensayos, las conclusiones a las que
se ha llegado, así como las líneas de trabajo que se preven para futuros
proyectos relacionados con la materia.

\begin{description}
    \item[Quinto capítulo] En el primer capítulo de esta parte se describe,
	de forma superficial, los distintos elementos que intervienen en un
	\sig{endus} desde un punto de vista teórico. De las distintas
	técnicas existentes para combatir el ruido estructural se dan
	detalles sobre las de procesado por partición del espectro que son
	las que se propone utilizar en futuros proyectos.
    \item[Sexto capítulo] El último capítulo recoge los resultados
	extraídos de las pruebas en forma de grafos y tablas comentados.
	Después se dan las conclusiones a las que se ha llegado a partir de
	estos resultados y finalmente se realizan una serie de comentarios
	sobre nuevas líneas de investigación que pueden seguirse de este
	proyecto
\end{description}
