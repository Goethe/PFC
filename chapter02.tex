\chapter{Tarjeta de adquisici�n de se�ales}

Para preparar el sistema de adquisici�n y procesado de se�ales se dispuso en origen de una tarjeta de adquisici�n con interfaz \textsc{pci}. En concreto se ha empleado el modelo \kpci{} de la casa \emph{Keithley}. A continuaci�n se exponen las caracter�sticas t�cnicas que ofrece este dispositivo, en apartados siguientes una breve descripci�n funcional y la disposici�n y usos de los puertos presentes en el mencionado dispositivo.


\section{Caracter�sticas t�cnicas del hardware}

La tarjeta \kpci{} es capaz de las operaciones de adquisici�n y muestreo de se�ales anal�gicas, construcci�n y emisi�n se�ales anal�gicas a partir de muestras previamente almacenadas, o del manejo de se�ales digitales de entrada o salida entre otras. A continuaci�n se resumen las caracter�sticas principales del m�dulo de adquisici�n anal�gico.

\begin{itemize}
	\item La frecuencia de muestreo m�xima es de 100 KS/s (cien mil muestras por segundo). Sin embargo esta velocidad de muestreo s�lo es alcanzable en caso de que se muestree con un s�lo canal y con una ganancia inferior a 10 V/V. Adem�s para poder mantener esta ganancia tan s�lo el m�dulo de adquisici�n anal�gica puede estar en funcionamiento.\par El dispositivo permite, como el lector habr� podido deducir, el muestreo con varios canales simult�neamente. En caso de que se muestree con m�s de un canal, la velocidad de muestreo en cada uno de ellos ser� el cociente entre la frecuencia m�xima y el n�mero de canales activos redondeado a la baja. Esto es cierto mientras la ganancia en todos los canales activos sea semejante y est� por debajo de los 10 V/V.
\end{itemize}
