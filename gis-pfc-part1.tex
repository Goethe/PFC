\chapter*{Introducción}

\section*{Organización de la primera parte}

Como se ha dicho, el presente documento consta de dos partes. El contenido
se ha distribuido en dos partes ordenadas como sigue: en primer lugar se
aborda la configuración del sistema digital de medida; para más tarde
tratar la teoría en la que se sustentan los \sig{endus} y presentar los
resultados obtenidos. Esta división del texto no es, sin embargo, del todo
natural, puesto que para llevar a cabo el proceso de diseño previo a la
implementación del sistema de medida es necesario haber reunido previamente
un determinado conocimiento en materia de \sig{endus}. Pese a todo, es
razonable considerar que la implementación del sistema de medida debe tener
lugar antes de la realización de las pruebas ---como ha ocurrido
efectivamente--- y de ahí la elección de este orden.

La primera parte se divide a su vez en cuatro capítulos: subsistema de
interacción con el medio físico; subsistema de adquisición; subsistema de
control y presentación; y resultados, conclusiones y líneas futuras de
trabajo. Los tres primeros capítulos responden a una división funcional del
sistema de medida que se explica con detalle en la introducción del primer
capítulo.


\begin{description}
	\item[Primer capítulo] El primer subsistema interactúa directamente
		con el medio físico, comprende el transmisor de
		ultrasonidos, el receptor y las etapas de acondicionamiento
		que los acompañan. El primer capítulo resume las
		principales características de estos tres elementos y
		realiza un recorrido por el proceso de diseño de este
		subsistema.
	\item[Segundo capítulo] El subsistema intermedio transforma la
		señal analógica que le es entregada por la sección de
		recepción en una señal digital.  El núcleo y el todo de
		este subsistema es la tarjeta de adquisición. El segundo
		capítulo repasa las características técnicas clave de este
		dispositivo, realiza una descripción funcional del mismo y
		reproduce los consejos que el fabricante proporciona en el
		manual de usuario para su uso correcto.
	\item[Tercer capítulo] El último de los subsistemas actúa como
		interfaz entre el supervisor y el sistema de medida. Es el
		subsistema de control y presentación. En el tercer capítulo
		se hace hincapié en el diseño conceptual de este subsistema
		y se proporcionan una serie de detalles técnicos en
		relación con el entorno de programación en el que se ha
		desarrollado.
	\item[Cuarto capítulo] En el último de los capítulos de la primera
		parte se exponen las conclusiones extraídas tras poner a
		prueba el sistema de medida ya terminado. Se comentan
		posteriormente líneas futuras de trabajo que persiguen
		mejorar la funcionalidad del sistema de medida.
\end{description}


\section*{El sistema digital de medida}

Un sistema de medida es cualquier instrumento formado por más de un
elemento que permite a quien lo usa evaluar una determinada propiedad de un
objeto, medio o evento. Para una definición más rigurosa de un sistema
electrónico de medida puede optarse por la encontrada en
\cite{pallas2004sas}, y dice lo siguiente:

{\small\begin{quotation}
	Se denomina sistema a la combinación de dos o más elementos,
	subconjuntos y partes necesarias para realizar una o varias
	funciones. En los sistemas de medida, esta función es la asignación
	objetiva y empírica de un número a una propiedad o cualidad de un
	objeto o evento, de tal forma que la describa.
\end{quotation}}

Un sistema de medida digital está constituido habitualmente por los
elementos mostrados en el esquema de la \cref{fig:digmeasstm}. Pueden
agruparse estos elementos según la función que desempeñan en el sistema, si
se hace de ese modo se encuentran tres subsistemas: un subsistema para la
interacción con el medio físico, un subsistema de adquisición, y un
subsistema de control y presentación. Puede observarse entonces el sistema
como una pila de capas superpuestas, en el que cada capa provee de servicio
a la capa inmediatamente superior y abstrae las capas inferiores, en cada
una de las capas se situaría cada uno de los subsistemas propuestos. De ese
modo cada subsistema puede estudiarse por separado con independencia de los
demás, así se ha hecho en esta memoria. Consecuentemente cada subsistema da
lugar a cada uno de los tres capítulos que empezando por este describen el
sistema digital de medida implementado durante el curso del desarrollo de
este proyecto.

\begin{figure}
	\begin{center}
		\includegraphics{gis-pfc-part1-01.mps}
	\end{center}
	\caption[Sistema digital de medida] {Distintos elementos
	funcionales que conforman el sistema digital de medida.}
	\label{fig:digmeasstm}
\end{figure}

Si se define una jerarquía en la que el nivel más bajo es aquel ocupado por
los elementos en contacto directo con el medio, y se tiene como nivel más
alto aquel en el que se encuentran los elementos que interactúan con el
supervisor, entonces los niveles ordenados de inferior a superior en la
jerarquía establecida se encuentran ocupados por los subsistemas precisados
anteriormente del siguiente modo: subsistema para la interacción con el
medio, subsistema de adquisición, y subsistema de control y presentación.
El orden en el que se han dispuesto los capítulos que tratan acerca de la
configuración del sistema digital de medida, los tres primeros capítulos,
coincide con el orden que guardan los distintos subsistemas en esta
jerarquía.
