\chapter{Prólogo}

% \subsubsection{Justificación}
%
% \subparagraph{Ensayos no destructivos con ultrasonidos}

Es en la práctica habitual emplear los \emph{ensayos no destructivos}
(\psig{end}) en controles de calidad efectuados en la industria de
manufactura de materiales, principalmente de metales y de compuestos para
la construcción. Este tipo de ensayos garantizan ---demostrada su
efectividad en este tipo de aplicaciones--- la ausencia de defectos
internos como fisuras y de otras imperfecciones como alteraciones puntuales
en la composición del producto, que de no superar controles semejantes
pueden pasar inadvertidos. La industria de materiales provee de materia
prima a otras industrias, la introducción en la cadena de producción de
partidas de material defectuoso repercute, sin duda, en una disminución en
la calidad del producto final. Cuando se trata de metales o conglomerados,
someter el material a \sig{end} con el fin de certificar su calidad es
crucial, se convierte en un ejercicio de responsabilidad, puesto que muchos
de estos materiales se destinan a la fabricación de medios de transporte
como barcos o aviones o se destinan a la construcción, aplicaciones en las
que el empleo de materiales defectuosos es una práctica inmoral que puede
poner en juego la seguridad de los individuos.

% >>|Fecha indeterminada|
%
%	Cuando se trata de metales o conglomerados, la realización de
%	\sig{end} para evaluar la calidad del material es crucial, es un
%	ejercicio de responsabilidad, puesto que muchos de estos materiales
%	se destinan a la fabricación de medios de transporte como barcos o
%	aviones o a la construcción, aplicaciones en las que está en juego
%	la seguridad de los individuos.
% <<<

Los ensayos no destructivos presentan una serie de ventajas frente a otro
tipo de ensayos. La primera a destacar es que permite una reducción en los
costes, lo que garantiza su viabilidad. Existen principalmente dos motivos
por los que los \sig{end} pueden ser más baratos que otros ensayos.
Obviamente, al no requerir que parte de la producción sea sacrificada en la
prueba se consigue una mayor producción en las mismas condiciones iniciales
con lo que el beneficio es mayor. Por otro lado, los controles de calidad
basados en \sig{end} pueden efectuarse de forma sistemática y automatizada
durante el mismo proceso de fabricación a medida que se fabrica el
producto. Dada la simultaneidad del proceso que evita p. e. la necesidad de
que el producto sea trasladado para su evaluación, se requiere menos tiempo
y se ahorra en costes de transporte y de personal. En cuanto a la calidad
del producto, a diferencia de otras pruebas en las que sólo se evalúa parte
del material, los ensayos no destructivos afectan a la totalidad de éste.
De ese modo es posible asegurar que las cualidades del producto que supera
los controles se mantienen uniformes a lo largo del mismo. Además, puesto
que los \sig{end} no alteran el material evaluado, es posible realizar
pruebas de este tipo en productos en los que se haya utilizado el material
como materia prima para su fabricación. Esto permite garantizar la calidad
de un producto con una confianza mucho mayor, ya que no sólo se controla la
calidad del material en múltiples ocasiones desde su fabricación, si no que
además se comprueba su estado cuando ya forma parte de otro producto,
después de sufrir las modificaciones propias de los procesos industriales
que lo llevan a adoptar su forma definitiva.

% >>|Fecha indeterminada|
%
%	¿De ese modo qué? Por un lado tengo todo en el mismo proceso, en la
%	misma máquina, en la misma cadena. No necesito otra cadena a la que
%	tenga que llevar después el producto para que me lo evalúen. Se
%	necesitan menos operarios, o máquinas, dado que el material sale ya
%	comprobado su buen estado, no tengo que trasladarlo a otra planta o
%	a otra cadena de maquinaria. La simultáneidad me evita la necesidad
%	de trasladar el material. El caso es que al hacerlo todo a la vez
%	me ahorro tiempo, y el tiempo es oro.
%
%	Desde el punto de vista de/en relación a/en cuanto a|la calidad del
%	producto A diferencia de otras pruebas en las que sólo se evalúa
%	parte del material, los ensayos no destructivos afectan a la
%	totalidad del producto, de modo que puede confiarse en que dicho
%	producto presenta unas cualidades uniformes, que se mantienen
%	invariables a lo largo del mismo.
%
%	Así el producto que supera los controles (se garantiza, no es que
%	sea verdad, pero uno asegura con buena fe) que el material es de
%	buena calidad, cualidades uniformes (más bien, que se mantienen
%	invariables uniformemente a lo largo del material | o que se
%	mantienen uniformes a lo largo de su superficie ---sí, en sus
%	profundidades, está trucado, ¿no?) de ese modo/de modo/de tal forma
%	| que
%
%	Confiable confiabilidad confiar confiarse
%
%	De modo que puede confiarse en que el material resultante/el
%	producto final presenta unas cualidades uniformes que se mantienen
%	por toda su superficie o volumen Desde el punto de vista de la
%	calidad del producto, al evaluarse el material de forma continuada
%	se consiguen materiales muy uniformes cuyas propiedades se
%	mantienen invariables en toda su superficie. Esto redunda en
%	materiales de gran calidad cuyas propiedades se mantiene
%	invariables uniformemente a lo largo del producto. Cabe destacar
%	que los controles de calidad basados en \sig{end} pueden efectuarse
%	durante el mismo proceso de fabricación de forma automatizada lo
%	que asimismo supone una reducción en los costes. Además, otra
%	ventaja de los \sig{end} es que es posible realizar de nuevo el
%	ensayo una vez el producto final se ha terminado o en posteriores
%	revisiones ya que no causa ningún daño en el material.
%
%	,en otras palabras, al ejercer un control exhaustivo (no es que sea
%	exhaustivo, es que se hace en el material a medida que se va
%	creando, por lo que lo cubre entero) puede garantizarse que el
%	producto es de gran calidad.
% <<<

Por su parte, el uso de ultrasonidos en \sig{end} está muy extendido, esto
es parcialmente debido a los precisos resultados que proporcionan los
\emph{ensayos no destructivos mediante ultrasonidos} (\psig{endus}).
Empleando transductores de alta frecuencia y gran ancho de banda es posible
distinguir defectos en el material de tamaño muy pequeño de forma
inequívoca. A esto debe sumársele las ventajas de emplear transductores de
tamaño relativamente pequeño como son los transductores de ultrasonidos.
Utilizando transductores de pequeño tamaño resulta fácil trabajar con
materiales que tras sufrir alguna modificación adoptan formas intrincadas,
lo que permite explotar al máximo las ventajas que ofrecen los \sig{end}.
Además los transductores de ultrasonidos reúnen una serie de ventajas
adicionales como su gran fiabilidad, su duración, su resistencia a las
condiciones externas y a un uso exigente, y son muy sencillos de
manejar.

Recientemente se están empleando los \sig{endus} en campos experimentales
con el propósito de detectar la presencia de anomalías en materiales
procedentes de la naturaleza. En estos experimentos el procedimiento a
seguir es conforme al uso de los controles de calidad que se practican en
el ámbito industrial. Conviene, sin embargo, diferenciar el tipo de medio
en el que los ensayos son aplicados. Los ensayos realizados habitualmente a
nivel industrial evalúan materiales sintéticos, bien conocidos, mientras
que en su lugar, los ensayos experimentales contemplan materiales orgánicos
heterogéneos como es en este caso la madera. Es por ello que la eficacia de
los \sig{endus} aplicados sobre materiales orgánicos no está todavía
demostrada. De la singularidad del material el interés de las pruebas,
interés que se ve reforzado por el hecho de que la madera es un medio
<<vivo>>, que cambia de muestra a muestra, las condiciones del ensayo
varían incluso para una misma muestra. Es, por tanto, el uso \sig{endus}
para la detección de defectos en madera de palmera un tema de interés, de
actualidad y muy atractivo, hacia el cual orientar el desarrollo de un
proyecto fin de carrera.


% \subparagraph{Tratamiento digital de señales}

El tratamiento digital de señales es una de las disciplinas fundamentales
que abarca la ingeniería de telecomunicaciones. Pese a los inconvenientes
inherentes al uso de circuitería digital (derivados del muestreo y de la
operación de cuantificación principalmente) el procesado digital de señales
ha demostrado ser una herramienta útil y versátil. Esa versatilidad se
sigue de la capacidad del hardware digital de ser programado, o lo que es
lo mismo la capacidad de un circuito de alterar su funcionamiento y
proporcionar varias funciones de acuerdo con una configuración por
software. Virtud que sumada al reciente desarrollo de las tecnologías
digitales ha propiciado su explotación y el uso extensivo de aplicaciones
basadas en circuitería digital. La configuración de un sistema digital de
medida a partir de una tarjeta de adquisición digital, un ordenador y una
suite de software matemático supone una oportunidad para comprender
aspectos del funcionamiento de los sistemas digitales difíciles de observar
en la teoría.


\subsubsection{Objetivos del proyecto fin de carrera}\label{sec:goals}

Inicialmente este proyecto persigue completar dos objetivos distintos: la
implementación de un sistema de medida a partir de una tarjeta de
adquisición con interfaz \sig{pci}; y la evaluación de los \sig{endus} como
método para la detección de defectos en madera de palmera.

Por un lado, se ha llevado a cabo la puesta en funcionamiento de un sistema
de medida con el que es posible obtener un conjunto de parámetros de una
determinada señal. Permite observar la forma que presenta la señal en cada
instante, la forma del espectro de la señal cuando se enmarca en una
ventana de una determinada duración temporal, valores instantáneos y
valores medios en un determinado intervalo. Por otro lado, se han llevado a
cabo una serie de pruebas experimentales en madera de palmera, empleando
para ello señales ultrasónicas generadas por un transductor, de forma
similar a como se hace en los \sig{endus} de carácter industrial.
Posteriormente las muestras se han procesado para averiguar a partir de los
datos que información sobre el medio ---presencia de fisuras internas,
presencia de agua en distintos niveles dependiendo de la altura en la que
se realiza la prueba o del estado de salud del espécimen, o en general
cualquier otra información de utilidad--- puede extraerse en este tipo de
ensayos.

Existe una relación visible entre ambos objetivos, la pretensión inicial
consiste en emplear el sistema de medida digital en la realización de las
pruebas. No obstante, pese a todo, durante el transcurso del proyecto ambos
objetivos se tratan por separado. Finalmente, la imposibilidad de realizar
las pruebas experimentales con el sistema confeccionado (por motivos que se
verán más adelante) ha conducido al uso de un sistema de medida
alternativo. Es por este motivo por el cual resulta comprensible la
división de este documento en dos partes, una por cada objetivo
diferenciado.


\subsubsection{Estructura del documento}

Como se ha dicho, el presente documento consta de dos partes. El contenido
se ha distribuido en dos partes ordenadas como sigue: en primer lugar se
aborda la configuración del sistema digital de medida; para más tarde
tratar la teoría en la que se sustentan los \sig{endus} y presentar los
resultados obtenidos. Esta división del texto no es, sin embargo, del todo
natural, puesto que para llevar a cabo el proceso de diseño previo a la
implementación del sistema de medida es necesario haber reunido previamente
un determinado conocimiento en materia de \sig{endus}. Pese a todo, es
razonable considerar que la implementación del sistema de medida debe tener
lugar antes de la realización de las pruebas ---como ha ocurrido
efectivamente--- y de ahí la elección de este orden.

La primera parte se divide a su vez en cuatro capítulos: subsistema de
interacción con el medio físico; subsistema de adquisición; subsistema de
control y presentación; y resultados, conclusiones y líneas futuras de
trabajo. Los tres primeros capítulos responden a una división funcional del
sistema de medida que se explica con detalle en la introducción del primer
capítulo.


\begin{description}
	\item[Primer capítulo] El primer subsistema interactúa directamente
		con el medio físico, comprende el transmisor de
		ultrasonidos, el receptor y las etapas de acondicionamiento
		que los acompañan. El primer capítulo resume las
		principales características de estos tres elementos y
		realiza un recorrido por el proceso de diseño de este
		subsistema.
	\item[Segundo capítulo] El subsistema intermedio transforma la
		señal analógica que le es entregada por la sección de
		recepción en una señal digital.  El núcleo y el todo de
		este subsistema es la tarjeta de adquisición. El segundo
		capítulo repasa las características técnicas clave de este
		dispositivo, realiza una descripción funcional del mismo y
		reproduce los consejos que el fabricante proporciona en el
		manual de usuario para su uso correcto.
	\item[Tercer capítulo] El último de los subsistemas actúa como
		interfaz entre el supervisor y el sistema de medida. Es el
		subsistema de control y presentación. En el tercer capítulo
		se hace hincapié en el diseño conceptual de este subsistema
		y se proporcionan una serie de detalles técnicos en
		relación con el entorno de programación en el que se ha
		desarrollado.
	\item[Cuarto capítulo] En el último de los capítulos de la primera
		parte se exponen las conclusiones extraídas tras poner a
		prueba el sistema de medida ya terminado. Se comentan
		posteriormente líneas futuras de trabajo que persiguen
		mejorar la funcionalidad del sistema de medida.
\end{description}

La segunda parte está dividida en dos capítulos: el quinto capítulo que
trata los fundamentos teóricos de los \sig{endus}; y el sexto en el que se
exponen los resultados extraídos de los ensayos, las conclusiones a las que
se ha llegado, así como las líneas de trabajo que se preven para futuros
proyectos relacionados con la materia.% , descripción del medio,
% resultados y conclusiones.


\begin{description}
	\item[Quinto capítulo] En el primer capítulo se describen de forma
		superficial los distintos elementos que intervienen en un
		\sig{endus} desde un punto de vista teórico. De las
		distintas técnicas existentes para combatir el ruido
		estructural se dan detalles sobre las técnicas de procesado
		por partición del espectro que son las utilizadas en este
		proyecto.
	% \item[Sexto capítulo] El sexto capítulo trata sobre el medio
	% en el que se realizan las pruebas experimentales, la madera de
	% palmera. Se proporciona una descripción teórica del material de
	% acuerdo con la documentación consultada. Se caracteriza el
	% material para un posterior análisis de los resultados encontrados
	% en las pruebas experimentales.
	\item[Sexto capítulo] El último capítulo recoge los resultados
		extraídos de las pruebas en forma de grafos y tablas
		comentados. Después se dan las conclusiones a las que se ha
		llegado a partir de estos resultados y finalmente se
		realizan una serie de comentarios sobre nuevas líneas de
		investigación que pueden seguirse de este proyecto
\end{description}

% >>|Fecha indeterminada|
%
%	De ese modo en estos tres capítulos se encuentra el siguiente contenido:
%	\begin{enumerate}
%		\item En primer lugar, el primer capítulo trata sobre el
%		subsistema que se encuentra en el estrato más bajo de la
%		jerarquía, el subsistema para la interacción con el medio.
%		Este subsistema comprende ---como se ha dicho en el
%		prólogo--- el sensor, el actuador y los circuitos
%		acondicionadores que los siguen. Estos son los elementos
%		fundamentales de un sistema electrónico de medida, con
%		ellos puede implementarse el sistema de medida más básico y
%		sin ellos no existen los sistemas de medida.
%		\item El segundo capítulo abarca el subsistema de
%		adquisición. Este subsistema es el encargado de digitalizar
%		la señal analógica que provee el subsistema para la
%		interacción en el medio y administrar la señal digital
%		resultante al bloque de presentación para que pueda
%		procesarla. Es un subsistema propio y distintivo de los
%		sistemas digitales de medida, ya que sólo esta presente en
%		este tipo de sistemas. En este caso se encuentra
%		constituido por un único elemento, la tarjeta de adquisición.
%		\item El último de los subsistemas, el subsistema de más
%		alto nivel, el subsistema de control y presentación, se
%		estudia en el tercer capítulo. Este subsistema interviene
%		como interfaz entre el usuario administrador y el resto de
%		capas (el sistema de medida). Como tal es su función
%		traducir la señal digital que recibe de capas inferiores en
%		información útil para el usuario, en información que éste
%		pueda interpretar. Por otro lado es su función también
%		gestionar el sistema de medida por medio de los comandos
%		que le envía el supervisor, en otras palabras, el
%		subsistema de control es el encargado de ejecutar las
%		órdenes del administrador del sistema.
%	\end{enumerate}
% <<<
