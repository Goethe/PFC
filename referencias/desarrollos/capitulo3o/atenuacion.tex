\documentclass[a4paper, 12pt]{article}
\usepackage[spanish]{babel}
\usepackage[latin1]{inputenc}
\usepackage{amsmath}
\title{Justificaci�n de las aproximaci�n del coeficiente de atenuaci�n en las distintas regiones}
\author{Jos� Ram�n Gisbert Valls}
\begin{document}

\maketitle{}

\begin{abstract}
	Para entender el por qu� de las diferentes aproximaciones que se proporcionan en la tesis de Miguel �ngel Garc�a Izquierdo y J. Rose.
\end{abstract}

En primer lugar debe sustituirse la longitud de onda por un factor $1/k$ que multiplica una longitud de onda inicial o b�sica $\lambda_0$, $\lambda=\lambda_0/k$. La longitud de onda inicial es una constante arbitraria. El di�metro del transductor puede darse en relaci�n con $\lambda_0$ y una segunda constante $l$, $D = \lambda_0/l$. Por tanto, y recordando $c = \lambda f$ y $\omega = 2\pi f$, para la regi�n de Rayleigh puede escribirse:

\begin{equation}
	\alpha_s(\omega) = s_1D^3\omega^4 = s_1\left(\frac{\lambda_0}{l}\right)^3\left(2\pi\frac{ck}{\lambda_0}\right)^4 = \frac{16\pi^4c^4s_1}{\lambda_0}\cdot\frac{k^4}{l^3} = \mathcal{C}\cdot\frac{k^4}{l^3}
	\label{eq:coeficiente-Rayleigh}
\end{equation}

Donde $\mathcal{C}$ es una constante que depende del medio y de la longitud de onda inicial elegida.\par
El factor $k$ aumenta con la frecuencia y, tal y como se ha mencionado en el primer p�rrafo, $l$ es una constante que a bajas frecuencias suele ser muy superior a $k$. Para aquellas frecuencias en las que $k\simeq l$ puede reescribirse \eqref{eq:coeficiente-Rayleigh} de forma que quede �nicamente en funci�n de $k$.

\begin{equation}
	\alpha_s(\omega) = \mathcal{C}\cdot m\cdot k\simeq\mathcal{C}\cdot k
	\label{eq:coeficiente-simplificado}
\end{equation}

Que como puede observarse es muy semejante a la forma propuesta para la regi�n de difusi�n. Sustituyendo $s_1$ por $s_2$ y $\mathcal{C}$ por $\mathcal{D}$ se tiene:

\begin{equation}
	\alpha_s(\omega) = s_2D\omega = 2\pi cs_2\frac{k}{l} = \mathcal{D}\cdot\frac{k}{l} = \mathcal{D}\cdot m\cdot k\simeq\mathcal{D}\cdot k
	\label{eq:coeficiente-estocastica}
\end{equation}

Para terminar, en la �ltima regi�n o regi�n de difusi�n $\lambda<D$, o lo que es lo mismo, $l<k$. En el l�mite el coeficiente de atenuaci�n ser�a infinito, sin embargo, se proporciona una constante.

\begin{equation}
	\alpha_s(\omega) = \frac{s_3}{D} = \frac{ls_3}{\lambda_0}
	\label{eq:coeficiente-difusion}
\end{equation}

Una suposici�n m�s, comparando las constantes $\mathcal{C}$ y $\mathcal{D}$ se llega a la siguiente conclusi�n.

\begin{equation}
	\mathcal{C} = \mathcal{D};\quad \frac{16\pi^4c^4}{\lambda_0}\,s_1 = 2\pi cs_2;\qquad\longrightarrow\qquad s_2 = s_1\,\frac{8\pi^3c^3}{\lambda_0}
	\label{eq:relacion-s1-s2}
\end{equation}

\end{document}


