\chapter{Subsistema de control y presentación}

\section{Introducción}

El último de los subsistemas que conforma el sistema de medida digital es el subsistema de control y presentación. Es el subsistema de más alto nivel de entre los que integran el sistema por interactuar directamente con el supervisor. El subsistema de control y presentación está constituido por los bloques de control y presentación propiamente dichos, tal y como muestra la \cref{fig:subconpre}. Sin embargo, en la práctica estos dos bloques se engloban en una misma entidad que en este documento se denomina \emph{software de control}.\par

\begin{figure}
	\begin{center}
		\includegraphics{gis-pfc-ch3-01.mps}
	\end{center}
	\caption[Subsistema de control y presentación]{Bloques de control y presentación dentro del subsistema de control y presentación.}
	\label{fig:subconpre}
\end{figure}

El subsistema de control y presentación, al actuar como interfaz entre el supervisor y el sistema de medida, debe ser capaz de poner en práctica las órdenes que administra el supervisor y gestionar el funcionamiento del resto de subsistemas. De entre los dos bloques que conforman el subsistema, es el bloque de control como su nombre indica el encargado de esta actividad. Como tal, es responsabilidad suya la ejecución de las siguientes tareas:

\begin{itemize}
	\item Iniciar y detener la sesión de adquisición.
	\item Interactuar con los drivers de la tarjeta por medio del sistema operativo para controlar parámetros de la sesión de adquisición como pueden ser, por ejemplo, puertos de entrada, modos de adquisición y terminación, o frecuencia de muestreo.
	\item Realizar el mantenimiento de los buffers de memoria. Tarea que puede dividirse, o consta a su vez, de dos tareas de menor complejidad: dar formato a las muestras almacenadas en el o los buffers situados en la memoria interna de la tarjeta para su comprensión por parte del ordenador anfitrión; trasladar las muestras formateadas adecuadamente a la memoria del \pc{} para que de ese modo puedan ser manipuladas por el usuario administrador.
\end{itemize}

Por su parte, al bloque de presentación le corresponde la tarea de presentar al supervisor información de utilidad a partir de los datos obtenidos del subsistema de adquisición, en este caso de la señal digital. La información debe mostrarse al supervisor de forma que este sea capaz de interpretarla. En consecuencia, y por criterio propio, se decide por diseñar el software de control para que sea capaz de proporcionar al supervisor la siguiente información de la señal:

\begin{itemize}
	\item El valor instantáneo cada 250 ms.
	\item El valor medio en el mismo intervalo de tiempo. Este valor debe calcularse a partir de las muestras obtenidas en dicho periodo a una frecuencia de muestreo que queda a elección del supervisor. Debe ser posible seleccionar la frecuencia de muestreo desde el propio software de control.
	\item La forma instantánea de la señal, de un fragmento de una duración determinada. Debe ser posible también seleccionar la duración del fragmento. Simultáneamente debe representarse el espectro en frecuencia del fragmento de señal que aparece en pantalla.
\end{itemize}

Se impone una condición de diseño adicional al software de control, debe ser poco exigente con respecto al \pc{} anfitrión en cuanto a capacidad de procesado y memoria principal utilizada. Para lograrlo se propone la siguiente solución: en lugar de mostrar al mismo tiempo el valor instantáneo de la señal, el valor medio y su forma, se concibe la aplicación para que el usuario pueda elegir uno de estos tres aspectos de la señal y sólo ese se muestra por pantalla.\par
Por conveniencia se adopta la siguiente nomenclatura en este documento: se dice que en el \emph{modo gráfico} la aplicación muestra gráficas que representan la forma de la señal y de su espectro en frecuencia; mientras que en los \emph{modos numéricos} o no gráficos lo que aparece en pantalla es un valor numérico relacionado con la magnitud variable de la señal ---el voltaje---. Aplicando de nuevo el criterio de austeridad por el cual la aplicación debe ser lo menos exigente posible, se consigue economizar recursos en el modo gráfico de funcionamiento si se permite al supervisor seleccionar si son ambos, forma y espectro de la señal, los representados, sólo uno y cual, o ninguno de ellos.


\subsection{Funcionamiento de un osciloscopio}

El objetivo primordial del software de control es implementar un modo gráfico con el que el supervisor pueda observar cual es el aspecto de la señal y como es su espectro en frecuencias. En el proceso previo de diseño se considera que modelo se puede tomar como referencia para implementar el modo gráfico. Por sus bondades, se elige el osciloscopio digital. En la representación de señales el instrumento empleado en la mayoría de aplicaciones es el osciloscopio digital. Los osciloscopios son instrumentos prácticos que permiten visualizar señales de forma efectiva.\par
Para tratar de imitar el funcionamiento del osciloscopio digital en el software de control, es preciso conocer como funciona este dispositivo y cuales son sus características principales. Como parte de este capítulo se ha querido proveer una introducción al comportamiento de los osciloscopios digitales que se extrae del estudio realizado sobre estos dispositivos al acometer este episodio del proyecto fin de carrera. Se expone esencialmente de que forma consigue el osciloscopio representar la señal, como lo hace. Para ello se documentan los distintos modos de representación de que cuenta el osciloscopio digital en comparación con su equivalente analógico. De ese modo el lector que acuda a esta introducción al comportamiento de los osciloscopios digitales puede formarse una imagen del aspecto que debe tener el software de control, lo que sirve a dos propósitos diferentes:

\begin{itemize}
	\item El primero y más inmediato, permitirle ubicar de forma muy precisa los objetivos del proceso de programación. De forma que le sea posible identificar, en cierto modo, que rutinas son necesarias para lograr el resultado final.
	\item Y además puede servirle como nota introductoria al \cref{chap:part1conclusions} en el que se comparan los resultados obtenidos en un experimento realizado primero mediante el sistema digital de medida y, posteriormente, utilizando un osciloscopio digital en conjunción con el subsistema para la interacción con el medio físico.
\end{itemize}


\subsection{Osciloscopios analógicos, osciloscopios digitales y retardo}

Los osciloscopios analógicos emplean un tubo de rayos catódicos y un monitor de fósforo para generar una imagen de la señal. El cursor que el tubo dibuja en el monitor lo va barriendo de izquierda a derecha con periodicidad, y su posición vertical refleja a cada momento el valor de tensión de la señal eléctrica que entra al osciloscopio. El monitor preserva durante breves instantes una traza del cursor y así se forma la imagen que representa a la señal.\par
En la actualidad los osciloscopios analógicos están siendo desplazados en su uso por los osciloscopios digitales. El procedimiento que sigue un osciloscopio digital para representar señales es completamente diferente al que siguen los osciloscopios analógicos. Constantemente el osciloscopio digitaliza la señal en seguimiento, procesa la señal digital resultante y la almacena en memoria. Cada cierto \emph{periodo de refresco} el osciloscopio muestra por pantalla una representación de los datos que previamente ha almacenado durante ese mismo periodo. Mientras se recogen datos para una nueva imagen se mantienen en pantalla los datos correspondientes al periodo anterior. La velocidad con la que se suceden las imágenes en el monitor del osciloscopio hace que se produzca la sensación de movimiento, y de ese modo la representación responde a las variaciones que se producen en la señal. Esto permite observar los eventos que se producen en la señal en determinadas situaciones como por ejemplo, al accionar un potenciómetro que forma parte de un circuito que está siendo analizado, al cambiar la configuración de un generador de señales para que genere una señal diferente a la que suministraba unos instantes antes, o simplemente al conectar la alimentación.\par
El periodo de refresco no es fijo, de lo contrario los datos mostrados en el monitor del osciloscopio aparecerían en exceso comprimidos o expandidos en función de la frecuencia de la señal monitorizada. Puede configurarse el periodo de muestreo para que adopte un valor de entre un rango de valores predeterminados. 
% LO IMPORTANTE AQUÍ ES EXPLICAR POR QUÉ LOS DATOS SALEN COMPRIMIDOS O EXPANDIDOS SI EL PERIODO DE REFRESCO ES FIJO

La posibilidad de ajustar el periodo de muestreo presenta un inconveniente relacionado con el modo de funcionamiento de los osciloscopios digitales.

Un periodo de refresco demasiado alto puede ser la causa de un retardo excesivo en la sucesión de imágenes, haciendo que la representación de la señal reaccione con lentitud a las variaciones que en ésta última se produzcan. Para impedir que ésto ocurra los osciloscopios digitales implementan un mecanismo por el cual alteran el modo en que una señal se representa en función del valor que toma el periodo de refresco.

% No existe ningún cursor, se generan fotogramas de la señal en seguimiento\footnote{En realidad los osciloscopios digitales modernos disponen de más de un canal de entrada y por tanto pueden representar por pantalla varias señales simultáneamente.}, imágenes que muestran la representación de un fragmento de la señal y que cubren la pantalla del osciloscopio por completo. Una vez aparece una imagen por pantalla permanece estática hasta que una nueva imagen la sustituye. Al sucederse las imágenes rápidamente en la pantalla del osciloscopio se produce una sensación de movimiento, y la representación responde a las variaciones que se producen en la señal.\par
% El periodo de tiempo que transcurre entre dos imágenes puede identificarse como \emph{periodo de refresco}. El periodo de refresco está estrechamente relacionado con la duración temporal del fragmento de señal que se representa por pantalla. La señal monitorizada se desconoce de antemano, por tanto, para representar un fragmento de una duración determinada el osciloscopio debe esperar hasta reunir la suficiente información. Esto no ocurre con los osciloscopios analógicos puesto que éstos representan la señal en tiempo real.
% Como en cualquier osciloscopio analógico un usuario debería ser capaz de configurar un osciloscopio digital para que represente los datos correspondientes a un periodo de tiempo más o menos largo. De lo contrario, supondría una gran dificultad para el seguimiento de señales cuya frecuencia difiriese en gran medida con el periodo de refresco.
% Los controles del osciloscopio digital deben permitir al usuario del osciloscopio ajustar el periodo de refresco.
% En ocasiones es preciso.
% Para poder trabajar con señales de frecuencias altas o bajas que difieran en gran medida de la frecuencia con la que se refresca el monitor del osciloscopio es preciso poder modificar el periodo de refresco. 
% Para poder trabajar correctamente con señales cuya frecuencia difiera en gran medida de la tasa de refresco del monitor del osciloscopio es preciso.
% Que exista la posibilidad/poder modificar/variar/alterar dicha tasa/el periodo de refresco.
% Debe poder modificarse.
% Si el periodo de refresco toma un valor muy alto y nada lo impide el retardo que se produce entre .
% Puede producir/es causa  que ocurre en el monitor del osciloscopio. Para impedir que ésto ocurra/este fenómeno ocurra/se produzca los osciloscopios digitales implementan un mecanismo.
% Por el cual alteran el modo en que una señal se representa en función del valor que toma el periodo de refresco.
% Cada dos imágenes/cada imagen.
% Por suerte. Implementa un mecanismo.
% Para poder trabajar con señales de frecuencias altas o bajas que difieran en gran medida de la frecuencia con la que se refresca el monitor del osciloscopio debe ser posible modificar el periodo.
% El aspecto que muestra la pantalla de un osciloscopio digital en un instante determinado se.
% El resultado es una imagen. La imagen que aparece en pantalla.
% Para resolver los problemas provocados por señales.
% Para evitar algunos de los problemas que en los osciloscopios analógicos provocaban las señales lentas o las señales rápidas (mentira, el problema es barrer la pantalla con el cursor y el osciloscopio ya lo resuelve per se).
% Dado que el usuario puede modificar la duración de los fragmentos de señal que aparecen por pantalla podría producirse una situación en la que la duración del fragmento es demasiado prolongada el retardo que se produce entre dos imágenes conduce a que el desfase entre la señal y la información representada en pantalla sea notable.
% El usuario puede modificar la duración del eje temporal de la representación --> luego, si la representación ocupa por completo el monitor del osciloscopio, lo que cambia es la duración de los fragmentos. Voy a clarificar la relación anterior, el eje de tiempos es siempre de la misma longitud, sin embargo no siempre representa la misma cantidad de tiempo. El usuario puede hacer que represente un tiempo mayor o un tiempo menor. Si representa un tiempo mayor entonces el fragmento de señal necesario para cubrir el eje de tiempos es mayor y se necesita más tiempo para adquirirlo. El objeto de que pueda modificar la duración del eje temporal es que pueda observar detalles que no se pueden apreciar de otra manera, o seguir señales de menor frecuencia.
% Dado que la duración de los fragmentos puede seleccionarse. Puede/Podría producirse una situación. Esta situación es inconveniente.
% Pese a las mejoras que los osciloscopios han experimentado desde su aparición, como por ejemplo la capacidad de los osciloscopios digitales de medir ciertos parámetros de una señal como su periodo o su valor de pico a pico, el propósito original de estos dispositivos es el de mostrar el aspecto de una señal de forma que pueda extraerse de éste cualquier información que pudiera deducirse de él. % Pueda extraerse cualquier información de la señal que pudiese deducirse a partir de su apariencia.
% En este sentido un osciloscopio debe mostrar una imagen de la señal actual y permitir al usuario modificar aspectos de la representación como la escala con la que se representan las tensiones.
% La dimensión temporal de la representación determina cuanto debe esperar el osciloscopio desde que se muestra por pantalla una imagen hasta que se puede generar la siguiente. El retardo que se produce es la consecuencia de representar de una vez un fragmento de una duración determinada de una señal que se desconoce de antemano. Al contrario que ocurre con los osciloscopios analógicos que representan la señal en tiempo real, los osciloscopios digitales deben esperar hasta reunir la suficiente información de la señal. En otras palabras, aparece un retardo que depende exclusivamente de la duración del fragmento de señal necesario para cubrir la ventana del osciloscopio, o lo que es lo mismo, el retardo es independiente de otros parámetros como son la velocidad de muestreo o la potencia de procesado del osciloscopio.\par
% Este modo de proceder tiene un inconveniente, el objetivo de un osciloscopio es mostrar en cada momento como es la forma de la señal, si el retardo introducido es demasiado alto el osciloscopio deja de ser eficaz pues la información que proporciona cada imagen es obsoleta. Problema que se ve agravado por la posibilidad de configurar el eje de tiempos. Tanto los osciloscopios analógicos (variando la frecuencia de barrido del cursor) como los osciloscopios digitales, permiten que el usuario modifique la cantidad de tiempo que refleja el eje horizontal de la representación, pudiendo ser ésta mayor o menor. A efectos prácticos, en los osciloscopios digitales el tiempo abarcado por el eje temporal actúa como el inverso de la frecuencia de refresco del monitor, por tanto, cuanto mayor sea ese tiempo con mayor lentitud se sucederán las imágenes y habrá un mayor desfase entre la señal y la representación de ésta.


\subsection{Modos de representación en los osciloscopios digitales}\label{subsec:repmodes}

% Los osciloscopios digitales tienen la capacidad de representar una señal de dos formas diferentes: una en la que la señal aparece anclada a la ventana del osciloscopio, que en este documento se ha asociado al nombre \emph{modo disparado}; y otra en la que la señal se desplaza de derecha a izquierda en la pantalla del osciloscopio, que aquí se conoce como \emph{modo continuo}.
Los osciloscopios digitales tienen la capacidad de representar una señal de formas diferentes. Cada forma de representar la señal se conoce en este documento como \emph{modo de representación}. Habitualmente los osciloscopios digitales cuentan al menos con dos modos de representación. Estos dos modos son, tal y como han sido bautizados aquí, el \emph{modo disparado} y el \emph{modo continuo}. Los modos de representación continuo y disparado forman parte de la solución que los osciloscopios digitales adoptan para combatir las situaciones en las que un hipotético retardo entre cuadros podría complicar el seguimiento de los eventos producidos en la señal. Para remediar el retardo, un osciloscopio digital conmuta al modo de representación continuo cuando el periodo de refresco supera los 200 ms, el resto del tiempo permanece en el modo de representación disparado.
para combatir el retardo que se produce en configuraciones/situaciones en las que el periodo de refresco es muy alto

La elección del límite/umbral en el que el osciloscopio conmuta entre uno u otro modo no es trivial. Este límite responde a un compromiso 
por el cual se consigue representar señales de frecuencia tal sin que el retardo suba.

Este límite responde a un compromiso 
que por un lado consigue que puedan seguirse 
Dada la peculiaridad con la que se representa una señal en el modo continuo, la representación en este modo no se ve afectada por el periodo de refresco.
complacer

% Aquí me sale la complicación de explicar que en el modo continuo aunque el eje de tiempos cubre un determinado intervalo, este no se corresponde con el periodo de refresco. Pero esto podría reservarlo para el punto en el que trato el modo de representación continuo. El modo de representación disparado debería explicarse al principio, ya que en este modo la señal se representa como lo hacen los osciloscopios analógicos y, además, todo lo que se ha dicho sobre como representa la señal un osciloscopio digital hace referencia a este modo. El modo disparado es el que normalmente está activado, por eso yo lo había llamado modo convencional, aparte de lo que he dicho en este párrafo sobre él.
% En el modo de representación disparado los datos se representan de modo que cubren la extensión horizontal de la ventana del osciloscopio. El eje de tiempos representa pues una medida similar al periodo de refresco. Después añado que el periodo de refresco es igual al tiempo que necesito para adquirir muestras suficientes para cubrir la ventana, para adquirir muestras adicionales para corregir la posición de la señal y que no aparezcan huecos, y para procesar, almacenar y representar las muestras. En el modo de representación continuo el eje de tiempos de la ventana del osciloscopio representa un periodo de tiempo que siempre es mayor que el periodo de refresco y en ocasiones mucho mayor. De ese modo cuando la imagen se refresca no desaparecen todos los datos si no que se van desplazando.

% La necesidad de disponer de dos modos de representación surge como parte de una solución al problema
% expuesto en el punto anterior/ocasionado por.
% Para evitar los altos retardos que se producen en configuraciones en las que el periodo de refresco es muy bajo. Los altos retardos que se producen en configuraciones del periodo de refresco muy bajo. En configuraciones en las que el periodo de refresco es muy bajo.
% Y surgen como parte de una solución al problema expuesto en el punto anterior ocasionado por.



\subsubsection{Criterios para la selección del modo}

% Existen dos modos de representación. Por lo general se toma como criterio utilizar el modo de representación disparado cuando el periodo de refresco es inferior a los 400 ms. Se toma como criterio, ¿qué es un criterio? Criterio = Discernimiento = Acción y efecto de discernir; Discernir = Distinguir algo de otra cosa, señalando las diferencias que hay entre ellas.
% Es decir, se toma criterio para discernir entre las situaciones en las que es conveniente representar la señal de uno u otro modo.
Para evitar que en determinadas configuraciones del eje temporal del osciloscopio el retardo sea excesivo, la mayoría de estos dispositivos implementan dos modos de funcionamiento: el modo convencional que se aplica en situaciones en las que el retardo no se considera importante; y una especie de modo continuo. Existen dos criterios que se siguen de manera habitual para diferenciar en que momento es más apropiado el uso de uno u otro modo.


\subparagraph{Criterio de animación}

El primer criterio evalúa el correcto funcionamiento del modo convencional. El modo convencional de representación en un osciloscopio requiere que la cantidad de imágenes que salen por pantalla cada segundo sea lo suficiente grande como para que se simule el movimiento. Por tanto para satisfacer el primer criterio, la tasa de refresco del monitor del osciloscopio debe superar las veintiséis imágenes por segundo o encontrarse alrededor de esta cifra.


\subparagraph{Criterio de disparo}

El segundo criterio está relacionado con la función de disparo que se da en el modo convencional de representación. Para poder efectuar correctamente el disparo sobre señales de baja frecuencia, al generar cada imagen el osciloscopio debe haber registrado al menos un ciclo de la señal pertinente\footnote{En la práctica se necesita algo más de un ciclo de una señal para poder garantizar el correcto disparo de ésta. Sin embargo, es común conseguir disparar una señal aunque la configuración del eje de tiempos del osciloscopio permita sólo obtener una fracción del ciclo completo de esa señal durante la realización de cada imagen.}. Lo cual reduce la tasa mínima de refresco a la frecuencia de la señal más lenta que se desee representar de forma correcta en el modo convencional de representación del osciloscopio.\par

Como puede verse el segundo criterio es más restrictivo pues depende de la frecuencia de la señal y ésta puede ser en efecto inferior a un hercio, es por ello que se emplea habitualmente. Siendo así, la tasa de refresco mínima que se permite en el modo convencional en la gran mayoría de dispositivos y que se ha adoptado para el software de control, es de cinco imágenes por segundo (5 Hz). Cuando se configura un osciloscopio en el modo convencional para que trabaje a esta tasa de refresco es habitual poder visualizar señales de frecuencia cercana a 1 Hz, sin embargo la representación se optimiza para configuraciones en las que el osciloscopio muestra, al menos, veinticinco imágenes por segundo, lo cual es apropiado para señales con una frecuencia mínima de 50 Hz. Si la configuración del eje de tiempos del osciloscopio obligara al dispositivo a trabajar con una frecuencia de refresco inferior a la especificada, automáticamente conmuta para funcionar en el modo continuo.\par


\subsubsection{Modos de representación}

Una vez visto cual es la frecuencia umbral a la que el dispositivo conmuta entre los dos modos, debe explicarse que diferencia un modo de funcionamiento de otro. Para ello se expone a continuación cuales son los fundamentos de uno y otro modo.


\subparagraph{Modo continuo}

La representación en \emph{modo continuo} es, por decirlo así, ininterrumpida. La imagen de la señal de interés se desplaza de derecha izquierda a medida que transcurre el tiempo. Para lograr este efecto se aumenta la frecuencia de refresco a expensas de que, como es sabido, el fragmento de señal obtenido para cada imagen no puede cubrir la pantalla al completo. Ocurre así por que la frecuencia de refresco es muy alta para el eje temporal, que en este caso representa un periodo tiempo relativamente alto. Por consiguiente, en el intervalo en el que una imagen de la señal da paso a la siguiente en la ventana del osciloscopio, no transcurre el tiempo suficiente como para captar un fragmento de señal de duración igual a la abarcada por el eje de tiempos.\par
Las figuras \cref{fig:freesignalcont,fig:modconti} muestran con detalle estos aspectos del modo de representación continuo. En la \cref{fig:freesignalcont} aparece la señal que entra en el osciloscopio, se muestran además unas marcas de tiempo que indican los instantes en los que la imagen que muestra el osciloscopio cambia\footnote{Se ha considerado el \emph{periodo de refresco} de la ventana del osciloscopio igual a la suma de la \emph{ventana de adquisición} y un \emph{periodo de inactividad} en la adquisición ocupado por el procesado de las muestras. La ventana de adquisición es equivalente a la cantidad de tiempo que el sistema necesita para adquirir un fragmento de señal suficiente para cubrir una porción del eje de tiempos (en el caso del modo de funcionamiento continuo) o para cubrir el eje de tiempos por completo más las muestras de descarte (en el modo disparado). El periodo de inactividad es igual al tiempo requerido para procesar el fragmento de señal adquirido en el espacio de una ventana de adquisición.}. La \cref{fig:modconti} muestra cual es el aspecto del osciloscopio en dos instantes posteriores al instante inicial en el que empieza a monitorizarse la señal.

\begin{figure}
	\begin{center}
		\includegraphics{gis-pfc-ch3-04.mps}
	\end{center}
	\caption[Fragmentos de señal ordenados según el orden en el que llegan al sistema]{Fragmentos de señal ordenados según el orden en el que llegan al sistema. Cada instante marcado se encuentra separado de los adyacentes por un periodo de refresco. Debe recordarse que el modo de funcionamiento continuo se emplea en la monitorización de señales lentas, por lo que el periodo de inactividad es en este caso despreciable frente a la duración de la ventana de adquisición.}
	\label{fig:freesignalcont}
\end{figure}

\begin{figure}
	\begin{center}
		\includegraphics{gis-pfc-ch3-05.mps}
	\end{center}
	\caption[Dos líneas de prueba]{Dos líneas de texto de prueba para comprobar como cambia la estructura del documento si lo comprimo un poquito.}
	\caption[<++>]<++>{}<++>
% 	\caption[Modo de funcionamiento continuo]{Puede observarse como la llegada de un nuevo fragmento de señal al sistema en el instante $t_2$ desplaza el fragmento previamente representado en la ventana del osciloscopio una periodo de refresco a la derecha.}
	\label{fig:modconti}
\end{figure}

En el instante $t_1$ el fragmento de señal que se ha digitalizado aparece en el extremo derecho de la representación, el resto se deja en blanco. En cada nuevo fotograma se desplazan los fragmentos de señal representados hasta el momento el suficiente espacio hacia la izquierda como para incorporar un nuevo fragmento de señal. Así la porción de señal representada aumenta cada vez más hasta que el extremo izquierdo de la figura alcanza el margen izquierdo de la ventana del osciloscopio. Cuando esto ocurre, empieza un desplazamiento cíclico, al incorporar un nuevo fragmento de señal a la derecha se retira un fragmento de la misma proporción temporal al otro extremo. Este fenómeno bien puede observarse en la \cref{fig:modcontii}, en el ejemplo el eje temporal es equivalente a cinco ventanas de adquisición, la llegada de un sexto fragmento de señal al sistema conduce al inicio de la representación cíclica.\par

\begin{figure}
	\begin{center}
		\includegraphics{gis-pfc-ch3-06.mps}
	\end{center}
	\caption[Dos líneas de prueba]{Dos líneas de texto de prueba para comprobar como cambia la estructura del documento si lo comprimo un poquito.}
% 	\caption[Modo de funcionamiento continuo]{La ventana temporal tiene, en este ejemplo, una duración de cinco ventanas de adquisición. Al llegar al sistema el sexto fragmento de señal, la representación avanza a la derecha retirándose el primer fragmento representado para dar cabida al fragmento obtenido recientemente.}
	\label{fig:modcontii}
\end{figure}

La alta tasa con la que aparecen las imágenes por pantalla garantizan que la representación siga casi en tiempo real ---por lo menos así lo percibe el ojo humano--- a la señal verdadera. Debe notarse, no obstante, que este método de representación no es adecuado para señales de alta frecuencia pues aunque el eje temporal abarque un tiempo mayor, la dimensión del monitor obviamente no varía y las señales de alta frecuencia aparecerán en exceso comprimidas.\par


\subparagraph{Modo convencional}

El \emph{modo convencional} de representación es algo más complicado. La representación convencional basa su funcionamiento en imágenes que muestran fragmentos de señal que cubren la dimensión temporal de la ventana del osciloscopio y que se suceden unas a otras con presteza. La idea que persigue este método es la de conseguir algo semejante a una película de la señal. Para ello, no es sólo suficiente con que aparezcan muchas imágenes por segundo en el monitor del osciloscopio, esas imágenes deben estar de algún modo relacionadas entre sí. Si las imágenes que aparecen en el monitor son de forma consecutiva muy diferentes es probable que no pueda discernirse nada claro, y en ese caso de nada vale la representación.\par
Ahí es donde entra en juego el procesado y la función de disparo de un osciloscopio digital. Cuando un fragmento suficientemente largo como para cubrir la ventana de representación se digitaliza y se almacena en memoria, empieza el procesado digital del mismo. El propósito del procesado es, ente otras cosas, eliminar las posible componente en continua, averiguar información adicional de la señal en la medida de lo posible, como p. ej. su valor de pico a pico, o implementar la función de disparo. La función de disparo del osciloscopio persigue alinear los ejes verticales de la ventana donde se representa con los cruces de la señal con respecto a un determinado valor de umbral que puede ser configurado por el usuario. La idea es alinear al menos el cruce más centrado de cada fragmento de señal con el eje de abscisas que corta en dos mitades la ventana. Para conseguirlo, durante el procesado se detectan todos los cortes de la señal con el umbral y después se desplaza el fragmento de señal para que el corte más centrado case con el eje de abscisas. Si la señal es periódica las variaciones entre un fotograma y el siguiente serán mínimas, puesto que ambos estarán alineados, y se simulará con éxito el movimiento.\par
No obstante, la necesidad de desplazar el fragmento de señal que va a representarse presenta un inconveniente importante. Para poder desplazar el fragmento de señal y que se cubra completamente la ventana del osciloscopio su duración debe ser superior al tiempo reflejado en el eje de tiempos de la ventana. De lo contrario la señal aparecerá truncada, se mostrará un espacio en blanco y con una tasa de refresco alta la visualización será confusa. Desplazar fragmentos de señal implica dos cosas: por un lado el osciloscopio debe esperar más tiempo para refrescar la pantalla, es decir, se reduce la tasa de refresco para una misma configuración del eje de tiempos; y por otro, al tener para representar un fragmento de señal de mayor duración que la reflejada por el eje temporal de la ventana parte del fragmento queda ahora fuera de la representación.\par
Este inconveniente se ve paliado por el hecho de que el modo convencional contempla en principio trabajar con señales de alta frecuencia. Estas señales cambian de forma tan rápida que de no ser por el disparo sería imposible observar los cambios. Por otro lado puede seleccionarse que información que se muestra por pantalla accionando un control de desplazamiento temporal o modificando el valor de umbral de disparo y, generalmente, la pérdida de información a causa del disparo no es significativa.\par
Para terminar este apartado se han adjuntado las \cref{fig:freesignaltrig,fig:modtrig} que muestran los distintos aspectos del modo disparado de funcionamiento del osciloscopio que se han expuesto aquí. La \cref{fig:freesignaltrig} muestra la señal que entra al sistema, se han marcado los instantes en los que se suceden los eventos más relevantes en la representación de un fragmento de señal. También se muestran el nivel de umbral y la porción de fragmento de señal que coincide con las muestras descartadas. Otra figura, la \cref{fig:modtrig} muestra cual es el resultado de representar el fragmento de señal en la ventana del osciloscopio.

\begin{figure}
	\begin{center}
		\includegraphics{gis-pfc-ch3-02.mps}
	\end{center}
	\caption[Fragmento de señal adquirido por el osciloscopio en el espacio de una ventana de adquisición]{Fragmento de señal adquirido por el osciloscopio en el espacio de una ventana de adquisición. En esta figura se ha representado también la duración de la ventana temporal del osciloscopio, pudiendo observarse que información se descarta. También se indica el instante en el que se inicia una nueva instancia del proceso de adquisición, de lo que puede deducirse que información se pierde entre instancias. Los cortes de la señal con el nivel de disparo también están representados.}
	\label{fig:freesignaltrig}
\end{figure}

\begin{figure}
	\begin{center}
		\includegraphics{gis-pfc-ch3-03.mps}
	\end{center}
	\caption[Modo de funcionamiento disparado]{Modo de funcionamiento disparado. La representación se centra en el corte central (veáse la \vref{fig:freesignaltrig}).}
	\label{fig:modtrig}
\end{figure}


\section{Elaboración del software de control}


\subsection{Elección del entorno de desarrollo}\label{subsec:environment}

El lenguaje o, mejor dicho, plataforma que se ha empleado para el desarrollo del software de control es \matlab{}, la razón, su inmejorable compatibilidad con el hardware disponible. A partir de ahí, los componentes de \matlab{} que se han empleado para conseguir que el software de control gozase de las características previstas en la etapa de diseño son dos: el entorno de desarrollo de interfaces gráficas de usuario (\emph{Graphical user interface}, en lo sucesivo \psig{gui}) de \matlab{}, más conocido como \psig{guide} (\emph{Graphical user interface development environment}); y la herramienta de adquisición de datos de \matlab{} (\emph{Data acquisition toolbox}, en adelante \psig{DAT}) . El primero de ellos se ha empleado para, como su nombre indica, crear la interfaz que comunica el dispositivo con el usuario. Esta comunicación se hace a través del segundo de los componentes mencionados, éste permite, mediante comandos de \matlab{} que pueden incluirse en rutinas o llamarse por separado, manejar el dispositivo ---convocarlo a muestrear, configurar sus propiedades--- y administra automáticamente los resultados almacenándolos en búffers situados en la memoria volátil del ordenador, haciéndolos de este modo accesibles al administrador de la tarjeta.


\subsection{Data Acquisition Toolbox}

La práctica de emplear \gui{} como fondo para aplicaciones desarrolladas con \matlab{} es bastante habitual, así como lo es programar esas interfaces mediante \guide{}. Por ello, y dado que la documentación que se ciñe a tratar la problemática que envuelve esta actividad es abundante, se ha preferido dejar a cargo de dichos documentos este asunto y centrar este escrito en presentar con concisión los principios necesarios para emplear con éxito la \datx{} de \matlab{} en el manejo de dispositivos para la adquisición de señales analógicas. La fórmula elegida con tal propósito consiste en describir cual es el procedimiento habitual en una sesión de adquisición de datos y en cada paso detallar las opciones más significativas y proporcionar ejemplos explicativos extraídos del mismo código que integra el software de control.\par
En cuanto a la documentación adicional que el lector puede consultar a continuación se citan varios documentos clasificados en función del tema que tratan. Para la programación de \gui{} con \matlab{} puede consultarse el manual de usuario, cite, disponible en la web del fabricante. Esta sección se basa en la guía rápida sobre el uso de la \datx{}, también a cargo de \emph{The MathWorks, Inc.}, la compañía que mantiene la suite matemática en cuya web puede obtenerse un documento extendido.


\subsubsection{Componentes de la herramienta}

Los elementos de \matlab{} que juegan un papel suficientemente importante en el funcionamiento de la \datx{} son los listados en el \cref{tab:toolcomp}. El diagrama representado en la \vref{fig:toolcomp} muestra las interdependencias que existen entre los elementos que aparecen en dicho cuadro.

\begin{table}
	\centering
	\begin{tabulary}{.9\textwidth}{C L}
		\toprule
		Componente & \multicolumn{1}{c}{Propósito} \\
		\midrule
		Ficheros *.m & Se emplean para automatizar la creación de objetos dispositivo, adquirir datos, configurar las propiedades del dispositivo y la sesión, y evaluar el estado de la adquisición y los recursos.\\
		\midrule
		Máquina virtual de adquisición de datos & Almacena objetos dispositivo y sus propiedades, controla el almacenamiento de los datos adquiridos y controla la sincronización de eventos.\\
		\midrule
		Adaptadores & Son la vía de comunicación entre la máquina virtual de adquisición de datos y el hardware por la cual se transmiten propiedades, datos y eventos.\\
		\bottomrule
	\end{tabulary}
	\caption[Descripción de los componentes de la \datx{}]{Descripción de los componentes de la \datx{}.}
	\label{tab:toolcomp}
\end{table}

\begin{figure}
	\begin{center}
		\includegraphics{gis-pfc-ch3-07.mps}
	\end{center}
	\caption[Elementos que intervienen en el funcionamiento de la \datx{}]{Elementos que intervienen en el funcionamiento de la \datx{}.}
	\label{fig:toolcomp}
\end{figure}

\subsubsection{Objetos dispositivo}

Los objetos dispositivo permiten el acceso a subsistemas específicos del hardware. Los objetos dispositivo soportados por la \datx{} son los objetos de entrada analógica o \emph{analog imput objects} (\psig{ai}), los objetos de salida analógica o \emph{analog output objects} (\psig{ao}) y los objetos de entrada/salida digital o \emph{digital I/O objects} (\psig{dio}).

\begin{figure}
	\begin{center}
		\includegraphics{gis-pfc-ch3-08.mps}
	\end{center}
	\caption[Comunicación entre los subsistemas del hardware y los objetos dispositivo]{Grafo que representa la comunicación entre los subsistemas del hardware y los objetos dispositivo.}
	\label{fig:subsystemsOO}
\end{figure}


\subsection{La sesión de adquisición de datos}

Una sesión completa de adquisición de datos consiste en cinco pasos:

\begin{enumerate}
	\item Crear el objeto dispositivo.
	\item Añadir canales al objeto dispositivo.
	\item Configurar las propiedades del objeto dispositivo y los canales añadidos para controlar el comportamiento de la aplicación de adquisición de datos.
	\item Adquirir los datos.
	\item Eliminar el objeto dispositivo.
\end{enumerate}

Cada uno de los pasos se detalla en los puntos subsiguientes.


\subsubsection{Crear el objeto dispositivo}

Para crear un objeto dispositivo, se debe llamar a la función de creación apropiada o constructor. Como se muestra en el \cref{tab:constructors}, los constructores reciben un nombre particular en función del tipo de objetos dispositivo que crean. Para iniciar una sesión de adquisición de datos analógicos es necesario un comando como el siguiente, \func{analoginput(`adaptador', ID)}. Un ejemplo extraído del código fuente de la aplicación de control muestra cómo hacerlo en el \cref{cod:constructor}.\par

\begin{table}
	\centering
	\begin{tabular}{l >{\tt}l}
		\toprule
		\multicolumn{1}{c}{Tipo de subsistema} & \multicolumn{1}{c}{\rm Constructor} \\
		\midrule
		Entrada analógica & analoginput(`adaptador', ID); \\
		\midrule
		Salida analógica & analogoutput(`adaptador', ID); \\
		\midrule
		Entrada / Salida digital & digitalio(`adaptador', ID); \\
		\bottomrule
	\end{tabular}
	\caption[Tipos de constructor en función del objeto dispositivo creado]{Tipos de función de creación de acuerdo con el tipo subsistema al que se orienta el objeto dispositivo creado.}
	\label{tab:constructors}
\end{table}

\begin{lstlisting}[style=displayed, caption={[Método a seguir para crear un objeto dispositivo]{Método que evalúa la existencia de un objeto dispositivo previo a la llamada de la aplicación, en caso positivo lo hereda para su uso posterior, de lo contrario crea uno nuevo.}}, label={cod:constructor}]
	set(handles.ai, 'TriggerType', 'Immediate', 'TimerFcn', '', ...
	handles.ai = [];

	if ~isempty(daqfind)
		oldObj = daqfind;

		for i = 1:length(oldObj);
			auxStr = daqhwinfo(oldObj(i));
			auxNum = findstr(' ', auxStr.DeviceName) - 1;
			if strcmp('KPCI-3108', ...
				auxStr.DeviceName(1:auxNum)) && ...
				strcmpi('analoginput', auxStr.SubsystemType);
				handles.ai = oldObj(i);
				warning('El dispositivo esta en uso.');
				break
			end
		end

	end

	if isempty(handles.ai)
		try
			handles.ai = analoginput('keithley');
		catch
			errordlg('No pudo crearse el manejador de dispositivo.');
		end
	end
\end{lstlisting}

El argumento \func{id} es un indicador de dispositivo hardware. Se trata de un argumento opcional para tarjetas de sonido con \func{id} 0. El argumento \func{adaptador} requiere el nombre del adaptador de dispositivo hardware. A continuación, en el \cref{tab:adaptors} se muestra una relación con los adaptadores de dispositivo cuyo uso es más frecuente y el nombre de adaptador que debe introducirse como argumento de la llamada a \func{analoginput}. Por conveniencia se ha añadido Keithley a esta lista.

\begin{table}
	\centering
	\begin{tabular}{l >{\tt\qquad}l}
		\toprule
		\multicolumn{1}{c}{Fabricante de Hardware} & \multicolumn{1}{c}{\rm Nombre de adaptador} \\
		\midrule
		Advantech & advantech \\
		\midrule
		Measurement Computing & mcc \\
		\midrule
		National Instruments & nidaq \\
		\midrule
		Parallel port & parallel \\
		\midrule
		Microsoft Windows & winsound \\
		\midrule
		Keithley Instruments, Inc. & keithley \\
		\bottomrule
	\end{tabular}
	\caption[Argumento que debe emplearse en la llamada a \func{analoginput} en función del fabricante]{Argumento que debe emplearse en la llamada a \func{analoginput} en función del fabricante.}
	\label{tab:adaptors}
\end{table}


\subsubsection{Añadiendo canales}

Antes de poder utilizarse, deben añadirse canales al objeto dispositivo. Para ello, debe emplearse la función \func{addchannel}. Puede pensarse en un objeto dispositivo como un contenedor de grupos de canales y en los canales añadidos a un objeto dispositivo como un grupo de canales. Si se desean añadir dos canales al objeto dispositivo \func{objeto} puede utilizarse la siguiente llamada \func{cans = addchannel(objeto, 1:2);}.


\subsubsection{Configurando propiedades}

Puede controlarse el comportamiento de una sesión de adquisición de datos o de una aplicación creada con tal propósito configurando las propiedades de los objetos dispositivo que intervienen en el proceso de adquisición y de los canales que dicho objeto contiene. Estas son las reglas principales en la configuración de propiedades desde la \datx{}.

\begin{itemize}
	\item Los nombres de las propiedades pueden escribirse en mayúsculas, minúsculas o combinación de ambas.
	\item Los nombres de las propiedades pueden abreviarse como se mostrará a continuación. % Hay que confirmar que se explican las reglas de abreviatura
	\item La función \func{set} aplicada a un objeto dispositivo ---\func{set(objeto)}--- devuelve todas las propiedades configurables de ese objeto. Si se llama a \func{set} utilizando como argumento un canal ---\func{set(objeto.Channel(índice)}---, la función devolverá todas las propiedades configurables de dicho canal.
	\item La función \func{get} devuelve todas las propiedades de un canal u objeto y el valor que toman en el momento en el que se llama a la función si se emplea como único argumento dicho canal u objeto ---\func{get(objeto)}, \\ \func{get(objeto.Channel(índice)}---.
\end{itemize}

Se distinguen dos tipos de propiedades distintas asociadas a los canales contenidos en un objeto dispositivo.

\begin{description}
	\item[Propiedades comunes] Son aquellas propiedades que se aplican todos los canales pertenecientes a un mismo objeto dispositivo.
	\item[Propiedades de canal] A diferencia de las propiedades comunes, las propiedades de canal pueden configurarse individualmente por canal.
\end{description}

Dentro de las propiedades comunes de los canales existen las \emph{propiedades básicas}, que se aplican a todos los subsistemas de un determinado tipo (\textsc{ai, ao, dio}); y \emph{propiedades específicas de dispositivo} aplicables únicamente al hardware específico que se está empleando.\par
Existen tres formas de configurar u obtener el valor de una propiedad: utilizando las funciones \func{set} y \func{get}; empleando la notación de punto; o recurriendo a los nombres indexados.

\begin{itemize}
	\item La sintaxis de las funciones \func{get} y \func{set} es similar a la empleada en la herramienta de \matlab{} \emph{Handle Graphics}.
		\begin{lstlisting}[gobble=16]
			out = get(objeto, `SampleRate');
			set(objeto, `SampleRate', 11025)
		\end{lstlisting}
	\item La notación de punto se emplea del siguiente modo:
		\begin{lstlisting}[gobble=16]
			out = objeto.SampleRate;
% 			objeto.SampleRate = 11025;
		\end{lstlisting}
	\item Por último, los nombres indexados permiten asociar un nombre descriptivo a cada canal. Por ejemplo para asociar el nombre \func{Can1} con el primer canal contenido en \func{objeto} debe procederse como se enuncia a continuación.
		\begin{lstlisting}[gobble=16]
			set(objeto.Channel(1), `ChannelName', `Can1');
			out = objeto.Can1.UnitsRange;
			objeto.Can1.UnitsRange = [0, 10];
		\end{lstlisting}
\end{itemize}


\subsubsection{Adquisición de datos}

La adquisición de datos puede dividirse en tres tareas básicas: iniciar el objeto dispositivo; registrar datos y detener el objeto dispositivo.\par
La función que se utiliza para iniciar un objeto dispositivo es la función \func{start}, p.e. para iniciar el objeto dispositivo \func{objeto} habría que llamar a la función de esta forma \func{start(objeto)}. Tras iniciar un objeto su propiedad \textsf{Running} pasa de manera automática al valor \textsf{On}.\par
No obstante haber iniciado el dispositivo, este no empieza a registrar datos hasta que no ocurre un trigger o disparo. Hay diversos tipos de trigger, en el \vref{tab:triggers} se muestran aquellos soportados por todos los dispositivos. Tras un trigger el dispositivo hardware inicia la adquisición de datos y la propiedad \textsf{Logging} del objeto dispositivo asociado conmuta al estado \textsf{On}.

\begin{table}
	\centering
	\begin{tabulary}{.9\linewidth}{>{\sf}c L}
		\toprule
		{\rm Tipo de disparo} & \multicolumn{1}{c}{Descripción} \\
		\midrule
		Inmediato & El disparo ocurre justo después de la llamada a \func{start}. Este es el tipo de trigger predeterminado. \\
		\midrule
		Manual & El disparo ocurre después de llamar manualmente a la función \func{trigger}. \\
		\midrule
		Software & El disparo sucede cuando se detecta una señal que satisface una determinada condición especificada de antemano. El objeto dispositivo debe disponer de más de un canal que hará las veces de la señal de disparo. Debe especificarse, como es obvio, que canal actúa como fuente del disparo. \\
		\midrule
		Reloj interno & Por añadidura, la \kpci{} cuenta con la posibilidad de recibir el disparo de la fuente de reloj interna. \\
		\bottomrule
	\end{tabulary}
	\caption[Tipos de disparo soportados por el hardware compatible con \matlab{}]{Tipos de disparo soportados por el hardware compatible con \matlab{} y una breve descripción de los mismos.}
	\label{tab:triggers}
\end{table}

Por último, existen tres causas por las que un objeto dispositivo puede detenerse: \matlab{} detiene un objeto dispositivo iniciado una vez obtenidos los datos precisados por el usuario; al ocurrir un error de tiempo de ejecución en relación con la actividad de un objeto dispositivo éste es detenido también; y tan sólo resta el método manual, que consiste en llamar a la función \func{stop}, por ejemplo \func{stop(objeto)}.\par
Como se ha mencionado la máquina virtual de adquisición de datos registra y controla los datos que extrae de un objeto dispositivo. Un usuario puede acceder a esos datos de dos formas diferentes:

\begin{itemize}
	\item La primera de ellas se conoce como previsualizar los datos. Se emplea con ese propósito la función \func{peekdata}. Si, por ejemplo, se quisiese previsualizar 1000 muestras obtenidas con el objeto dispositivo \func{objeto}, la llamada a \func{peekdata} sería la siguiente: \func{out = peekdata(objeto, 1000);}. La función \func{peekdata} devuelve el control a \matlab{} de inmediato y no elimina los datos previsualizados de la máquina virtual de adquisición.
	\item En cualquier momento tras adquirir datos mediante un objeto dispositivo estos pueden extraerse de la máquina virtual de adquisición mediante la función \func{getdata}. Partiendo del ejemplo anterior, si se desea extraer 1000 muestras procedentes del objeto dispositivo \func{objeto}, esta es la llamada adecuada \func{out = getdata(objeto, 1000);}. Al contrario que la función \func{peekdata}, \func{getdata} no devuelve el control a \matlab{} hasta haber extraído todas las muestras solicitadas. Es evidente que las muestras extraídas dejarán de estar disponibles en la máquina virtual de adquisición.
\end{itemize}

Es importante señalar que en cualquiera de los procedimientos descritos intentar acceder a más datos de los obtenidos en un determinado momento causará un error que detendrá el funcionamiento del objeto dispositivo.


\subsubsection{Eventos y Callbacks}

Puede decirse que un evento sucede en un determinado instante después de que una cierta condición se cumple. A menos que ocurra un error, en todas las sesiones de adquisición de datos debe producirse un evento de inicio, uno de disparo y uno de parada. Puede accederse a la información que transporta un evento mediante la propiedad \textsf{EventLog}:

\begin{lstlisting}
	Events = ai.EventLog;
	EventTypes = {Events.Types}
	EventTypes =
		`Start'    `Trigger'	`Stop'
\end{lstlisting}

Cuando se produce un evento, puede ejecutarse una función de \emph{callback}. Es posible seleccionar una función para un callback especificando como valor de la propiedad asociada a dicho callback el nombre de la función (si ésta se encuentra en el mismo fichero *.m que contiene el código que ejecuta la aplicación que realiza la adquisición de datos), o el nombre del fichero *.m con el código de la función. Así mismo, pueden pasarse argumentos de entrada a la función de callback asignándolos a la mencionada propiedad.\par
Por ejemplo, los siguientes comandos configuran \func{objeto} de forma que la función \func{datadqcallback} se ejecute desde el fichero cuyo nombre está compuesto por una raíz idéntica al nombre de la función y con extensión *.m, cuando se produzca un evento de trigger o de parada durante la actividad del objeto dispositivo. Además se pasa como argumento de la función el valor de la propiedad \textsf{Running} de \func{objeto} en el momento del callback.

\begin{lstlisting}
	set(objeto, `TriggerFcn', @datadqcallback, objeto.Running)
	set(objeto, `StopFcn', @datadqcallback, objeto.Running)
\end{lstlisting}

Un segundo ejemplo, este extraído del código fuente de la aplicación de control muestra como pasar argumentos a la función de callback y cuál es la sintaxis de la definición de la misma en el \cref{cod:callback}.

\begin{lstlisting}[style=displayed, caption={[Configuración de \emph{callback}]{Configuración de \emph{callback} para responder a eventos en la sesión de muestreo, la función de \emph{callback} recibe un argumento.}}, label={cod:callback}]
	set(handles.ai, 'TriggerType', 'Immediate', 'TimerFcn', '', ...
		'SamplesAcquiredFcn', {@localDaqCallback, gcbo});

				[...]

	function localDaqCallback(obj, event, hObject)
		handles = guidata(hObject);
		EventType = event.Type;

		switch lower(EventType)
			case 'samplesacquired'

				[...]

			case 'timer'

				[...]

		end
\end{lstlisting}

\subsubsection{Suprimiendo y borrando las trazas de los objetos dispositivo}

La función \func{delete} elimina el objeto dispositivo especificado de la máquina virtual de adquisición, pero no del espacio de trabajo de \matlab{}, ---\func{delete(objeto)}---. Tras una llamada semejante \func{objeto} sigue apareciendo en el espacio de trabajo de \matlab{}, pero se trata de un objeto inválido desde el momento en el que deja de encontrarse ligado al hardware. Deben suprimirse los objetos dispositivo faltos de validez con el comando \func{clear}, p.e., \func{clear objeto}.\par
Si se suprime un objeto dispositivo del espacio de trabajo de \matlab{} no deja de existir en la máquina virtual. Para poder recuperar objetos borrados accidentalmente puede utilizarse la función \func{daqfind}.

\begin{lstlisting}
	out = daqfind;	ai = out(1);
\end{lstlisting}
