\chapter{Introducci�n}

Una de las aplicaciones industriales de los ultrasonidos es la caracterizaci�n interna de materiales. De entre las distintas t�cnicas empleadas con dicha finalidad, la inspecci�n mediante ultrasonidos ha demostrado encontrarse entre las m�s fiables. Adem�s, debido a su naturaleza esta t�cnica es, obviamente, no destructiva. El procedimiento de medida habitual consiste en hacer incidir un haz de ondas de alta frecuencia ---generalmente entre los 40 KHz y 25 MHz--- sobre la superficie de la muestra, posteriormente se registra la onda que atraviesa el material y se miden uno o varios de sus par�metros. De este modo es posible determinar la presencia de defectos tales como grietas, o la inclusi�n de materiales extra�os, en el interior de la muestra.


\section*{Objetivos del proyecto fin de carrera}\label{sec:goals}

Con objeto de explotar esta tecnolog�a, principalmente en la caracterizaci�n de bloques de madera de palmera, este proyecto pretende inicialmente la creaci�n de un sistema electr�nico de medida. Este sistema constar� de dos partes diferenciadas: 
\begin{itemize}
	\item Un sistema de adquisici�n y procesado de se�ales.
	\item Y el conjunto formado por un transmisor y un sensor de ultrasonidos junto con sus respectivos circuitos acondicionadores.
\end{itemize}


\section*{Estructura de este documento}

Debido a la independencia entre los dos bloques de que se compone el sistema de medida, se ha cre�do conveniente dividir este documento en dos partes distintas. En la primera parte se expone el procedimiento seguido durante el acondicionamiento del sistema de adquisici�n y procesado. La segunda se ocupa de dar unas breves nociones de teor�a general de ultrasonidos, justificar cual debiera ser el equipo �ptimo para realizar los experimentos, y discutir los resultados obtenidos.
