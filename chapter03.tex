\chapter{Aplicaci�n para el control de la tarjeta}

\section{Introducci�n}

Desde los inicios de este proyecto lo que se ha pretendido es emplear un sistema de adquisici�n de se�ales como un osciloscopio digital. Para ello son necesarios la tarjeta de adquisici�n y un software que permita al usuario controlar el dispositivo desde una interfaz intuitiva y visualizar los resultados de forma sencilla. El lenguaje o, mejor dicho, plataforma que se ha empleado para el desarrollo de dicho software es \matlab{}, la raz�n, su inmejorable compatibilidad con el hardware disponible. A partir de ah�, los componentes de \matlab{} que se han empleado para conseguir que el software de control gozase de las caracter�sticas previstas por los objetivos del proyecto son dos: el entorno de desarrollo de interfaces gr�ficas de usuario (\textsc{gui}) de \matlab{}, m�s conocido como \textsc{guide} (\emph{graphical user interface development environment}); y el \emph{Data Acquisition Toolbox} de \matlab{}. El primero de ellos se ha empleado para, como su nombre indica, crear la interfaz que comunica el dispositivo con el usuario. Esta comunicaci�n se hace a trav�s del segundo de los componentes mencionados, �ste permite, mediante comandos de \matlab{} que pueden incluirse en rutinas o llamarse por separado, manejar el dispositivo ---convocarlo a muestrear, configurar sus propiedades--- y administra autom�ticamente los resultados almacen�ndolos en b�ffers situados en la memoria vol�til del ordenador, haci�ndolos de este modo accesibles al administrador de la tarjeta.\par % por un espacio de tiempo limitado
Como se ha mencionado en el p�rrafo anterior, el modelo de funcionamiento utilizado como referencia en el desarrollo de la aplicaci�n de control es el de un osciloscopio. Por ello, y en base a las conclusiones extra�das durante la realizaci�n de este proyecto, se ha cre�do conveniente remarcar un aspecto del funcionamiento habitual de cualquier osciloscopio. La tasa de refresco del monitor de un osciloscopio se dise�a para que los datos que aparecen en pantalla cambien lo suficientemente r�pido como para que el ojo humano experimente una variaci�n continua.
% Tengo que poner �nfasis en el hecho de que la configuraci�n del tama�o de ventana hace que el osciloscopio se comporte de una forma distinta, procesando la informaci�n antes de salir por pantalla o sacando datos sin m�s
Para ello se limita la cantidad de datos que se almacena en los b�ffers de entrada y se procesa antes de mostrar por pantalla.
